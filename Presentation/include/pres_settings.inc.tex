% !TeX root = ../connecting_pres.tex
\usepackage[utf8]{inputenc}
\usepackage[T1]{fontenc}
\usepackage{lmodern}
\usepackage{times}
\usepackage[american]{babel}
\usepackage{amsmath}
\usepackage{xcolor}
\usepackage{tikz}
\usepackage{listings}
\usepackage{amsfonts}
\usepackage{bm}
\usepackage{subfig}
\usepackage{booktabs}
\usepackage{amssymb}


%%Tikz
\usetikzlibrary{shapes,arrows,positioning}
%% use UOS theme
\usetheme{UOS}
%\usetheme[uosyellow]{UOS}

%======================================================================
%	Settings

%% where to find the graphics
\graphicspath{{img/}}

\newcommand{\source}[1]{\caption*{\textcolor{uos-grey-full}{Source: {#1}}} }
\newcommand{\RM}[1]{\MakeUppercase{\romannumeral{} #1{}}}
\newcommand{\code}[1]{\texttt{#1}}
\newcommand{\bull}[0]{\textbullet{}}
\newcommand{\tabitem}{{\color{uos-red-full}$\blacksquare$}}

\lstset{ %
  language=XML,                  % the language of the code
  basicstyle=\scriptsize,       % the size of the fonts that are used for the code
  numbers=left,                   % where to put the line-numbers
  numberstyle=\tiny\color{gray},  % the style that is used for the line-numbers
  stepnumber=1,                   % the step between two line-numbers. If it's 1, each line
                                  % will be numbered
  numbersep=5pt,                  % how far the line-numbers are from the code
  backgroundcolor=\color{uos-grey-full},  % choose the background color. You must add \usepackage{color}
  showspaces=false,               % show spaces adding particular underscores
  showstringspaces=false,         % underline spaces within strings
  showtabs=false,                 % show tabs within strings adding particular underscores
  frame=single,                   % adds a frame around the code
  rulecolor=\color{black},        % if not set, the frame-color may be changed on line-breaks within not-black text (e.g. commens (green here))
  tabsize=4,                      % sets default tabsize to 2 spaces
  captionpos=b,                   % sets the caption-position to bottom
  breaklines=true,                % sets automatic line breaking
  breakatwhitespace=false,        % sets if automatic breaks should only happen at whitespace
  title=\lstname,                 % show the filename of files included with \lstinputlisting;
                                  % also try caption instead of title
  keywordstyle=\color{blue},          % keyword style
  commentstyle=\color{green},       % comment style
  stringstyle=\color{black}         % string literal style
}

% ---------------------------------------------------------------------

%% insert outline at begin of every section
\AtBeginSection[]{
  \frame{
    \frametitle{\iflanguage{ngerman}{Gliederung}{Outline}}
    \tableofcontents[current, currentsubsection]
  }
}

%% format: \title[short title]{long title}
%%   - short title will be used in foot line
%%   - long title will be used on title page
\title[MPTCP Scheduling]{MPTCP - Scheduling\\A brief overview of algorithms, their strengths and areas of application.}

%% format: \author[short name]{long name}
%%   - short name will be used in foot line
%%   - long name will be used on title page
\author[Henrik Gerdes]{Henrik Gerdes\\ {\scriptsize hegerdes@uos.de}}

\date{\today}
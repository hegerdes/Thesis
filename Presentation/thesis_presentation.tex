\documentclass{beamer}

% !TeX root = ../thesis_presentation.tex
\usepackage[utf8]{inputenc}
\usepackage[T1]{fontenc}
\usepackage{lmodern}
\usepackage{times}
\usepackage[american]{babel}
\usepackage{tikz}
\usepackage{listings}
\usepackage{subcaption}
\usepackage{bm}
\usepackage{booktabs}
\usepackage{amsfonts}
\usepackage{amsmath}
\usepackage{amssymb}
\usepackage{amsthm}
\usepackage{marvosym}
\let\marvosymLightning\Lightning


%	Settings
%======================================================================
%%Tikz
\usetikzlibrary{shapes,arrows,calc,positioning,fit}

%% THEME: UOS-red or UOS-yellow
\usetheme{UOS}
% \usetheme[uosyellow]{UOS}

%% where to find the graphics
\graphicspath{{img/}}

%% Command setup
\newcommand{\source}[1]{\caption*{\textcolor{uos-grey-full}{Source: {#1}}} }
\newcommand{\RM}[1]{\MakeUppercase{\romannumeral{} #1{}}}
\newcommand{\code}[1]{\texttt{#1}}
\newcommand{\bull}[0]{\textbullet{}}
\newcommand{\tabitem}{{\color{uos-red-full}$\blacksquare$}}

% ---------------------------------------------------------------------
%% insert outline at begin of every section
\AtBeginSection[]{
  \frame{
    \frametitle{\iflanguage{ngerman}{Gliederung}{Outline}}
    \tableofcontents[current, currentsubsection]
  }
}

%% format: \title[short title]{long title}
%%   - short title will be used in foot line
%%   - long title will be used on title page
\title[Development with DevContainers]{Analysis of Adopting DevOps Tools for a Homogeneous Production and Development Environment}

% Color
\definecolor{codebg}{HTML}{EEEEEE}
\definecolor{dkgreen}{rgb}{0,0.6,0}
\definecolor{gray}{rgb}{0.5,0.5,0.5}
\definecolor{mauve}{rgb}{0.58,0,0.82}
\definecolor{LightCyan}{rgb}{0.88,1,1}



%======================================================================
% * * * * * * *   T E X T    S T A R T S    H E R E   * * * * * * * * *
%======================================================================
\begin{document}

\begin{frame}[plain]
  \titlepage{}
\end{frame}

\begin{frame}[allowframebreaks]{\iflanguage{ngerman}{Inhaltsverzeichnis}{Table of Contents}}
  \tableofcontents
\end{frame}

%%%%%%%%%%%%%%%%%%%%%%%%%%%%%%%%%%%%%%%%%%%%%%%%%%%%%%%%%%%%%%%%%%%%%%%%%%%%%%%%
% NOTES:
% Starten mit was aktuellem: MPTCP is here
% Korea ISPs nutzen es. Erfolgreich
% Croatian Hrvatski Telekom Hybrid access
% OVH bietet eigene Services mit MPTCP an
% Linux und SOCKS wachsen und ist einfach zu benutzen
%%%%%%%%%%%%%%%%%%%%%%%%%%%%%%%%%%%%%%%%%%%%%%%%%%%%%%%%%%%%%%%%%%%%%%%%%%%%%%%%
\section{MPTCP introduction}
\subsection{State of MPTCP}
\begin{frame}{MPTCP IS HERE!}
  \vspace{-0.3cm}
  \begin{block}{\normalsize Korean \& Hrvatski Telecom enabled MPTCP}
    \begin{itemize}
      \small
      \setlength\itemsep{0em}
      \item Android smartphones can bond WiFi and LTE
      \item Allows download speeds of 800 Mbps
      \end{itemize}
  \end{block}

  %\pause

  \begin{block}{\normalsize French ISP OVH 2015}
    \begin{itemize}
      \small
      \setlength\itemsep{0em}
      \item OverTheBox service is MPTCP capable
      \item Bond several DSL cable links
    \end{itemize}
  \end{block}

  %\pause

  \begin{block}{\normalsize Other}
    \begin{itemize}
      \small
      \setlength\itemsep{0em}
      \item Apples SIRI
      \item Several MPTCP Linux implementations
      \item Deployable for everyone with SOCKS
    \end{itemize}
  \end{block}
\end{frame}


%%%%%%%%%%%%%%%%%%%%%%%%%%%%%%%%%%%%%%%%%%%%%%%%%%%%%%%%%%%%%%%%%%%%%%%%%%%%%%%%
% NOTES:
% Geschichte über einen größeren überwachten MPTCP Datensatz
% Rahmenbedingungen Katholische UNI Louvain Belgien
% Dutzend 12 Clients im täglichen gebrauch
% MPTCP 0.89 und Android 4.4, SOCKS und Traffic gecaptured
% 6 Wochen
%%%%%%%%%%%%%%%%%%%%%%%%%%%%%%%%%%%%%%%%%%%%%%%%%%%%%%%%%%%%%%%%%%%%%%%%%%%%%%%%
\begin{frame}{A First Analysis of Multipath TCP on Smartphones}
  \begin{columns}

    \begin{column}{0.48\textwidth}
      \begin{itemize}
        \setlength\itemsep{0.8em}
        \item Catholic University of Louvain
        \item Real word usage with a dozen of users
        \item Linux MPTCP v0.89.5 and Android 4.4
        \item Android with ShadowSocks
      \end{itemize}
    \end{column}

    %\pause

    \begin{column}{0.48\textwidth}
      \begin{itemize}
        \setlength\itemsep{0.8em}
        \item SOCKS proxy with tcpdump
        \item Belgium from March 8th to April 28th 2015
        \item Total traffic of 25.4 GB in test period
        \item Over 390,782 MPTCP connections
      \end{itemize}
    \end{column}
  \end{columns}
  \vfill
\end{frame}


%%%%%%%%%%%%%%%%%%%%%%%%%%%%%%%%%%%%%%%%%%%%%%%%%%%%%%%%%%%%%%%%%%%%%%%%%%%%%%%%
% NOTES:
% Ergebnisse: Über 50% der fälle konnte nur ein Interface genutzt werden
% LANGE VERBINDUNG
% PORTS: DNS;WEB;SSL;SOPTYFY;GOOGLESERVICES;CHROMECAST;
% 65% of the observed connections last less than 10 s.
% 86.9% carry less than 10 KBytes
% 450 MBytes and was spread over
% A connection establishing 42 different subflows was observed.
%%%%%%%%%%%%%%%%%%%%%%%%%%%%%%%%%%%%%%%%%%%%%%%%%%%%%%%%%%%%%%%%%%%%%%%%%%%%%%%%
\begin{frame}{Data analysis}
  \vspace{-0.2cm}
  \begin{columns}[totalwidth=\textwidth]
    \begin{column}{\textwidth}
      \begin{table}
        \scalebox{0.7}{
        \begin{tabular}{@{}l|l|l|l|l|l|l@{}}
          \toprule
          Number of subflows        & 1       & 2       & 3      & 4      & 5      & \textgreater{}5 \\ \midrule\midrule
          Percentage of connections & 67.75\% & 29.96\% & 1.07\% & 0.48\% & 0.26\% & 0.48\%          \\ \bottomrule
        \end{tabular}
        }
        \caption{\footnotesize Number of subflows per MPTCP connection, \textcolor{uos-grey-full}{Source:\cite{realMPTCP}}}
      \end{table}
    \end{column}
  \end{columns}

  %\pause

  \vspace{-0.5cm}
  \begin{table}[]
    \scalebox{0.7}{
      \begin{tabular}{@{}l|l|l|l|l@{}}
      \toprule
      \textbf{Port}   & \textbf{\# connections} & \textbf{\% connections} & \textbf{Bytes}     & \textbf{\% bytes}       \\ \midrule\midrule
      53     & 107,012        & 27.4           & 17.4 MB   & \textless{}0.1  \\ \midrule
      80     & 103,597        & 26.5           & 14,943 MB & 58.8           \\ \midrule
      443    & 104,223        & 26.7           & 9,253 MB  & 36.4           \\ \midrule
      4070   & 571            & 0.1            & 91.7 MB   & 0.4            \\ \midrule
      5228   & 10,602         & 2.7            & 27.3      & 0.1            \\ \midrule
      8009   & 10,765         & 2.8            & 0.97      & \textless{}0.1 \\ \midrule
      Others & 54,012         & 13.8           & 1.090 MB  & 4.4            \\ \bottomrule
      \end{tabular}\label{tab:traffic}
      }
      \caption{\footnotesize Statistics about destination port, \textcolor{uos-grey-full}{Source:~\cite{realMPTCP}}}
    \end{table}

  %\textcolor{uos-red-full}{\Large{\(\bm{\Rightarrow}\)}} Won't result in ONE good application
\end{frame}


\begin{frame}{MPTCP promises to: }
  \begin{columns}
    \begin{column}{\textwidth}
        \begin{center}
          Offer resilience \\
          \vspace{0.7cm}
          Increases bandwidth\\
          \vspace{0.7cm}
          Provide seamless handovers\\
          \vspace{0.7cm}
          Backwards compatible and deployable\\
          \vspace{0.6cm}
          \ldots \\
          \vspace{0.6cm}
        \end{center}
      \end{column}
    \end{columns}
\end{frame}


%%%%%%%%%%%%%%%%%%%%%%%%%%%%%%%%%%%%%%%%%%%%%%%%%%%%%%%%%%%%%%%%%%%%%%%%%%%%%%%%
% Netzwerke sind unterschiedlich
% RTT, Bandbreite, Zuverlässigkeit
%%%%%%%%%%%%%%%%%%%%%%%%%%%%%%%%%%%%%%%%%%%%%%%%%%%%%%%%%%%%%%%%%%%%%%%%%%%%%%%%
\section{MPTCP challenges}
\subsection{Asymmetric paths}
\begin{frame}{Asymmetric paths}

  \begin{center}
    \Large \color{uos-red-full} Multiple subflows have different characteristics in terms of:
  \end{center}
  \vspace{0.5cm}

  %\pause

  \begin{table}
    \centering
    \begin{tabular}{@{}l@{}}
      \large\tabitem{} Round Trip Time (Latency)\\[10pt]
      \large\tabitem{} Bandwidth (Throughput)\\[10pt]
      \large\tabitem{} Error rate (Reliability)
    \end{tabular}
  \end{table}
\end{frame}


%%%%%%%%%%%%%%%%%%%%%%%%%%%%%%%%%%%%%%%%%%%%%%%%%%%%%%%%%%%%%%%%%%%%%%%%%%%%%%%%
% Zwei Pfade mit RTT1 = 10x RTT2
% Angenommen CWND 2
%%%%%%%%%%%%%%%%%%%%%%%%%%%%%%%%%%%%%%%%%%%%%%%%%%%%%%%%%%%%%%%%%%%%%%%%%%%%%%%%
\subsection{Blocking}
\begin{frame}{Blocking}
  \begin{figure}
    \centering
    % \includegraphics[width=0.85\textwidth]{blocking.png}\label{fig::block}
    \caption{\footnotesize Head-of-line blocking and receive buffer blocking, \textcolor{uos-grey-full}{Source: {\cite{blocking}}}}
  \end{figure}
\end{frame}


\section{MPTCP scheduler}
\begin{frame}{MPTCP scheduler}

  \begin{enumerate}
    \setlength\itemsep{1.6em}
    \item Lowest-RTT-First (LRF)
    \item BLock ESTimation (BLEST)
    \item Shortest Transfer Time First (STTF)
    \item Earliest Completion First (ECF)
    \item Additional schedulers
  \end{enumerate}
\end{frame}


%%%%%%%%%%%%%%%%%%%%%%%%%%%%%%%%%%%%%%%%%%%%%%%%%%%%%%%%%%%%%%%%%%%%%%%%%%%%%%%%
% Standard
% Füllt CWND und dann nächster. Gut für symmetrische pfade
% Berechnung RTT
% RTO Retransmission time out
%%%%%%%%%%%%%%%%%%%%%%%%%%%%%%%%%%%%%%%%%%%%%%%%%%%%%%%%%%%%%%%%%%%%%%%%%%%%%%%%
\subsection{LRF}
\begin{frame}{Lowest-RTT-First}
  \begin{columns}
    \begin{column}{0.48\textwidth}
      \begin{enumerate}
        \setlength\itemsep{1.2em}
        \item Uses the subflow with the lowest RTT
        \setcounter{enumi}{2}
        \item Use next lowest RTT
      \end{enumerate}
    \end{column}

    \begin{column}{0.48\textwidth}
      \begin{enumerate}
        \setlength\itemsep{1.2em}
        \setcounter{enumi}{1}
        \item Until congestion window is filled
        \setcounter{enumi}{3}
        \item Standard Linux scheduler
      \end{enumerate}
    \end{column}
  \end{columns}

  \vspace{0.8cm}
  {\color{uos-red-full}\rule{\textwidth}{1.5pt}}
  %\pause

  \begin{table}[]
    \begin{tabular}{@{}lll@{}}
      \(err\)             & \(=\)& \(measured_{RTT} - predic\)                                \\
      \(predic_{new}\)    & \(=\)& \(predic_{old} + \frac{1}{8} * err\)                       \\
      \(variation_{new}\) & \(=\)& \(\frac{3}{4} * variation_{old} + \frac{1}{4} * \|err\|\)  \\
      \(RTO\)             & \(=\)& \(predic + 4 * variation\)
    \end{tabular}
    \caption{\small Jacobson’s Algorithm}
  \end{table}
\end{frame}

%%%%%%%%%%%%%%%%%%%%%%%%%%%%%%%%%%%%%%%%%%%%%%%%%%%%%%%%%%%%%%%%%%%%%%%%%%%%%%%%
% Anwendungen die geringe Latenz brauchen VoIP, Google Maps, Gaming
% MSS Max Segment Size
% Skips subflows
% Mehr Metriken
%%%%%%%%%%%%%%%%%%%%%%%%%%%%%%%%%%%%%%%%%%%%%%%%%%%%%%%%%%%%%%%%%%%%%%%%%%%%%%%%
\subsection{BLEST}
\begin{frame}{BLock ESTimation scheduler}
  \begin{columns}
    \begin{column}{0.48\textwidth}
      \begin{itemize}
        \setlength\itemsep{1.2em}
        \item Specifically designed for asymmetric paths
        \item Estimate the amount of blocking
      \end{itemize}
    \end{column}

    \begin{column}{0.48\textwidth}
      \begin{itemize}
        \setlength\itemsep{1.2em}
        \item Additional send window on the connection level
        \item Added to Linux MPTCP in June 2019
      \end{itemize}
    \end{column}
  \end{columns}

  %\pause

  \vspace{0.6cm}
  \begin{center}
    Skips subflow if:
    \begin{equation}
      X \times \lambda > |MPTCP_{SW}| - MSS_S * (\text{\textit{inflight}}_S + 1)
    \end{equation}\small
    \begin{equation}
      X=MSS_F * \left(CWND + \left(\frac{RTT_S}{RTT_F} -1 \right)/ 2 \right) \times \frac{RTT_S}{RTT_F}
    \end{equation}
  \end{center}
\end{frame}

%%%%%%%%%%%%%%%%%%%%%%%%%%%%%%%%%%%%%%%%%%%%%%%%%%%%%%%%%%%%%%%%%%%%%%%%%%%%%%%%
% Kümmert sich nur um RTT (Smoothed RTTT)
% Schnelles Rescheduling von ALLEN Paketen
% Aufwendig
% Zeit Berechnung im Anhang wenn später noch Zeit ist
%%%%%%%%%%%%%%%%%%%%%%%%%%%%%%%%%%%%%%%%%%%%%%%%%%%%%%%%%%%%%%%%%%%%%%%%%%%%%%%%
\subsection{STTF}
\begin{frame}{Shortest Transfer Time First}
  \begin{columns}
    \begin{column}{0.48\textwidth}
      \begin{itemize}
        \setlength\itemsep{1.2em}
        \item Similar approach as LRF
        \item Considers enqueued packets
        \item Quick response to network changes
      \end{itemize}
    \end{column}

    %\pause

    \begin{column}{0.48\textwidth}
      \begin{itemize}
        \setlength\itemsep{1.2em}
        \item Fast rescheduling
        \item Latency orientated scheduler
        \item Resource heavy and complex implementation
      \end{itemize}
    \end{column}
  \end{columns}
\end{frame}

%%%%%%%%%%%%%%%%%%%%%%%%%%%%%%%%%%%%%%%%%%%%%%%%%%%%%%%%%%%%%%%%%%%%%%%%%%%%%%%%
% Nicht Latenz orientiert
% Berechnet auf welchem flow der Paket am ehesten ankommen würde
% Wenn RTT_S kleiner ist wird S und F genutzt
%%%%%%%%%%%%%%%%%%%%%%%%%%%%%%%%%%%%%%%%%%%%%%%%%%%%%%%%%%%%%%%%%%%%%%%%%%%%%%%%
\subsection{ECF}
\begin{frame}{Earliest Completion First}
  \begin{columns}
    \begin{column}{0.48\textwidth}
      \begin{itemize}
        \setlength\itemsep{1.2em}
        \item Similar approach as LRF and STTF
        \item Is able to skip subflows for small amounts of data transmissions
      \end{itemize}
    \end{column}

    \begin{column}{0.48\textwidth}
      \begin{itemize}
        \setlength\itemsep{1.2em}
        \item Prioritizes throughput over latency
        \item Can also combine subflows for big transmissions
      \end{itemize}
    \end{column}
  \end{columns}

  %\pause

  \vspace{0.6cm}
  \begin{center}
    Skips subflow if:
    \begin{equation}
      RTT_F + \frac{k}{CWND_F}\times RTT_F\ \leq RTT_S
    \end{equation}
\end{center}
\end{frame}

%%%%%%%%%%%%%%%%%%%%%%%%%%%%%%%%%%%%%%%%%%%%%%%%%%%%%%%%%%%%%%%%%%%%%%%%%%%%%%%%
% Veranschaulichen wie sie funktionieren
% LRF, BLEST1, BLEST2, STTF
% 15 Pakete CWD 10 und RTT_2 = 10x RTT_1
% RLF switched schnell die anderen nicht
%%%%%%%%%%%%%%%%%%%%%%%%%%%%%%%%%%%%%%%%%%%%%%%%%%%%%%%%%%%%%%%%%%%%%%%%%%%%%%%%
\begin{frame}{Inner workings}
  \begin{figure}
    \centering
    % \includegraphics[width=0.9\textwidth]{eval_func.png}\label{fig::workings}
    \caption{\small Scheduling decisions for a burst of 15 segments, using the LRF, BLEST, and STTF schedulers, \textcolor{uos-grey-full}{Source: {\cite{blocking}}}}
  \end{figure}
\end{frame}

%%%%%%%%%%%%%%%%%%%%%%%%%%%%%%%%%%%%%%%%%%%%%%%%%%%%%%%%%%%%%%%%%%%%%%%%%%%%%%%%
% OTIAS Shortest Transfer time. Assumes that 1 way delay is RTT/2 Allows ensuing data. TransferTime = Transfertime + Inflight * RTT. Unable to adopt to changes
% DAPS Delay-Aware Packet Scheduler min  receive-buffer blocking by max probability of in-order reception. Just sorts the subflows not the packages
% RR wie CPU scheduler immer 10 auf Flow 1 und dann 10 auf Flow 2 ....
% Redundant flute auf allen
%%%%%%%%%%%%%%%%%%%%%%%%%%%%%%%%%%%%%%%%%%%%%%%%%%%%%%%%%%%%%%%%%%%%%%%%%%%%%%%%
\begin{frame}{Additional schedulers}
  \begin{itemize}
    \large
    \setlength\itemsep{1.2em}
    \item Out-of-Order Transmission for In-Order Arrival Scheduler (OTIAS)
    \item Delay-Aware Packet Scheduler (DAPS)
    \item RoundRobin (RR)
    \item Redundant Scheduler
  \end{itemize}
\end{frame}

\section{MPTCP evaluation}
\subsection{Functional}
\begin{frame}{Schedulers in short}
  \begin{center}
    \begin{enumerate}
      \setlength\itemsep{1.2em}
      \item LRF und RR are not suitable for asymmetric paths\\
      \item LRF does not consider the relative subflows RTT difference\\
      \item BLEST and STTF both prioritize latency. Both are able to skip subflows\\
      \item STTF is complex and heavy\\
      \item ECF improves utilization of the fastest subflow. Prioritizes throughput\\
    \end{enumerate}
  \end{center}
\end{frame}

\subsection{Benchmarks}
\begin{frame}{}
  \begin{enumerate}
    \setlength\itemsep{2.5em}
    \Large
    \item \textcolor{uos-red-full}{\textbf{Controlled latency analysis}}
    \item \textcolor{uos-red-full}{\textbf{Controlled bandwidth analysis}}
    \item \textcolor{uos-red-full}{\textbf{Real-World analysis}}
  \end{enumerate}
\end{frame}

%%%%%%%%%%%%%%%%%%%%%%%%%%%%%%%%%%%%%%%%%%%%%%%%%%%%%%%%%%%%%%%%%%%%%%%%%%%%%%%%
% Vergleich der suboptimalen Scheduler: Als Basisline
% Symmetrie is LRF gut
% Asymmetrie ist ECF meist etwas schneller
% Ansonsten sehr ähnliche Ergebnisse
% Soll nur zum vergleich für die nächsten Folie dienen
%%%%%%%%%%%%%%%%%%%%%%%%%%%%%%%%%%%%%%%%%%%%%%%%%%%%%%%%%%%%%%%%%%%%%%%%%%%%%%%%
\begin{frame}{Performance baseline}
  \begin{columns}
    \vspace{-2cm}
    \begin{column}{0.38\textwidth}
      \small
        \bull{} Results for LRF, DAPS, OTIAS, ECF\\
        \vspace{0.3cm}
        \bull{} Figure a): WLAN/3G setup \\
        \vspace{0.3cm}
        \bull{} Figure b): WLAN/WLAN setup\\
        \vspace{0.3cm}
        \bull{} All schedulers perform similarly with marginal differences in a)\\
        \vspace{0.3cm}
        \bull{} Increased page load times for DAPS and OTIAS in b)\\
    \end{column}
    \begin{column}{0.58\textwidth}
      \vspace{-0.8cm}
      \begin{figure}
        \centering
        % \includegraphics[width=0.90\textwidth]{default_sched_base.jpg}\label{fig::base}
        \vspace{-0.2cm}
        \caption{\footnotesize Page load time for web traffic, \textcolor{uos-grey-full}{Source: {\cite{lowlatschedulers}}}}
      \end{figure}
    \end{column}
  \end{columns}
\end{frame}


%%%%%%%%%%%%%%%%%%%%%%%%%%%%%%%%%%%%%%%%%%%%%%%%%%%%%%%%%%%%%%%%%%%%%%%%%%%%%%%%
% Künstliches Testsetup für Latenz
% Jetzt schneidet LRF schlechter ab. Bei steigender Load und wachsender Asymmetrie
% Schlechter als TCP
% BLEST und STTF sehr ähnlich, wobei STTF etwas schneller meist
% Denke an letzte Folie zum vergleich
%%%%%%%%%%%%%%%%%%%%%%%%%%%%%%%%%%%%%%%%%%%%%%%%%%%%%%%%%%%%%%%%%%%%%%%%%%%%%%%%
\begin{frame}{Latency}
  \begin{figure}
    \centering
    % \includegraphics[width=1.05\textwidth]{eval_sync_async.png}\label{fig::latency}
    \caption{\small Average transmission time for schedulers, \textcolor{uos-grey-full}{Source: {\cite{lowlatschedulers}}}}
  \end{figure}
\end{frame}


%%%%%%%%%%%%%%%%%%%%%%%%%%%%%%%%%%%%%%%%%%%%%%%%%%%%%%%%%%%%%%%%%%%%%%%%%%%%%%%%
% Künstliches Testsetup für Durchsatz
% Steigende Asynchronität bei Bandbreite und unterschiedliche Übertragungen
% DAPS nicht so dolle. Keine Dateien haben kaum Einfluss
% ECF und BLEST erzielen im Vergleich gute Ergebnisse

% Zweite Folie hat auch STTF mit drin. Da sieht man das ab einer gewissen Asynchronität fast kein Scheduler mehr die Vorteile von zwei  Subflows mehr nutzen kann. Wenn die Differenz > 4 ist. Auffällig ist, das STTF die große Asynchronität besser handelt
%%%%%%%%%%%%%%%%%%%%%%%%%%%%%%%%%%%%%%%%%%%%%%%%%%%%%%%%%%%%%%%%%%%%%%%%%%%%%%%%
\begin{frame}[allowframebreaks]{Throughput}
  \vspace{-1.5cm}
  \begin{center}

    \begin{figure}
      % \includegraphics[width=1.06\textwidth]{ecf_dl.png}\label{fig::bandwith}
      \caption{\small Average Download Completion Time, \textcolor{uos-grey-full}{Source: {\cite{ECF}}}}
    \end{figure}

  \end{center}
  \begin{columns}
    \begin{column}{0.5\textwidth}
\begin{itemize}
  \item Increasing asynchrony
  \item WLAN 1 MBps and LTE 1 to 10 MBps
\end{itemize}
    \end{column}
    \begin{column}{0.5\textwidth}
      \begin{itemize}
        \item DAPS is always worse
        \item ECF fastest with BLEST second
      \end{itemize}
    \end{column}
  \end{columns}

  \begin{columns}
    \begin{column}{0.6\textwidth}
      \begin{figure}
        % \includegraphics[width=0.85\textwidth]{low_bulk.png}\label{fig::bulk}
        \caption{\small Average goodput for TCP and MPTCP (LRF, BLEST, and STTF), \textcolor{uos-grey-full}{Source: {\cite{lowlatschedulers}}}}
      \end{figure}
    \end{column}
    \begin{column}{0.4\textwidth}
      \vspace{-0.8cm}
      \begin{itemize}
        \item Shows intact of growing RTT asynchrony
        \item Throughput degrades drastically if \(\frac{\text{RTT}_2}{\text{RTT}_1} \leq4\)
        \item A single TCP connection can deliver higher throughput (on highly asymmetric paths)
      \end{itemize}
    \end{column}
  \end{columns}
\end{frame}



%%%%%%%%%%%%%%%%%%%%%%%%%%%%%%%%%%%%%%%%%%%%%%%%%%%%%%%%%%%%%%%%%%%%%%%%%%%%%%%%
% Real-World
% Page load time und Object load time
% Setting 25ms und 75ms
% Im Vergleich zu RRF kann BLEST Objekte 26% und STTF 32% schneller laden
% Page load time oft von der jeweiligen Internetseite abhängig und deren Aufbau
%%%%%%%%%%%%%%%%%%%%%%%%%%%%%%%%%%%%%%%%%%%%%%%%%%%%%%%%%%%%%%%%%%%%%%%%%%%%%%%%
\begin{frame}{Real-World}
  \begin{columns}
    \begin{column}{0.3\textwidth}
      \small
        \bull{} \(RTT_{WLAN} \approx 25ms\) \& \(RTT_{3G} \approx 75ms\)\\
        \vspace{0.2cm}
        \bull{} Object load times improved by up to 26\% (BLEST) and 32\% (STTF)\\
        \vspace{0.2cm}
        \bull{} Practical ECF test showed 26\% faster load of CNN-page\\
        \vspace{0.2cm}
        \bull{} ECF improved video streaming throughput by 16\% (WLAN-Spot \& LTE)
    \end{column}
    \begin{column}{0.7\textwidth}
      \vspace{-0.8cm}
      \begin{figure}
        \centering
        % \includegraphics[width=\textwidth]{low_lat_real_web.png}\label{fig::base}
        \caption{\small Page load and object times for HTTP2, \textcolor{uos-grey-full}{Source: {\cite{lowlatschedulers}}}}
      \end{figure}
    \end{column}
  \end{columns}
\end{frame}


\begin{frame}{Conclusion time}
  %\pause
  \begin{center}
    \begin{itemize}
      \setlength\itemsep{1.2em}
      \item \textcolor{uos-red-full}{\Large Best scheduler is \dots{} it depends}\\%\pause
      \item Know your use cases\\%\pause
      \item Default scheduler are not optimized for asymmetric paths\\%\pause
      \item BLEST is a promising scheduler for asymmetric paths\\%\pause
      \item Linux-MPTCP is flexible and easy adoptable\\%\pause
      \item Development is active and MPTCP evolves fast new MPTCP RFC 6884 revision (March 2020)
    \end{itemize}
  \end{center}
\end{frame}

\appendix


\begin{frame}[allowframebreaks]{References}
  \small
  \setbeamertemplate{bibliography item}[book]
  \bibliography{bib/mybib}{}
  \bibliographystyle{IEEEtranSA}
\end{frame}

\begin{frame}{Questions?}
  \begin{center}
    Thank you for your attention!

    Any questions?
  \end{center}
\end{frame}


\begin{frame}[fragile]{Shortest Transfer Time First}
  \begin{lstlisting}[frame=single, basicstyle=\tiny, caption={STTF pseudo code,  \textcolor{uos-grey-full}{Source: \cite{lowlatschedulers}}}, label=code::sttf]
  function TRANSFER_TIME:
      if cwnd_free > 0 and data_to_send < cwnd_free then
          return rtt / 2
      transfer_time = transfer_time + rtt
      cwnd = increase_cwnd(current_cc_state)

      if data_to_send <= max_segments_in_ss then
          transfer_time = transfer_time + rtt * (rounds_in_ss-1) + rtt/2
          return transfer_time
      else if cwnd < ssthresh then
          transfer_time = transfer_time + max_rounds_in_ss * rtt
      if ends_in_ss(data_to_send) then
          return transfer_time

      cwnd = ssthresh
      transmission_time += rtt * (rounds_in_ca - 1) + rtt / 2
    return transfer_time
  \end{lstlisting}
\end{frame}

\end{document}

% !TeX TXS-program:compile = txs:///pdflatex/[--shell-escape]
\documentclass[12pt, a4paper]{article}

% IMPORTANT
% Set status of the doc
\def\draft{draft}
\def\final{final}
\def\status{draft}

% Include setup stuff
% !TeX root = ../main.tex
\usepackage{a4wide}

\usepackage[utf8]{inputenc}

%\usepackage[ngerman]{babel}
\usepackage[english]{babel}
\usepackage[T1]{fontenc}
\usepackage{palatino}
\usepackage{graphicx}
\usepackage{caption}
\usepackage{url}
\usepackage{acronym}
\usepackage{tocloft}
\usepackage{mathpazo}
\usepackage{amsmath}
\usepackage{amsfonts}
\usepackage{adjustbox}
\usepackage{hhline}
\usepackage{fancyhdr}
\usepackage{amssymb}
\usepackage{floatflt}
\usepackage{setspace}
\usepackage{float}
\usepackage{booktabs}
\usepackage{color}
\usepackage{enumitem}
\usepackage{listings}
\usepackage{array}
\usepackage{scrhack}
\usepackage[inkscapearea=page]{svg}
\usepackage{xcolor}
\usepackage{wrapfig}
\usepackage[hidelinks]{hyperref}
\usepackage{lmodern}
\usepackage{multirow}
\usepackage{tabularx}
\usepackage{etoolbox}
\usepackage{subfig}
\usepackage{cleveref}
\usepackage{pstricks}
\usepackage{lipsum}
\usepackage[bottom, hang]{footmisc}
\usepackage{tikz}
\usetikzlibrary{shapes,arrows,calc,positioning,fit}
% !TeX root = ../main.tex
%%%%%%%%%%%%%%%%%%%%%%%%%%%%%%%%%%%%%%%%%%%%%%%%%%%%%%%%%%%%%%%%%%%%%%%%
% Data about you and the Document%
%%%%%%%%%%%%%%%%%%%%%%%%%%%%%%%%%%%%%%%%%%%%%%%%%%%%%%%%%%%%%%%%%%%%%%%%

% % Main Title of Document:
\newcommand{\myMaintitle}{Untersuchung und Aufbau einer DevOps Development Umgebung}

% % Sub Title of DocInput:
\newcommand{\mySubtitle}{Developing holistic software solutions through integration of existing individual solutions.}

% % Ihr Name:
\newcommand{\myName}{Henrik Gerdes}

% % Matrikelnummer:
\newcommand{\myMatrikel}{MatNr: 969272}

% % Ihr Geburtsort:
\newcommand{\brith}{Osnabrück}

% % Ihr Geburtsort:
\newcommand{\place}{Osnabrück}

% % Ihr Abgabedatum:
\newcommand{\submission}{\today}

% % Ihr Abgabedatum:
\newcommand{\mycourse}{Exposé für den B.Sc.}

% % Name des Betreuers/Erstprüfenden:
\newcommand{\fistSupervisor}{Dennis Ziegenhagen}
\newcommand{\secSupervisor}{Achim Hendrix}

% % In welchem Semester befinden Sie sich?
\newcommand{\mySemester}{6. Semester}

\title{\myMaintitle}

\author{\myName}
% !TeX root = ../main.tex
% % Zeilenabstand im Haupttext auf anderthalb-zeilig setzen
%\linespread{1.25}\selectfont

% Line spacing
%\onehalfspacing{}

%Pfad für Grafiken
\graphicspath{{fig/}}

%Styleregeln
\widowpenalty10000 % Vermeidet einzelne Zeilen eines Absatzes zu Beginn einer Seite
\clubpenalty10000 % Vermeidet einzelne Zeilen eines Absatzes am Ende einer Seite
\addtocontents{toc}{\protect\sloppy}
\setcounter{tocdepth}{3}


% % \sloppy bewirkt, dass Latex beim Blocksatz nicht über den rechten Rand hinausschreibt.
% % und dafür größere Lücken in einer Zeile in Kauf nimmt
\sloppy

% % Setzt Dokumenteigenschaften für PDFs, wenn das Paket 'hyperref' geladen wurde.
\hypersetup{pdftitle=\myMaintitle,pdfauthor=\myName,bookmarksopen=true}

%Source for picture captions
\newcommand{\source}[1]{\caption*{Source: {#1}} }

\newcommand{\code}[1]{\texttt{#1}}

\newcommand{\myparagraph}[1]{\paragraph{#1}\mbox{}\\}

\newcommand{\RM}[1]{\MakeUppercase{\romannumeral{} #1{}}}

\newcommand{\HRule}{\rule{\linewidth}{0.5mm}} % Defines a new command for horizontal


\definecolor{dkgreen}{rgb}{0,0.6,0}
\definecolor{gray}{rgb}{0.5,0.5,0.5}
\definecolor{mauve}{rgb}{0.58,0,0.82}

\lstset{ %
  language=Java,                  % the language of the code
  basicstyle=\footnotesize,       % the size of the fonts that are used for the code
  numbers=left,                   % where to put the line-numbers
  numberstyle=\tiny\color{gray},  % the style that is used for the line-numbers
  stepnumber=1,                   % the step between two line-numbers. If it's 1, each line
                                  % will be numbered
  numbersep=5pt,                  % how far the line-numbers are from the code
  backgroundcolor=\color{white},  % choose the background color. You must add \usepackage{color}
  showspaces=false,               % show spaces adding particular underscores
  showstringspaces=false,         % underline spaces within strings
  showtabs=false,                 % show tabs within strings adding particular underscores
  frame=single,                   % adds a frame around the code
  rulecolor=\color{black},        % if not set, the frame-color may be changed on line-breaks within not-black text (e.g. commens (green here))
  tabsize=4,                      % sets default tabsize to 2 spaces
  captionpos=b,                   % sets the caption-position to bottom
  breaklines=true,                % sets automatic line breaking
  breakatwhitespace=false,        % sets if automatic breaks should only happen at whitespace
  title=\lstname,                 % show the filename of files included with \lstinputlisting;
                                  % also try caption instead of title
  keywordstyle=\color{blue},          % keyword style
  commentstyle=\color{dkgreen},       % comment style
  stringstyle=\color{mauve}         % string literal style
}

%%%%%%%%%%%%%%%%%%%%%%%%%%%%%%%%%%%%%%%%%%%%%%%%%%%%%%%%%%%%%%%%%%%%%%%%%%%%%%%%%%%%%%%%%
%Examples
%%%%%%%%%%%%%%%%%%%%%%%%%%%%%%%%%%%%%%%%%%%%%%%%%%%%%%%%%%%%%%%%%%%%%%%%%%%%%%%%%%%%%%%%%
% \pdfmarkupcomment[markup=Squiggly,color=green]{with pdfcomment}{move to the front}.
% \pdfmarkupcomment[markup=StrikeOut,color=red]{stupid}{replace stupid with funny}
% \pdfmarkupcomment[markup=Highlight,color=yellow]{Of course, you can highlight complete sentences.}{Highlight}
% \pdfcomment[icon=Note,color=blue]{insert graphic!}
% Listings style file
\definecolor{forestgreen}{rgb}{0.0, 0.27, 0.13}
\definecolor{ao}{rgb}{0.0, 0.5, 0.0}
\definecolor{delim}{RGB}{20,105,176}
\definecolor{numb}{RGB}{106, 109, 32}
\definecolor{string}{rgb}{0.64,0.08,0.08}

\newcommand\YAMLcolonstyle{\color{red}\mdseries}
\newcommand\YAMLkeystyle{\color{black}\bfseries}
\newcommand\YAMLvaluestyle{\color{ao}}

\lstdefinelanguage{json}{
    numbers=left,
    frame=single,
    rulecolor=\color{black},
    comment=[l]{//},
    commentstyle=\color{blue}\ttfamily,
    showspaces=false,
    showtabs=false,
    breaklines=true,
    postbreak=\raisebox{0ex}[0ex][0ex]{\ensuremath{\color{gray}\hookrightarrow\space}},
    breakatwhitespace=true,
    basicstyle=\ttfamily\small,
    upquote=true,
    morestring=[b]",
    stringstyle=\color{string},
    literate=
     *{0}{{{\color{numb}0}}}{1}
      {1}{{{\color{numb}1}}}{1}
      {2}{{{\color{numb}2}}}{1}
      {3}{{{\color{numb}3}}}{1}
      {4}{{{\color{numb}4}}}{1}
      {5}{{{\color{numb}5}}}{1}
      {6}{{{\color{numb}6}}}{1}
      {7}{{{\color{numb}7}}}{1}
      {8}{{{\color{numb}8}}}{1}
      {9}{{{\color{numb}9}}}{1}
      {\{}{{{\color{delim}{\{}}}}{1}
      {\}}{{{\color{delim}{\}}}}}{1}
      {[}{{{\color{delim}{[}}}}{1}
      {]}{{{\color{delim}{]}}}}{1},
}

\lstdefinelanguage{yml}{
  keywords={true,false,null,y,n},
  keywordstyle=\color{darkgray}\bfseries,
  basicstyle=\YAMLkeystyle,
  sensitive=false,
  comment=[l]{\#},
  morecomment=[s]{/*}{*/},
  commentstyle=\color{blue}\ttfamily,
  stringstyle=\YAMLvaluestyle\ttfamily,
  moredelim=[l][\color{orange}]{\&},
  moredelim=[l][\color{magenta}]{*},
  moredelim=**[il][\YAMLcolonstyle{:}\YAMLvaluestyle]{:},
  morestring=[b]',
  morestring=[b]",
  literate =    {---}{{\ProcessThreeDashes}}3
                {>}{{\textcolor{red}\textgreater}}1
                {|}{{\textcolor{red}\textbar}}1
                {\ -\ }{{\mdseries\ -\ }}3,
}

\lstdefinelanguage{docker}{
  keywords={FROM, RUN, COPY, ADD, ENTRYPOINT, CMD,  ENV, ARG, WORKDIR, EXPOSE, LABEL, USER, VOLUME, STOPSIGNAL, ONBUILD, MAINTAINER},
  keywordstyle=\color{orange}\bfseries,
  identifierstyle=\color{black},
  % basicstyle=\small,
  sensitive=false,
  comment=[l]{\#},
  commentstyle=\color{blue}\ttfamily\emph,
  stringstyle=\color{dkgreen}\ttfamily\textbf,
  morestring=[b]',
  morestring=[b]"
}

\lstdefinelanguage{docker-compose}{
  keywords={image, environment, ports, container_name, ports, volumes, links},
  keywordstyle=\color{blue}\bfseries,
  identifierstyle=\color{black},
  sensitive=false,
  comment=[l]{\#},
  commentstyle=\color{purple}\ttfamily,
  stringstyle=\color{red}\ttfamily,
  morestring=[b]',
  morestring=[b]"
}
\lstdefinelanguage{docker-compose-2}{
  keywords={version, volumes, services, networks, image, environment, ports, container_name, ports, links, build, expose, env_file, restart, depends_on, entrypoint},
  keywordstyle=\color{blue}\bfseries,
  keywords=[2]{environment, ports, container_name, ports, links, build, expose, env_file, restart, depends_on}
  keywordstyle=[2]\color{olive}\bfseries,
  identifierstyle=\color{black},
  sensitive=false,
  comment=[l]{\#},
  commentstyle=\color{purple}\ttfamily,
  stringstyle=\color{ao}\ttfamily,
  morestring=[b]',
  morestring=[b]"
}

\lstset{basicstyle=\ttfamily,
  inputencoding=utf8,
  extendedchars=true
  basicstyle=\footnotesize,       % the size of the fonts that are used for the code
  numbers=left,                   % where to put the line-numbers
  numberstyle=\tiny\color{gray},  % the style that is used for the line-numbers
  stepnumber=1,                   % the step between two line-numbers. If 1 = all lines numberd
  numbersep=5pt,                  % how far the line-numbers are from the code
  backgroundcolor=\color{white},  % choose the background color. You must add \usepackage{color}
  showspaces=false,               % show spaces adding particular underscores
  showstringspaces=false,         % underline spaces within strings
  showtabs=false,                 % show tabs within strings adding particular underscores
  frame=single,                   % adds a frame around the code
  rulecolor=\color{black},        % if not set, the frame-color may be changed on line-breaks
  tabsize=4,                      % sets default tabsize to 2 spaces
  captionpos=b,                   % sets the caption-position to bottom
  breaklines=true,                % sets automatic line breaking
  breakatwhitespace=false,        % sets if automatic breaks should only happen at whitespace
  title=\lstname,                 % show the filename of files included with \lstinputlisting;
  keywordstyle=\color{blue},      % keyword style
  commentstyle=\color{dkgreen},   % comment style
  stringstyle=\color{mauve}       % string literal style
}


% Page style: plain or fancy pants
\pagestyle{fancy}
\fancyhf{}
\fancyhfoffset[L]{1cm} % left extra length
\fancyhfoffset[R]{1cm} % right extra length
\setlength{\headheight}{14.5pt}
\rhead{\thepage}
\lhead{\nouppercase\leftmark}
\cfoot{\fancyplain{}{\thepage} }
% \pagestyle{plain}

% DOC START
\begin{document}
\pagenumbering{gobble}
\include{inc/title.inc}

% For printing add an empty page
% \newpage\null\thispagestyle{empty}\newpage

% Title & Abstract
\maketitle
\begin{otherlanguage}{english}
    \begin{abstract}
        \textbf{English:} \lipsum[20]
    \end{abstract}
\end{otherlanguage}
\begin{otherlanguage}{ngerman}
    \begin{abstract}
        \textbf{German:} \lipsum[20]
    \end{abstract}
\end{otherlanguage}
\newpage

% Table of Contents
\tableofcontents
\newpage

% Normal page numbering
\newcounter{lastroman}
\setcounter{lastroman}{\value{page}}
\pagenumbering{arabic}

% Line spacing
% \onehalfspacing{}

%%%%%%%%%%%%%%%%%%%%%%%%%%%%%%%%%%%%%%%%%%%%%%%%%
%%%%%%%%%%%%%%%%%%%% Content %%%%%%%%%%%%%%%%%%%%
%%%%%%%%%%%%%%%%%%%%%%%%%%%%%%%%%%%%%%%%%%%%%%%%%

% Introduction
% !TeX root = ../thesis_main.tex

\section{Introduction}\label{sec::intro}
% Symbic erwähnen
Digital service solutions are becoming more and more relent and popular as a result of the ever-increasing possibilities and availability of technology. Due to the COVID-19 pandemic new remote working tools, digital education resources, a connected healthcare system and seamless (video) communication systems are needed. The e-commerce marked grew by 32\% between 2019 and 2020 up to sales volume of \$791.70 billion \cite{online_shopping_inc} and the overall revenue in the software market is expected to grow from 532 billion (2020) to 772 billion in 2025 \cite{software_industry_groth}. The need for digital business solution is bigger than ever before. The availability of on-demand computing capacity is greater than ever before thanks to the ever-growing cloud platforms such as \ac{AWS}, Microsoft Azure and Google Cloud. Services can be made available quickly across the world without having to build an own infrastructure. To provide fast and suitable software solutions developers need adequate development setups, called development environments. This thesis analyses modern agile software development environments, points out potential problems and purposes a virtualized software development solution to improve development efficiency. This solution is implemented on a real project and then evaluated.\newline
The following section will briefly describe the fundamental core question of this thesis, point out its relevance and presents a possible solution approach. Subsequently, a clear delineation is then given as to what is and is not covered in this thesis. The last Section of the introduction gives a structural overview of how the thesis is structured and how it will approach the given problem.

    \subsection{Problem Description}
    The rise of agile development and microservices was accelerated by the emergence of new technologies that changed the way \textit{where} and \textit{how} software is running. Applications could be distributed and scaled quickly through cloud computing which benefited customers favors for quick changes. Yet the actual coding setup has not changed significantly.\newline
    Requirements for development environments differ depending on the project type. The development process of (\ac{GUI}) applications for PC's and smartphones is quite different to web-services development. As the functionality of web services continues to grow, while being a cost-effective way to deliver across platforms products, this method is becoming increasingly popular. However, the operating system used for the development process often differs to the operating system used to run these applications in production. This can cause platform dependent errors. Managing programming language runtime versions between team members or different projects become a challenge, as these can lead to unexpected program behavior or result in library version conflicts. The usage of a \acl{MSA}, due to its easy scalability, adaptability and rapid development, brings further problems. Microservices require additional configuration effort and increase the difficulty level for whole system tests, called end-to-end tests. The initial setup for new developers can get quiet complex, requires time and might even discourage developers in open source projects.\newline
    These characteristics add additional effort, can introduce new errors and slow down the development speed, resulting in higher cost and lower customer experience.

    \subsection{Goal of the Theses}
    In the process of this work, the challenges and obstacles of modern software development environments are identified and presented. Based on these findings, a solution concept, for these challenges, is created based on virtualization technologies. The technologies used are explained, and the solution concept is applied to a real project in cooperation with Symbic GmbH. Subsequently, the practicability of the solution is evaluated and classified. In principle, the goal of this thesis is to identify obstacles in current software development setups and to analyze the effectiveness of the proposed solution.
    % TODO
    % Im „Goal“-Abschnitt sollte zumindest grob auf die vorige Problem Beschreibung eingegangen werden. Denn: Warum sollten Probleme benannt werden, wenn nicht erkennbar ist, wie sie gelöst werden könnten? Strategisch ist es hier sinnvoll, für mind. eines der beschriebenen Probleme eine konkrete Lösung zu skizzieren; also einen Ausblick auf einen Teil deines Konzepts zu geben. In ein oder zwei knappen Sätzen, sodass eine minimale Idee vermittelt wird, wohin die Reise geht.

    \subsection{Scope of the Thesis}
    The range of different software development environments is large and differs significantly from each other depending on the project. Since the development of web-based solutions is becoming more and more popular, priority is given only to these types of projects. Native application development for PCs and smartphones, as well as the development of embedded systems and other hardware-related solutions are not covered in this thesis.\newline
    In particular, the solution concept shown is not a general solution that is a perfect fit for all projects and therefore is not generally transferable. It is intended only as a solution template that contains many tricks for dealing with the problems described in section \ref{sec::problem}. Although section \ref{sec::backgrund} offers an explanation of fundamental topics and section \ref{ssec::toolsused} a deeper insight into the functionality of technologies used, nevertheless not all addressed contents can be explained fundamentally. Basic knowledge of the software development process, essential programs and abstract understanding of different operating systems and network techniques are recommended.

    \subsection{Structural Overview}
    The following section provides overall background information, which are recommended for further understanding. Common and modern development methodologies are described in section \ref{ssec::devops}, followed by microservice and container basics in section \ref{ssec::microservices} and \acl{CI} tools in \ref{ssec::devops}. In section \ref{sec::problem} will be a detailed analysis of current development setups and the problems encountered in these setups. The problems are illustrated with examples and their consequences for the development process, the overall project and its quality, which is emblematic of user satisfaction, are presented. Section \ref{sec::solution_concept} proposes a conceptual solution, defines its usability and workings while also providing conditions and limitations for such a solution. This solution concept is applied in section \ref{sec::solution_code} by implementing it in a practical reference project. The implementation process is described and additional challenges and tricks for practical use are presented. Section \ref{sec::eval} discusses the proposed solution and shows its strengths and weaknesses. At the end, an outlook on future application areas and alternative solutions is given in section \ref{sec::outlook}, followed by the conclusion in section \ref{sec::conclusion}.


% Background info
% !TeX root = ../thesis_main.tex

\section{Background Information}\label{sec::backgrund}
The following section provides some basic definitions and principles necessary for the further understanding of this paper and ensures a common level of knowledge of these topics. A formal definition of Agile and DevOps is given, followed by a brief description of the emergence and its current adoption. Next, the basics of virtualization and its usage in a microservice architecture are explained, before the clarification of the connection between microservices, DevOps and familiar tools in a DevOps environment are then pointed out, which forms the end of the chapter.

    \subsection{Modern Software Methodologies}\label{ssec::devops}
    This Section contains a formal definition of agile software development paradigms and the DevOps methodology, explains their connection and points out the fundamental principles of a DevOps enabled culture. These principles and their workflows will be revisited in Section~\ref{sec::solution_concept} as part of the proposed solution concept.
    % \comment{This may be too much and maybe not all of it is needed for the work. Make more compact or remove some of it.}

        \subsubsection{Agile Development}
        \wordhighlight{Agile}, such as the \wordhighlight{V-Model}, \wordhighlight{Waterfall} and \wordhighlight{Prototyping} model, is a software development paradigm. It was proposed and popularized by the \wordhighlight{"Manifesto for Agile Software Development"}, written and published in 2001 by various authors~\cite{manifesto}. Rapid adoption to changes, continuous evolution of software and customer communication are the fundamentals of agile software development.\newline
        The four core principles are:

        \begin{itemize}[label=\(\star\)]
            \setlength\itemsep{0em}
            \item \textbf{Individuals and interactions} over processes and tools
            \item \textbf{Working software} over comprehensive documentation
            \item \textbf{Customer collaboration} over contract negotiation
            \item \textbf{Responding to change} over following a plan
        \end{itemize}

        % Requirements mit s nach dem s
        \noindent Paradigms like Waterfall describe a comprehensive model with an in-depth analysis of requirements and detailed architecture design phase. This results in a fixed and exact sequential schedule for the implementation and testing phase as well as the phase in which the requirements fulfilment is verified. Errors in the requirements analysis or changes in the requirements can cause difficulties in later phases of the project. Agile on the other hand only provides guidelines instead of a complete model. Changes are expected and the project realization is designed to be adaptable. Project phases such as implementation and testing run simultaneously and the entire \ac{SDLC} is shorter and more adaptable \cite{agile_practice}.\newline
        Agile provides a mindset for projects with uncertain or continuously changing requirements. With the increase of software solutions in rapidly changing markets, software requirements also became more uncertain. Accordingly, the agile manifesto became the foundation of several other software development methodologies and frameworks that extend these fundamental guidelines and provide workflows and tooling for the software creation process. Well known exemplary methodologies are \wordhighlight{SCRUM}, \wordhighlight{\ac*{XP}} and \wordhighlight{DevOps}.

        \subsubsection{Definition of DevOps}
        \wordhighlight{DevOps} is a set of practices in software development that aims to increase customer value and software quality by shortening the development life cycle through active collaboration and continuous delivery of improvements \cite{base_devops}, \cite{effective_devops}. The term DevOps is a neologism of development (Dev) and operation (Ops). The combination of these terms is symbolic for the tighter collaboration between the development and operations team, which were strictly separated beforehand. Therefore, DevOps is considered more than just software development principles, it is considered a mindset and company culture. Shorter development times and closer collaboration are the goals of the agile software development paradigm. DevOps builds upon the guidelines and goals of Agile and offers workflows and tools for the software development process for the purpose of an increased user experience value \cite{azuredevops}, \cite{effective_devops}.

        \subsubsection{Principles of DevOps}\label{ssec::devops_princibles}
        In addition to Agile principles, DevOps also extends the \wordhighlight{\acl{XP}} approach by applying its principles to the operations and infrastructure aspects of the application \cite{effective_devops}. The goal is to provide a structured and comprehensive process from coding through testing, packaging and deployment, to operation and monitoring. This process is referred to as a workflow which is made up of several steps. Table \ref{tab::devops_steps} shows an exemplary workflow with a minimum number of steps~\cite{base_devops}.

        \begin{table}[]
            \centering
            \begin{tabularx}{0.85\textwidth}{llX}
                \multicolumn{2}{l}{Step} & Description \\ \midrule\midrule
                1 & \textbf{Coding}& Code development \& review and source control.  \\
                2 & \textbf{Building}& \acs{CI} build and build status.  \\
                3 & \textbf{Testing}& \acs{CI} testing and testing feedback.  \\
                4 & \textbf{Packaging}& Bundle and package to a central registry.  \\
                5 & \textbf{Releasing}& Release management, approvals and automation.  \\
                6 & \textbf{Configuring}& Infrastructure configuration and management.  \\
                7 & \textbf{Monitoring}& Applications performance monitoring.  \\
            \end{tabularx}
            \caption{Common DevOps Workflow Steps. \\\textit{Source:~\cite{base_devops}}}
            \label{tab::devops_steps}
        \end{table}

        \noindent Tasks and activities representing the core process of a DevOps operation emerge with such a workflow. These tasks include configuration management, release management, \ac{CI}, \ac{CD}, infrastructure provisioning, test automation and application performance monitoring~\cite{azuredevops}.
        Generally, it is the overall aim to conduct structuring and optimizing processes on the entire path of the \ac{SDLC}, while also providing the team with appropriate tooling. Even the infrastructure provisioning is structured and automated as its specifications are stored as code, just like application configurations. This approach is referred to as \wordhighlight{\ac{IaC}} \cite{base_devops}. This work will focus on applying this structuring of workflows to the development environments in order to enable homogeneous environments and an efficient development process. For this purpose, tools from the \ac{CI} and \ac{IaC} concepts will be used. \newline
        A DevOps-principled architecture has multiple deployment stages. A staging and live environment is considered the minimal foundation of any such project \cite{azuredevops}. In order to test and operate this deployment stages software teams need to have a reliable and efficient way to manage their infrastructure state and configuration. This is closely related to the \ac{IaC} principle, which describes the provisioning and maintenance process of computing resources. Tools like \wordhighlight{Ansible}, \wordhighlight{Puppet} and \wordhighlight{Terraform} can automatically maintain, create and set up new cloud \ac{VM}s based on rules specified in a playbook. Playbooks are structured policies written in a custom language which are also stored in a \ac{VCS}~\cite{ansible2020}, \cite{azuredevops}. These files describe how the host operating system is configured. Changes on these files can be handled the same way as regular code, starting with a pull request, reviews, test and approval of the changes. The \acl{IaC} principle will later be used to properly handle the setup of development containers. It is also used to configure \acs{CI} tools as described in Section~\ref{ssec::getting_devops}.

    \subsection{Microservice and Virtualization Concepts}\label{ssec::microservices}
    The following section provides a basic understanding of the concepts of microservices, how they are implemented and what possibilities they bring with them as well as what limitations they have. In this context, different strategies of virtualization are explained, which are then used in section ~\ref{sec::solution_concept}.

        \subsubsection{Fundamental Idea of Microservices}\label{sssec::micro}
        The fundamental idea of a \ac{MSA} is to have small self-contained, independently deployable software applications. Connections between these applications enable greater service with an extensive range of functions \cite{micro}. Figure \ref{fig::micro} visualizes an exemplary structure of a microservice cluster. It shows multiple microservices that each expose one functionality to the outside world. Application end-points, such as a \ac{UI}, provide a broad range of functionality by internally calling multiple microservices. In the event of an application failure only that specific functionality becomes unavailable while the remaining system maintains its operational status. Due to this architectural design feature, it is advised that while working with data, each microservice has its own \ac{DB}. This prevents a central \acl{DB} from becoming a single point of failure that can bring down the entire service. Since the service is build upon multiple small applications, it is possible to use different technologies for every application. The initial cost of new technologies is thus lower and individual service parts can be replaced at a lower cost compared to a monolith. Another core feature of microservices is its scalability. If the load on one part of the service increases, new instances of that application can be deployed to balance the load across multiple instances \cite{micro}. This concept requires a fast and reliable process of creating new application instances which may build upon different technology stacks.\newline
        The installation and setup process of new hardware can be a time-consuming task. In order to reduce the amount of time it takes to create new application instances, the software industry uses the concept of virtualization.

        \begin{figure}
            \centering
            \includegraphics[width=0.55\textwidth]{monolithic-vs-microservices_altered.png}
            \caption{Structure of Microservices - [Altered], \\\textit{Source:~\cite{redhat_micro}}}\label{fig::micro}
        \end{figure}

        \subsubsection{Virtualization and Containerization}\label{sssec::virtual}
        Virtualization is an abstraction layer. The physical hardware (host) runs a hypervisor that allows the execution of (multiple) virtual machines (guests), which act like a regular computer~\cite{vmbasics}. This approach allows the usage of heterogeneous hardware without an impact on the guest operating systems due to the abstraction provided by the hypervisor. Without the need for specialized hardware and the dynamic allocation of resources, efficiency is increased~\cite{redhat_venv}. Additionally, virtual systems can be managed more easily because they are fundamentally just one big file on the host's storage device. They can be created on command, cloned and deleted without the configuration steps of a physical system. In the enterprise industry it is common to use this flexibility to start additional \ac{VM}s on high load. According to a study by the \ac{IDC} more than 80\% of data center workloads are virtualized~\cite{virtualaddoption}. Virtualization comes with the benefit of security. The majority of hypervisors strictly separate the host and the guest system in that the guest system is not allowed to use the hosts resources and access its files unless it is explicitly configured to do so. Compromising a \ac{VM} does not affect the host or any other \ac{VM}s~\cite{vmbasics},~\cite{redhat_venv}.\newline
        Full guest virtualization emulates a complete \ac{OS}, including the kernel, system libraries and even the majority of hardware devices. This abstraction comes with a performance penalty called overhead~\cite{vmbasics}. A supposedly more lightweight approach of virtualization is called containerization. Studies by Ericsson Research, Nomadic Lab~\cite{ieee_perfomance} and the Zhengzhou University~\cite{zhengzhou_university} conclude that, in fact, container based virtualization solutions provide better performance, especially in disk \acs{I/O} and network \acs{I/O} bound scenarios. Containerization focuses on the isolation of a single application process in a virtual runtime using control groups and namespace technology~\cite{cgroups}. Unlike \ac{VM}s, system and kernel functions are not virtualized and are passed through to the host machine. Consequently, overhead is reduced and allow the ability to run additional application instances compared to a \ac{VM}-based approach with the same amount of compute resources. Figure~\ref{fig::vm_docker} visualizes the differences between these approaches. The left side shows a traditional \ac{VM}-based approach. On top of the host \ac{OS} runs a hypervisor which provides three full guest \acl{OS}s with one application each. Each guest is fully isolated and features its own kernel, \ac{OS} and runtime libraries. The container-based solution on the right side only needs the host \ac{OS} and provides multiple application instances with shared libraries and runtimes in separated, isolated namespaces.\newline
        Apart from the performance benefits, the presumably main advantage of containers is their scalability, due to faster creation and startup times, which is what makes them an adequate fit for microservices \cite{cintainer_scale}. Docker, Podman and LXC are, amongst others, the most popular container-based virtualization solutions. A more detailed explanation of Docker can be found in Section~\ref{sssec::docker}.

        \begin{figure}
            \centering
            \includegraphics[width=0.9\textwidth]{docker-vm-redhat.png}
            \caption{Comparison of \ac{VM}s to Containers, \\\textit{Source:~\cite{redhat_pic}}}\label{fig::vm_docker}
        \end{figure}

        \subsubsection{Usage of Containerization in Microservices}
        As described above, microservices are small, bounded applications that communicate with each other. In order to follow the principles of loose coupling between applications, the communication should be performed via a protocol independent of the programming language. Typical IP based protocols used are \wordhighlight{\ac{REST}}, \wordhighlight{WebSockets} or \wordhighlight{GraphQL}~\cite{micro}. These loose-coupled applications have the advantage of being able to be developed simultaneously from different teams as well as upgraded and replaced independently. As a result, applications can be developed much faster and more flexible, following the principles of agile development~\cite{micro},~\cite{redhat_micro}.\newline
        The usage of containers drives this speed and flexibility even further. Containers provide a consistent and  isolated, yet flexible runtime for applications \cite{micro_container}. Applications are packaged within known good runtimes. This reduces the setup time of the deployment and eliminates host-specific errors. New application instances can be started without additional configuration. As a result of these successful concepts, tools like \wordhighlight{Docker Swarm} and \wordhighlight{Kubernetes} have been developed which can scale distributed applications in a managed dynamic, even automatic, way.\newline
        Packaging and eventually deploying the application introduce additional work for developers, which was previously a task of the operations team. As already stated above, the developer and the operations team are not separated in a DevOps culture. This approach values team communication, flexibility and autonomy, enabled by the structured, automated workflows between the development and operation tasks~\cite{effective_devops}. Eliminating manual tasks allows developers to focus on the actual application development. One of the main concepts in this process is the usage of \ac{CI} and \ac{CD} workflows.

    \subsection{Tools for Achieving a DevOps Environment}\label{ssec::getting_devops}
    Continuous integration and continuous deployment are two working concepts particularly well known in a \ac{MSA}. Since every application only provides one part of the overall service functionality, it is uncertain what effects a change in one application has on other applications and on the whole service. Accordingly, testing must take place between applications, especially if they are developed by different teams. This requires workflows checking these changes and simplifying recurring tasks through automation. For this purpose, explicit concepts and programs, which will also be used in the later Symbic project are presented below.

        \subsubsection{Version Control}
        A \acl{VCS} (\ac{VCS}) is an essential tool in the software development industry to keep track of which change has been made when and by whom. Contact persons for specific code sections are thus directly known and the responsibilities are clearly defined. \wordhighlight{Git} has become the de facto standard over the last 10 years and is also used in the following sample project, just like by over 93\% of all developers according to the StackOverflow Developer Survey 2021~\cite{stackoverflow2018}. Git allows collaborative, distributed work in own branches without affecting others. Changes can easily be merged back into the main branch, after an optional review has approved the changes. The new code is then made available for all developers. In the Symbic project, these functionalities are provided by a self-hosted instance of \wordhighlight{GitLab}. This is a service used to coordinate teamwork via issues and to document the project. Additionally, it serves as a central code repository. Its adaptability, extensive integration options and built-in \acs{CI} tools provide all the necessary software lifecycle programs from a single service bundle \cite{gitlabdocs}. \wordhighlight{GitHub} and \wordhighlight{Bitbucket} are alternatives, both of which offer code hosting, issue management, and \ac{CI}/\ac{CD}.

        \subsubsection{Continuous Integration}
        Continuous Integration is a practice in which new code is regularly integrated into the main code branch and into the overall service. Instead of having isolated functional branches that are worked on independently for months, changes flow back regularly into the main code branch to ensure it is free from conflicts and errors. Especially in interconnected services, it is indispensable to ensure that a change in one application do not have an unintended effect on other applications. \ac{CI} systems can perform automatic tasks, called jobs, on the code when it is checked into the \ac{VCS} repository. These tasks can ensure a specific code-style, run Unit or \acs{API}-tests and can also run integration tests against other services. If a task fails, the corresponding developer is notified and the changes are not incorporated into the main development branch. In case of an error, it becomes easier to identify its source due to these small and continuous integrations. Accordingly, errors and their causes can be identified more quickly and resolved without delay \cite{azuredevops}, \cite{base_devops}.\newline
        Popular services providing continuous integration functionality are \wordhighlight{Travis}, \wordhighlight{CircleCI}, \wordhighlight{GitHub Actions} and the previously mentioned GitLab \ac{CI}. These services differ in their function range, type of job configuration, the level of control, in their pricing models, or the number of free jobs, respectively, and in whether they can be self-hosted.

        \subsubsection{Continuous Deployment}
        Continuous deployment, on the other hand, is a practice bringing these regular updates into production quickly and automatically, making them available for customers, respectively. It typically involves two steps, continuous delivery and continuous deployment. The continuous creation of software bundles, which are called artifacts, falls under Continuous Delivery. The transfer of these artifacts and the process of making them usable is described as Continuous Deployment. To ensure product quality, it is best to have multiple deployment environments, as described in section \ref{ssec::devops_princibles}. Each new version is automatically deployed to a test or development environment, in which it undergoes automatic or manual testing. Common test types are security scans, load-, usability-, and acceptance tests. If all tests are successful, the version is promoted to the next deployment environment. In case of an error, the version is discontinued, and the developers get the corresponding notification. Only if none of the testing environments reveal any errors, the version will be transferred to production as a new release. \ac{CD} is an extension of the \ac{CI} principle and represents the next logical step in the software development workflow, which is why \ac{CI}/\ac{CD} are often used together \cite{azuredevops}. Well known \ac{CD} solutions are \wordhighlight{Jenkins}, \wordhighlight{Azure Pipelines}, \wordhighlight{\ac{AWS} CodeDeploy} and \wordhighlight{Argo \ac{CD}}.

        \subsubsection{Building a Streamlined Workflow}
        By combining these principles mentioned above, powerful workflows can be created that accelerate software development through automation and ensure consistent quality \cite{base_devops}. Figure~\ref{fig::cd} visualizes a complete exemplary \ac{CI}/\ac{CD} pipeline. It is typically built from multiple stages, each of those stages bundles one or more related tasks. A stage gets triggered by an event, such as code check-in or a previously successful stage. Common tasks are code style tests just as unit tests, automatic compilation or packaging and the deployment to a test system \cite{azuredevops}.\newline
        Both \ac{CI} and \ac{CD} are practices to automate tasks that have previously been performed manually. Integration testing and packaging both happen within the scope of a dedicated, autonomous system, which gives developers and operators more time for other activities. Deploying and using such a system are typical tasks in a DevOps practicing team. Here, the focus is on collaboration between the development and operations teams. At times new services must be added to the pipeline or existing services must be adapted to the current development status, depending on the current task. This workflow allows a team to react quickly and flexibly to changes and thus to comply with the agile principles. In collaborating closely and developing these well-functioning systems, the main goal is to add value to the product and to customer experience \cite{azuredevops}.\newline
        The GitLab instance, as used in the following project, serves to automatically test, build and deploy the code to a testing environment. Docker images are used to bundle the individual applications, which are explained in the next section. Thus, the built-in private image registry of GitLab also allows the creation of always up-to-date Docker images and makes them accessible to all authorized entities \cite{gitlabdocs}.
        % !TeX root = ../thesis_main.tex
\begin{figure}[]
    \centering
    \tikzstyle{block} = [rectangle, draw, fill=green!80!blue!70,
    text width=5em, text centered, rounded corners, minimum height=4em]
    \tikzstyle{line} = [draw, very thick, color=black!50, -latex']

    \begin{tikzpicture}[scale=2, node distance = 5cm, auto]
        % Place nodes
        \node [block] (init) {Code Check-in};
        \node [block, right of=init] (test) {\ac{CI} Pipeline \& Code-Test};
        \node [block, right of=test] (build) {Build \& Package Code};
        \node [block, below of=build, node distance=3.5cm] (d_test) {Deploy to Testing};
        \node [block, left of=d_test] (d_staging) {Deploy to Staging};
        \node [block, left of=d_staging] (d_live) {Deploy to Production};

        % Draw edges
        \path [line] (init) -- node {Triggers} (test);
        \path [line] (test) -- node {If successful} (build);
        \path [line] (build) -- node [left]{If successful} (d_test);
        \path [line] (d_test) -- node [above]{On approval} (d_staging);
        \path [line] (d_staging) -- node [text width=2.5cm, align=center, above]{On release approval} (d_live);
    \end{tikzpicture}
    \caption{Continuous Deployment Workflow, \\\textit{Source: Modeled after~\cite{azuredevops}}}
    \label{fig::cd}
\end{figure}



        \subsubsection{Containerization with Docker}\label{sssec::docker}
        In order to use the containerization concepts described in section \ref{sssec::virtual}, they have to be implemented in software. One solution for the implementation of container-based virtualization is Docker. Currently, Docker is the most popular containerization platform that is supported on Linux, Windows and macOS~\cite{docker_share}. It utilizes the Linux kernel of the host and its cgroups functionality for resource isolation. On none Linux host systems, Docker virtualizes the kernel via a built-in hypervisor. Docker was chosen for its wide usage, extensive documentation and its platform independency, which enables the usage in a local and productive operation.\newline
        Docker allows the isolated execution of applications within a container in a well-defined state without affecting the host. The initial state of a container is defined in a disk-image that bundles all the software, libraries and configuration files needed. Images are built based on a \wordhighlight{Dockerfile} that contains sequential, imperative instructions on how the container \ac{OS} should be configured. An exemplary Dockerfile can be found in the appendix in Listing \ref{code::docker}. Common steps are the \code{COPY} command to add files from the host to the container and the \code{RUN} command to execute shell commands within the container. Each file-changing instruction adds another layer to the overlay-filesystem that is used by Docker. Effectively making each layer a read only filesystem once it is written. At runtime, a file-lookup is performed in the top filesystem layer; lookups in the layers below follow if the file is not found in the first one. This allows the reuse of layers, because changed versions of files are simply pushed into a new layer and overshadow the original files. Images are usually kept as narrow as possible, so that they can be transferred more quickly and  present less a target, when it comes to matters of security. Therefore, only the minimum of necessary programs are provided in an image. Docker images are distributed via a registry to other systems. The company which develops Docker provides an official and public Docker image registry called \wordhighlight{DockerHub}. It offers official images of common applications, such as \acl{DB}s and webservers, as well as base images for own projects \cite{docker2020}, \cite{dockerdocs}.\newpage
        % !TeX root = ../thesis_main.tex

\subsection{Code Listings}
% \addcontentsline{toc}{subsection}{Code for you}
\begin{lstlisting}[language=docker, frame=single, caption={Exemplary NodeJS Dockerfile},label=code::docker]
# The base image to start from
FROM ubuntu:18

# Install nodeJS
RUN apt-get update && apt-get install -y nodejs npm

# Copy content of the Hosts "backend" folder
# to the "app" folder in the container
WORKDIR /app
COPY ./backend /app

# Install all node-modules
RUN npm install

# Specifies the command that is executed
# when the container starts
ENTRYPOINT [ "npm", "start" ]

\end{lstlisting}

        When a container is started, the program specified in the image is executed, and the container runs until the started program exits. Each container is treated as an independent full-fledged computer with its own \ac{OS}. For this reason, each container has its own IP address and is by default not accessible from the host system. The user has to explicitly expose a container port to the host system in order to access the application inside the container. Each container does not only have its own IP address, but is also on its own virtual, software-defined network, which is managed by Docker and even uses a \ac{DNS} server that is inherent to Docker. Accordingly, containers within a defined network can communicate with each other by their (host-) names and provide their functionality outside the network by exposing their published ports to the host \ac{OS} \cite{docker2020}.\newline
        Unlike \ac{VM}s, the container's guest \ac{OS} is not maintained, as containers are considered disposable. Instead of updating the software within the container, it is simply replaced by a newer one from an updated image. All changes and new files within the old container are irreversibly lost. In order to avoid the loss of user data, private keys and the like, corresponding files can be stored persistent on the host system and made available in the container via mounts. A mounted directory behaves like a new top layer in the containers overlay filesystem. Docker distinguishes between volume-mounts and bind-mounts. Volumes are managed by Docker, have a unique name, can easily be shared between containers and are transferable between hosts whereas, in the case of bind-mounts, a specified directory is mounted from the host to the container. All filesystem attributes and file properties are passed on to the container, while their management is up to the user \cite{docker2020}, \cite{dockerdocs}.\newline
        Alternatives to Docker are \wordhighlight{Podman}, \wordhighlight{Rkt} and \wordhighlight{LXC}, which Docker was originally based on. The downside of these tools are their lower degree of integrability and lack of multi-platform support.

    \subsection{Relevance of Modern Development Techniques}
    The concepts explained above are becoming more and more relevant with the increasing adoption of agile principles, even outside the software industry. The annual agile report shows a significant growth in agile adoption from 37\% in 2020 to 86\% in 2021. The main reason for adopting agile practices seem to be enhanced ability to manage changing priorities and an increased speed of software delivery. Customers benefit from new features directly and developers receive immediate feedback. The same reasons for adopting agile also apply to the investment in a DevOps culture. The annual survey reveals, that more than 70\% of all respondents, are executing or planning a DevOps initiative. One of the biggest challenges in implementing DevOps seems to be inconsistencies in processes and practices, which is also reflected in the fragmented and heterogeneous processes described in the next section \cite{agilereport2021}.


% Problem description
% !TeX root = ../thesis_main.tex


\section{Analysis of the Current State of Development Environments}\label{sec::problem}
The following section will display the current status of a typical development environment, point out recurrent problems and provide approaches for possible solutions.

    \subsection{Current State of Development Environments}
    Software development involves a broad variety of tasks. Depending on the project, tasks can range from web development, embedded system development, desktop or mobile application development, data analysis or the pure maintenance of one of these areas. Each of these software development fields has its own workflows and requirements for the actual development setup. Even within these specialized categories, there are different requirements, depending on the scope, size, and general demand for capabilities and power of a project. Accordingly, the setups of development environments can differ greatly from one another. Due to the large scope of the subject and in order to set useful limitations, this work will only refer to web services built on a microservice architecture. Nevertheless, parts of this work can also be adapted to other types of development.\newline
    Despite the increasing use of web-based \ac{IDE}s, such as \wordhighlight{CodeSandbox}, \wordhighlight{StackBlitz}, \wordhighlight{Codespaces} and \wordhighlight{Gitpod}, native development environments are still used primarily. The StackOverflow Survey 2021 supports this thesis and clarifies further that Windows is the primarily used operating system, among professional developers, followed by macOS and Linux~\cite{stackoverflow2021}. Accordingly, developers need modern hardware, on which all the required applications must be installed and set up locally. Depending on the application area, this may include some applications for which license costs may have to be paid and which must be kept up to date after the initial installation. The setup and maintenance of these tasks are necessary supporting processes in software development, but they can cost a lot of time and effort, and therefore money without producing any actual progress and value. This is especially true if there are many developers in a large company with frequently changing developers. Once the environment is running, developers can start to code and contribute to the project. Programming is about changing things and changes can break things, as can be seen in the case of GitHub.com. Here, developers created a series of scripts just to get developers to a working environment in less than a day. Because of frequent error these scripts included a global clean option (called \code{--nuke-from-orbit}) to reset the environment the initial, known good, state~\cite{githubblogcodespace}.\newline
    Such a local working environment leaves developers in a position, in which they have more control over their development environment than in the case of the pre-defined and functionally restricted web-based environments.
    However, local setups require much effort and are result in an elaborately error-proneness's environment as the next section demonstrates. A more detailed comparison to the web-based concepts is given in the evaluation in section \ref{sses::eval_compare}.

    \subsection{Common Issues in Modern Development Setups}
    The current congestion of development environments, as described above, is a collection of different programs and their configurations that each developer has to spend worktime on in order to set up a working state suitable to their personal preferences. Only when this has been accounted for the respective project can the developers begin with their actual work. However, their work may be restricted by a chosen software architecture, and interrupted if changes in the code or its runtime cause problems with the development environment. The following section describes the most common issues with local development environments.

        \subsubsection{The Initial Setup Process}\label{sss::initial}
        Newly recruited developers or those changing projects need time and support to settle into the new project. They are not familiar with the code base and instructions on how to set up the environments correctly. Evidently, good code documentation helps as long as it is available, up-to-date and detailed enough. Specific software packages such as \ac{DB}s, interpreters or compilers need to be installed and configured in the correct versions in order to make the local environment operational. Programming languages such as C/C++ and PHP do not included a debugger in their language framework, and they need to be installed and set up separately. Depending on the project, this can be very extensive and require a lot of time and support from other team members. Setting up a platform independent NodeJS project is considerably simpler thanks to the included package manager \wordhighlight{npm} than it is with system specific C++ or PHP projects. A survey by ActiveState shows that more than 25\% of developers need five more hours to set up a development environment. Considering that more than 65\% of all developers do this one to four times a year, 30\% even five to twelve times a year, this adds up to a significant percentage of work hours \cite{setuppain}. A Web-Search also reveals that broad discussions about automating this process actually do exist. Further problems arising from the configuration of the respective programming language are discussed in more detail in the next point. \newline
        Even though some of this initial work can be automated through the use of scripts, these scripts must be created and maintained for each operating system used. They may be a time saver as long as they don't break due to moved download links or unavailable files.

        \subsubsection{Dependency Management \& Configuration Shift}\label{sssec::dependency}
        % Ist etwas vermischt
        Once the development environment is set up, it is a crucial to maintain it in such a way that developers can perform their actual tasks and contribute value to the product. Some programming language frameworks are able to keep dependencies consistent across multiple systems without much effort thanks to their integrated framework tools. NodeJS' current \ac{LTS}-Version 14, for example, installs all required dependencies into the local project folder and keeps a precise record of their versions by means of a \code{package-lock.json} file. Other languages either come without any package manager at all (like C/C++, PHP and Java), or their package manager installs dependencies globally. The existence of tools like Pythons \wordhighlight{venv} package, which creates virtual isolated environments for each project, proves that dependency management is, in fact, a problem in local development environments~\cite{pythonvenv}. Apart from architectural designing, collaborative meetings and testing time, the investigation of bugs and the maintenance of the application setup are the most time-consuming tasks when developers are not coding. Among the problems that can occur, dependency issues are ranked as the third-largest group, just after of package building issues \cite{setuppain}.\newline
        Even the local package installation approach of NodeJS does not solve the problem of developing against multiple versions of the NodeJS framework. Testing a new major version of NodeJS can break existing project configurations, while undoing these changes can be tedious and time-consuming. Additional tools like the \wordhighlight{\ac{NVM}} have to be used in order to run multiple NodeJS versions on the same system. It becomes particularly problematic when developers are working on several projects with different dependency requirements at the same time. Legacy applications may require runtimes and libraries that are no longer supported on modern systems. Versions prior to PHP 7, for example, can no longer be installed (without additional effort) on current Windows or Linux distributions. These issues are assigned to the category of runtime dependency management. In a microservice architecture, the service dependencies further extent this problem. \newline
        In a microservice architecture, there are several granular applications, each providing functionally related logic and having bounded endpoints. Each endpoint can be considered a public \ac{API}, even if the service is only accessible within the overarching service. Communication between these endpoints is called inter-service communication. Maintaining compatibility between applications becomes a challenge, especially with many microservices or when using a service mesh \cite{micro}. If a team member changes the interface of one microservice, this affects other services and the developers responsible for these applications need to be notified and respond in order to avoid unexpected errors. Tools like \wordhighlight{Swagger}, for applying the OpenAPI-Specification, can help cope with these challenges. Nevertheless, the \ac{API} version used must be defined and configured accordingly in other applications. The management of inter-service communication has a direct impact on the testing possibilities, which are discussed in the next section.\newline
        % ToDo
        % Würde terraform zuvor erwähnt und erklärt
        In ideal environments, the local setup would always work as long as no changes are made. However, there are always changes, \ac{OS} and security updates, changes to the dependencies to test a new version or the temporary change of the network port for a side-by-side comparison. The modification of the database schema by one developer can cause a broken environment for other developers. These slight changes over time are called configuration shift. No system is identical to another, which can lead to unreproducible builds and fluky errors. For this reason, the popularity of tools such as \wordhighlight{Terraform}, \wordhighlight{Puppet} and \wordhighlight{Ansible} that implement the \ac{IaC} principle and rely on the automated creation and configuration of \ac{VM}s instead of a manual creation, is increasing in the operational cloud sector. Each instance of a machine is configured in such an exact and consistent way a human being would be incapable of. If a \ac{VM} behaves abnormally, it is torn down and recreated. However, local environments are not considered disposable, they are personalized and therefore hard to create and maintain automatically. Some projects may require exact reproducibility, especially in a scientific context. Yet, this is hard to achieve in indiscriminately configured and personalized environments.\newline
        The tools mentioned above help to keep the system in a consistent, good state, but do not help much with the many changes within the application. Well-known software practices, such as database migrations and test suites, are required to minimize errors within applications, but testing is a particular challenge in a microservice architecture.

        \subsubsection{Lack of Testing Options}\label{sss::testing_problem}
        The goal of tests is to verify the behavior of the system under the given conditions. It is a crucial practice to ensure product quality and high custom value \cite{azuredevops}. Functional tests can be categorized in the four stages shown in Table~\ref{tab::tests}. According to the test pyramid, unit tests are the tests with the smallest scope, but which should be implemented most heavily. They ensure the correct behavior of a function or a class and only relay on the code to be tested itself. Only preceded by a compiler or a linter, they represent the earliest stage able to detect errors. In a \ac{TDD} environment, the tests are even written before the application code itself to ensure that the application meets the project requirements. The use of unit tests remains unchanged in a microservice architecture, as each service can still have its own unit test cases. Due to the number of applications, though, \ac{IPC} in a microservice architecture increases significantly, making integration and higher testing scenarios more complex \cite{microtest}. \newline
        Integration tests should verify that two applications can successfully interact with each other. In order to perform these kinds of tests, it is necessary to run both applications simultaneously \cite{azuredevops}. Triggering an \ac{API} event and verifying its result reveals integration errors, but makes them as complex as end-to-end tests thanks to the configuration and orchestration of multiple services \cite{microtest}. While developers can use tools such as \wordhighlight{Postman} to check the results of an \ac{API} call, this does not guarantee that two applications can actually communicate successfully. Integration tests can either be done in \ac{CI}, where an error can only be detected after the code has already been committed and pushed and \ac{CI} tasks completed, or locally with immense effort for the entire configuration of all services. Due to the high rate of inter process communication, the use of tracing tools may be necessary to isolate an inter-application error. Integrating tracing programs into \ac{CI} and accessing them locally is another significant technical task. This results either in a very late error detection, which slows down development and makes it more expensive, or in an individual, local environment that is difficult to maintain and prone to errors. Both scenarios leave developers with insufficient testing capacities for efficient software development. Although contract testing for compliance with shared \ac{API} specifications can be used for integration testing, this merely postpones the problem to a later test stage \cite{microtest}.\newline
        Entire application tests, called end-to-end tests, are extensive and time-consuming. They require all applications to be deployed to cover entire business logic operations \cite{microtest}. They should be used thoughtful and performed on a testing or staging environment. However, these only form a small part of the test scope and are typically performed by a separate \ac{QA} team. When an error is detected, its cause must be found, for which tracing and debugging programs are used. The absence or previous configuration of these tools complicates the work and is one of the problems already described above. The lack of tools and too complex, differentiated environments are also one of the main challenges in implementing DevOps practices \cite{devops_challenge}.

        \begin{table}[]
            \centering
            \begin{tabularx}{0.9\textwidth}{lX}
                Test-Type & Description \\ \midrule\midrule
                Unit test& Test a small part of a service, such as a class.\\
                Integration tests & Verify that a service can interact with infrastructure services \\
                Component tests & Acceptance tests for an individual service. \\
                End-to-end tests & Acceptance tests for the entire application.
            \end{tabularx}
            \caption{Types of Software Tests, \\\textit{Source:~\cite{microtest}}}\label{tab::tests}
        \end{table}

        \subsubsection{Issues Caused by Heterogeneous Environments}\label{sss::hetero}
        According to the StackOverflow Survey 2021, Windows is the most often used operating system for software development, but in the server area, Unix-based systems clearly dominate with a market share of 75.3\% among webservers \cite{stackoverflow2021}, \cite{unixusage}. The current state of affairs is thus obvious: The development takes place on Windows, whereas the actual operation occurs on Linux. This fundamental difference in the base runtime environment of applications can be the cause of a variety of operating system specific errors, even if the used programming language supports cross-platform compatibility. One major difference is the structure and functioning of the file system. Windows uses alphabetical identifiers like \code{c} and \code{d} for drive partitions. Linux, on the other hand, has a root directory (\code{/}), in which all underlying folders and partitions are placed. External partitions can be mounted at any place and with any name. A frequently used directory for external partitions is to be found in the \code{/mnt} directory. Accordingly, the representation of file paths also differ. While user data under Windows can be accessed via \code{C:\textbackslash Users\textbackslash USERNAME} using backslashes, it is usually accessible via \code{/home/USERNAME} under Linux, with a normal forward-slash character. The inclusion of external libraries, assets and other files via paths is a common task in programming. Hardcoding these paths can lead to unexpected behavior on a different host. Although some programming languages offer constructs such as \code{File.Separator} (Java), or attempt to perform the path-separator conversion automatically (NodeJS and Python), errors still occur on heterogeneous systems. One of the reasons is the special meaning of the backslash, which is often used to escape reserved special characters.\newline
        In a way similar to the file paths, the encoding of the newline character differs. While Windows uses, by default, the \code{CR} line ending format, Linux the \code{LF} format. There are applications, which can read both encodings, but nevertheless, there are also applications that either only support \code{CR} or that can only read newline characters encoded with the \code{LF} format. Bash scripts created under Windows cannot be executed under Linux without a conversion from \code{CR} to \code{LF}. The permission system of Windows and Linux also differs. Windows uses the central uses the central user account control to manage the processes of reading, modifying and changing the owner and to delete permissions. Linux, on the other hand, has a read, write and execute flag every file, determining owner, group and other permissions. Windows does not support the executable flag at all. Accordingly, all files created under Windows, which are then copied to Linux environment, cannot be executed without further steps. Another example for these problems is the handling of keys and certificates. The widely used key-based cryptosystem \ac{RSA} can be used under both Windows and Linux, yet differently. \ac{RSA}-Keys created under Windows are not accepted by many Linux applications because they cannot be read due to \code{CR} line endings or are rejected because \wordhighlight{nginx}, \wordhighlight{apache webserver} and \wordhighlight{sshd} require that private keys are only readable by the user and the permissions are too open by default.\newline
        These heterogeneous based problems occur in addition to \acl{OS} specific programming. Low level operations, such as forking a process, spawning a new one and sending (exit) signals, are fundamentally system-specific. Higher level programming languages abstract some of these operations, yet, some functions are only available on one specific system such as Unix sockets. Dependencies with native libraries are also \acl{OS}-specific. They either have different variants for each system or they are compiled into native libraries on the target system at install time, as to be seen in the cases of NodeJS and \wordhighlight{node-gyp} or Python and \wordhighlight{wheels}, respectively. This deviating behavior adds extra complexity and creates new error sources.

        \subsubsection{Additional Effort for Developers}
        the preceding sections already show how heterogeneous environments, many small services and an extended testing effort cause additional work for developers. In addition to that, there is also the aspect of the management of secrets such as, \ac{API}-tokens or keys, and, for new developers, the setup of \ac{VPN}s and communication tools. Even when the local environment works without errors, there are ongoing obstacles. As mentioned before, the value of microservice is the multiplicity of small applications, which together provide a greater functionality. In order to start multiple microservice developers most likely need multiple terminals to execute each application. A larger amount of terminal sessions can quickly become confusing. Eventually, services even have a specific order for startup. This phenomenon is also referred to as terminal hell. However, there are still obstacles to overcome after the start of the individual applications; tracking their output is also one of the tasks. Log outputs provide insights into what the application is currently doing. This makes it possible to quickly check the expected behavior of an application. The output of many logs on different terminals, however, only contributes to a rapidity increasing confusion. Analogous to the terminals, this is called log hell \cite{micro}.\newline
        These circumstances lead to developers spending more time managing, configuring and analyzing applications rather than actively developing them, which leads to a slower pace of development. Slower development increases cost and leads to customers having to wait longer for new features and bug fixes, which diminishes the customer experience. If a company's business model is to offer software development as a service, slower and higher development costs mean that the company cannot compete with rival companies and loses contracts.

        \subsection{Proposed Solution for more Efficient Workflows}
        The root cause of these problems is that local development environments fundamentally differ from production environments. Local environments are deeply individualized and personalized, while production environments are scandalized but very complex. Local environments cannot be set up automatically and thus cannot simply be torn down and replaced in the event of errors. Long troubleshooting sessions are the consequence resulting in a slow-down of the development progress.\newline
        In the server world, virtualization technology is used for uniform and well-defined environments. A consistent and isolated application runtime environment can be provided by a container based virtualization solution such as Docker. In order to allow interaction between applications, they can be orchestrated and scaled with via a Docker Swarm or Kubernetes cluster.\newline
        This thesis proposes to use exactly this containerization approach for local development environments. Thus, the configuration effort, the lack of testing possibilities and the occurrence of local and operating system specific errors should be reduced.
        % With tools like Terraform or Ansible, any number of server instances can be created in a short time in exactly the same configuration.


% Solution concept
% !TeX root = ../thesis_main.tex


\section{Solution Concept of DevContainers}\label{sec::solution_concept}
This section proposes the usage of Development Containers (DevContainers) as a solution to the problems of local development environments described in Section \ref{sec::problem}. The concept is oriented according to the \ac{PaaS} principles known in the cloud. Developers should only have to deal with the application. The runtime environment and network configuration are managed by the DevContainer. They combine the application runtime and its configuration into an isolated environment by using the lightweight virtualization approach of containers. One difference between DevContainers and the \ac{PaaS} principle is that DevContainers do not provide own hardware neither its management. \newline
The details of such a DevContainer concept are described below. This is followed by a delimitation of when its application scenarios make sense and when they do not. Finally, the advantages and limitations of this solution are outlined.

    \subsection{Description of a Conceptual Environment}
    The idea of DevContainers is to bundle the application code, its runtime and configuration into an isolated system. Only the minimal necessary scope for accessing the application is made available on the host system, all other resources remain isolated. According to the design of \ac{VM}s and containers, they can be started and stopped at will without the having the risk of contained applications causing permanent changes on the host system. When the container is started, all port, path and secret configurations are already set. Developers are not required to do any further manual configuration and additional steps. Different branches and versions of the applications all have their own container which are completely independent of the local settings. Auxiliary applications and interdependent services can all be started simultaneously and in the correct order. The DevContainer environment builds upon the Linux kernel and is completely independent of the host \ac{OS} and thus much closer to the production environment, potentially decreasing system specific errors. Even in case something goes wrong and the environment ends up in an undefined state, the principles from the server world can be applied. The DevContainer can simply be discarded and recreated from a well-defined template within seconds. Which even can enable environments that are 100\% reproducible.\newline
    Such a setup allows developers to choose any host \acl{OS} because all the application code is in a virtualized environment. It increases the initial setup time and prevents configuration drift because all configuration settings are following the \ac{IaC} principle and are stored as code. This way, new runtimes or dependencies can be tested without having to risk corrupting the local environment. Through automatic orchestration, integration tests become easier and can be performed earlier, shortening the time until an error is detected. Dependencies and common software like debugger are already present in the container, so they do not need to be installed separately. DevContainers promise to solve the problems described in Section \ref{sec::problem} and allow developers to focus on programming rather than configuring and maintaining their working environment.\newline
    Despite these potential advantages, DevContainers are not the miracle solution to all problems in software development. They have their limits and their area of application wich they are suitable for.

    \subsection{Pre-requirements for DevContainers}
    Before the use of DevContainers one has to verify hat these are the right approach for the encountered problems and that all necessary requirements are fulfilled.\newline
    DevContainers are based on virtualization technology, and one of their goals is to achieve the greatest possible similarity between the development and production environments. In order to take full advantage of this possibility, the production environment must already be designed for the use of virtualization with containers. As can be seen in Section \ref{ssec::toolsused}, the majority of virtualization solutions are based on Linux, which is also the most widely used system in the server domain. Applications that require Windows can also use Windows-based containers, but these may require additional licenses and configuration, accordingly they will not be discussed further in this paper. The application to be developed must be suitable for use in a container. Containers do not offer direct support of graphical output by default. Functions are exposed to the outside world via the usage of sockets or mounted devices. Experience in the area of virtualization and a functioning \ac{CI}/\ac{CD} pipeline for the rapid delivery of new application versions is therefore also recommended.\newline
    Although DevContainers reduce the configuration effort for each developer, the architecture and settings files for the use of DevContainers must be created once and then be maintained. As Section \ref{sssec::virtual} already shows, with any kind of virtualization, regardless of whether it is VMs or containers, there is a performance overhead. If, as in a typical microservice architecture, several applications are run at the same time, this overhead adds up and places an additional load on the developer's system. Modern hardware with sufficient memory and computing capacity is therefore necessary for the use of DevContainers. The amount of additional load depends on the technology stack used.

    \subsection{Creating a DevContainer setup}\label{ssec::toolsused}
    Section \ref{ssec::getting_devops} already presented concrete programs for providing the virtualization concept and automating certain processes with \ac{CI}. These form the basis for the provision of DevContainers. This must now be made available to developers in actual concepts and, in doing so, solve further problems described in Section \ref{sec::problem} as good as possible. Developers need to be able to initialize a DevContainer based environment quickly and must be offered a way to interact with the DevContainer, especially the application within, without significant impacting on their workflow. Therefore, tools and concepts that make this possible are presented below.

        \subsubsection{Defining the Application Runtime}
        The entire application runtime environment is virtualized using Docker. However, the basics of this runtime environment must be defined beforehand. In the case of Docker, this is done with Dockerfiles. These provide the build instructions for the Docker-images that are needed for a container in order to start. \ac{CI} platforms typically execute the building process of images automatically \cite{docker2020}.\newline
        % !TeX root = ../thesis_main.tex

\begin{lstlisting}[language=docker, frame=single, caption={Python DevContainer Dockerfile},label=code::docker_dev_example]
FROM debian:buster
RUN apt-get update && apt-get install -y \
    git python3 python3-pip make vim emacs && \
    pip3 install pandas numpy matpltlib

#Optional
Copy . /app
\end{lstlisting}

        Listing \ref{code::docker_dev_example} shows such a Dockerfile for a Python application. The Linux distribution Debian is chosen as the starting point for the image, followed by installation of all required dependencies and programs needed. There are two possibilities for implementing a DevContainer concept, when it comes to the source code. Optionally the program code can be copied into the image so that the image already contains all program components. The resulting fully self-contained image is the typical method in production operation. The disadvantage of this approach is that the program code can only be accessed within the container and is not present on the host system. While this is desirable in production, due to security and strict isolation, it results in additional development overhead. Developers can only use applications within the container to edit the code and the \ac{CI} system must create new images for every change. Prematurely discarded containers can lead to the loss of changes that have not yet been published to the remote source code repository \cite{dockerdocs}.\newline
        Alternatively, copying the program code can be omitted and only the well-defined runtime environment is provided in the image. The program code on the host system is then mapped into the container via a bind-mount at the start of the container. So the program code can be edited and used by any editor on the host. In the case of a discarded container, the changes made are still available on the host. Furthermore, a new image only needs to be created when something changes in the Dockerfile or dependencies, which saves a lot of computational effort in the \ac{CI} system. After the build process all images are than made available on a private or public image registry in oder to be accessible. In the working directory of the application the following commands are executed by the \ac{CI} system in order to create, name and upload the image to a registry:
        \begin{lstlisting}[language=bash, frame=none, numbers=none, backgroundcolor=\color{codebg}]
docker build -t my-python-app .
docker push my-python-app
        \end{lstlisting}
        \vspace{-1cm}

        \subsubsection{Orchestrating the Application Containers}
        In order for the Python application above to work, it needs a database. Instead of each developer having to install and set up a database on their own, which may then shared between different projects, this can be done in isolation and automatically for each project thanks to virtualization and composition of programs. Applications in an \ac{MSA} have even more services on which they depend on. Arranging all these programs into one greater service is called orchestration. Well-known applications for container based orchestration are \wordhighlight{Docker-Compose} and \wordhighlight{Kubernetes}. Since Kubernetes, does not ship with Docker by default, is quiet complex and production oriented, Docker-Compose is suggested for local orchestration.\newline
        Docker-Compose requires a configuration file, declared in the \acs{YAML} format, which contains all necessary information for orchestrating multiple applications. Each application is defined as a service with a unique name when using Docker-Compose. The \code{docker-compose.yml} file in Listing \ref{code::compose_example} defines the services \code{app} and \code{db}. Each service has an image which is the initial state for each container. With Docker containers, it is common to use environment variables to influence and configure the application within the container. The initial credentials for the database service are set via environment variables and the python app uses them to set the connection information to the database server. It should be noted that the database host can be simply the service name of the database server instead of an IP address. Within the Docker managed network, the integrated DNS server automatically resolves the service names to the IP address of the \ac{DB} server. This virtual network allows the containers to communicate with each other, but in order for the functionality to be available on the host, the network ports used must be explicitly exposed to the host. The Python application uses the alternative \acs{HTTP} port \code{8080} and the database server uses the standard MySQL port. When starting the containers, the corresponding ports are allocated on the host system and local programs can access the defined services using \code{localhost} and the corresponding network port \cite{dockerdocs}.\newline
        If the program code is not already copied into the image it must be mapped into the container using bind mount point. Line 11 in Listing \ref{code::compose_example} binds the local \code{app-src} directory from the host into the container at the location \code{/workspace}. The database service, on the other hand, uses a volume mount in which docker manages the allocated memory itself and persistently. The configuration is completed with an entry-point for the python app that will be executed when the container is started.\newline
        % !TeX root = ../thesis_main.tex

\begin{lstlisting}[language=docker-compose-2,caption={Exemplary Python Project \code{docker-compose.yml}},breaklines=true,label={code::compose_example}]
services:
  app:
    image: my-python-app
    entrypoint: python3 app.py
      - DB_HOST=db
      - DB_PW=yes
      - DB_USER=root
    ports:
      - 8080:8080
    volumes:
      ./app-src:/workspace

  db:
    image: mysql
    environment:
      - MYSQL_ROOT_PASSWORD=yes
    ports:
      - 3306:3306
    volumes:
      - sql_data:/var/lib/mysql
volumes:
  sql_data:
\end{lstlisting}

        Docker-Compose ensures interdependent services are started in the correct order, all persistent volumes are created, and all containers can communicate within their network \cite{docker2020}, \cite{dockerdocs}. Since the all necessary configurations are stored in an ordinary text-file, the \ac{IaC} principle can be applied here. The state all changes of the \code{docker-compose.yml} file are tracked in \ac{VCS}. Accordingly, even development work requiring changes to the runtime can take place in a separate branch without affecting the work of other developers.\newline
        Starting all services specified in the \code{docker-compose.yml} file is done with the console command \code{docker-compose up}. If the docker images used are not already present on the host, they are automatically downloaded from available image-registries. When all services are started, Docker-Compose attaches itself to all running containers and outputs all program output of each container color-coded to the console.\newline
        % \begin{lstlisting}[language=bash, frame=none, numbers=none, backgroundcolor=\color{codebg}]
        %     docker-compose up [app db]
        % \end{lstlisting}
        Through this concept, developers have isolated program environments that can be created quickly, easily and producible. Projects running in parallel no longer have to share a database-servers and different runtime environments, interpreters and compilers can be tested independently. In case a fast cache-server is needed, another service can be created quickly in order to provide a \wordhighlight{Redis} or \wordhighlight{Memcached} server without the need of installing it to the host. Individual, manual and diverging configurations for each developer are eliminated. New environments that are similar to the production environment can be created quickly, independent of the host operating system. Even in a microservice architecture, developers have the opportunity to test the interaction of multiple applications before committing changes to the \ac{VCS}. In this way, possible errors are already detected before CI integration tests, which means that they can be corrected more quickly. These characteristics promise to solve the problems described in Section \ref{sec::problem} regarding heterogeneity, lack of testing facilities, and the tedious configuration of the setup. In order for this concept to be adopted by developers, there must be an equivalently effective way for developers to interact with their applications within the DevContainer.

        \subsubsection{Interacting with DevContainers}
        The primary way developers interact with their code and the application is through their editor. Their variability and the number of different editors is great. Well known general purpose editors are Visual Studio, XCode, Atom, Sublime, Eclipse, Emacs and VIM. There is usually a distinction between simple text editors and \acl{IDE}. While text editors are quite simple and only provide basic functionalities like syntax highlighting, \ac{IDE}s are much more comprehensive with powerful IntelliSense suggestions, built-in project management, \ac{VCS}, debugger, graphical visualization and build-tools. The choice of the editor is a personal decision for most developers, and they are customized according to their preferences. For this reason, the proposed solution does not require a specific editor for DevContainers, but gives a recommendation that will be used as a reference for the rest of the work.\newline
        \ac{VSCode} is a free platform independent editor with extensive extendibility which is used by over 70\% of all developers accordingly to StackOverflow~\cite{stackoverflow2021}. One of its key beneficial features for virtualized workloads is its Remote Development functionally. This, officially provided, extension enables the intuitive comfort of a graphical \ac{UI} while running the code, the application and auxiliary processes like a debugger on another, remote machine. Figure \ref{fig::vscodecontainer} shows how the \ac{VSCode} frontend connects to a remote machine or container and installs a server instance of the editor. The server side instance manages access to the remote file system and the execution of processes while communicating with the local \ac{VSCode} frontend instance for comfortable access to these functions. This type of remote development works for \ac{SSH} connections, the \ac{WSL}, and on Docker containers. \ac{WSL} is a built-in Windows feature to provide a Linux environment on Windows hosts without the need of a separated \ac{VM}. The implementation of Docker for Windows is build upon this functionality \cite{vscodedevcontainer}. In the further course, only the remote functionality for containers will be considered.
        \begin{figure}[]
            \centering
            \includegraphics[width=.95\textwidth]{architecture-containers.png}
            \caption{Architecture of \ac{VSCode} Development Container Setup \\\textit{Source:~\cite{vscodedevcontainer}}}\label{fig::vscodecontainer}
        \end{figure}
        In order for container services to be accessed, the appropriate ports are exposed via Docker-Compose, \ac{VSCode} provides a similar feature through the remote extension. Remote processes on one host are automatically made available locally through port forwarding. Since this functionality is not available for multiple containers simultaneously, this feature is only useful for a service which is currently being worked on or services with dynamically changing network ports. However, this feature can be used to quickly make a short-lived process available on the host without having to modify the configuration in the \code{docker-compose.yml} file. This allows the usage of local testing or exploration tools on any remote services.\newline
        In order for \ac{VSCode} to connect to or start a DevContainer, a configuration file is required. This is provided by a \code{devcontainer.json} file. For each service being developed, the path to the \code{docker-compose.yml} file is specified and to which service \ac{VSCode} should connect to. Listing \ref{code::devcontainer_json} shows such a configuration file. It also specifies which directory the \ac{VSCode} server should open, which extensions have be installed and what happens when the container starts or stops. If this file is present, \ac{VSCode} automatically offers to reload the local project using the DevContainer. All services are started automatically, directories are mounted, and the network ports are allocated accordingly. Thus, the development work is nearly identical to a local setup \cite{vscodedevcontainer}.\newline
        % !TeX root = ../thesis_main.tex

\begin{lstlisting}[language=json,caption={\ac{VSCode}s Container Configuration File \code{devcontainer.json} },breaklines=true,label={code::devcontainer_json}]
{
  "name": "MY-PYTHON-APP",

  // Path to the docker-compose.yml file.
  "dockerComposeFile": ["../../docker-compose.yml"],

  // The service property for the container that VSCode uses.
  "service": "app",

  // The project folder to be opened by VSCode when connected.
  // Keep your containers running after VS Code shuts down.
  "workspaceFolder": "/workspace",
  "shutdownAction": "none",

  // Extensions to be installed when the container is created.
  "extensions": [
      "ms-python.python",
      "ms-python.vscode-pylance"
  ],

  // Runs this command after the container is created
  "postCreateCommand": "/workspace/.devcontainer/kill_app.sh"
}


\end{lstlisting}

        All these functions can be archived without \ac{VSCode} by using remote filesystem mounts, (\ac{SSH}) port-forwarding or terminal based editors. Even without VSCode, any file changes made by any editor in the container will take effect, since the source code directories are bind-mounted. Nerveless, it cannot be ruled out that when using editors other than \ac{VSCode}, adjustments to the \code{docker-compose.yml} file may are necessary compared to the setup described in Section \ref{ssec::imp_approach}.

        \subsubsection{Variations and Additional Supporting Tools}
        In addition to the tools mentioned above and in Section \ref{ssec::getting_devops}, other auxiliary programs can be used. To simplify certain workflows and automate recurring tasks scripts will be used. The \wordhighlight{Bash} scripting language can be used natively on Linux and macOS, the installation of Git for Windows also brings Bash support to Windows. Accordingly, one uniform scripting language can be used to perform platform-independent operations and to simplify complex instructions. In order that developers do not have to wait for the creation of the DevContainer images, these should be built automatically by a Ci and made available in a private container registry. Accordingly, uniform and up-to-date images are always available to all developers. \newline
        It is also possible to use other management tools via Docker. Database management tools such as \wordhighlight{PHP-MyAdmin} can simplify administration by adding further services. The same applies to the management of containers via \wordhighlight{Portainer}. Although Docker-Compose offers a color-coded log output, even this can become overwhelming if there are too many logs. Since the Docker stack is used anyway, enterprise log aggregators and analysis tools like \wordhighlight{Grafana} or \wordhighlight{Elastic-Search} can be used to get a persistent and searchable log dashboard.

    \subsection{Security Aspects of this Concept}\label{ssec::sec}
    The presented solution is oriented to the development process and not to productive operation. For this reason, common security practices for container operation are not implemented. The images contain extra programs that are not mandatory but convenient. The processes in the container run as super-user to avoid additional configuration effort and interruptions due to access errors. In the \code{docker-compose.yml} file, passwords are defined in plain text to allow consistent and easy setup across any number of systems. Functions and data within the containers are made available through exposed ports on the host. Accordingly, it must be ensured that the host is not accessible from the Internet. If certain services such as databases or log dashboards are shared between developers through a central server, it must be ensured that this server is only accessible within the company network in order to avoid the unintentional publication of confidential information. The same security measures must be applied as for a non-containerized local development environment.

    \subsection{Strengths, Weaknesses and Limits}\label{ssec::limits}
    The software and methods described in Section \ref{ssec::toolsused} are already standard practice in DevOps enabled teams. Accordingly, the entry barrier is small in compared to new and unknown tools. DevContainer allow for a homogenization of development and production environment. The application runtime is identical, accordingly \ac{OS} or runtime specific errors are prevented. Developers can use a ready to go development project setups without having to install and configure the application runtime themselves. Dependencies, keys, supporting tools like debuggers and configurations can already be shipped within the container to enable a quick initial setup. Instead of examining the environment for a long time in the event of an error it can quickly been torn down and recreated into a known good state. These are the key features that DevContainer promise to provide, in addition to solutions to the problems described in Section \ref{sec::problem}. They extend the existing toolset of agile and DevOps teams with another tool that allows developers to be more flexible and focus more on coding.\newline
    It should be noted that DevContainers are not a perfect solution to all problems in the software development stack. Like any other tool they come with their own set of quirks. Expertise for the use of Docker containers must be available, and the production architectural must be transformable to a local DevContainer-based configuration. This configuration must be kept up to date and be maintained. Developers may need to adapt to minor adjustments in their workflow. The \ac{CI}/\ac{CD} solution used must always be available and provide up-to-date container images. Every virtualization approach creates an additional overhead that cannot be ignored, especially when using multiple containers on Windows based systems. Details of the exact effects are given in Section \ref{sec::eval}.\newline
    As mentioned before, there are also projects for which the use of DevContainers is simply not suitable. Heavy monolithic applications are not in the sense of containers and therefore not suitable for DevContainers. Graphical applications can run in containers but require a graphical X-Server on the host and the resulting experience is not comparable to a native \ac{UX}. Similar limitations apply to applications that require a Windows stack. This can be implemented, but is no longer platform-independent, requires additional licenses and further adaptations. For embedded projects that require specific hardware, a virtualization approach is just not feasible and accordingly these are also not suitable for DevContainers.\newline
    To demonstrate a suitable use of DevContainers, the next section describes how to migrate from a traditional development environment to a DevContainer-based solution based on a real project.

\subsection{Alternative Solutions}\label{ssec::alternatives}
Besides the solution concept presented here, there are already alternative solutions on the market that promise to solve similar problems as in section \ref{sec::problem}. A selecting of these solutions is presented in the following. These alternative virtualization-based development environments can be roughly classified into two categories. Purely browser based and container based solutions. A comparison of these to the solution presented here will be described in Section \ref{sses::eval_compare}.
\myparagraph{Browser based Development Enviorments}
Browser based development solutions provide an editor that is entirely built upon of web technologies. Therbey they can run on every device with a modern web-browser. Their goal is to provide a quick functioning setup without any configuration.\newline
The products codesandbox.io and Stackblitz implement this approach. Both solutions offer a \ac{VSCode}-like editor in the browser and allow the development of NodeJS based JavaScript projects.
While codesandbox.io provides the server-side infrastructure und functionality themselves and developers are dependent on an Internet connection, Stackblitz can also be used without an Internet connection. Stackblitz is a \ac{PWA} which brings the nodeJS runtime into the browser by using Webassembly. While codesandbox.io does not provide any console at all and the management is completely done by the service provider, Stackblitz provides a minimal shell for installing, copying and launching files. However, since both solutions run within the browser, it is not possible to open any regular TCP/UDP ports due to the security regulations of browsers. Network connections are only made available via the http based WebSockets protocoll \cite{codesandbox}, \cite{stackblitz}.

\myparagraph{Container based Development Enviorments}
Similar to the solution presented in this paper, containers are used in which theoretically everything can be installed. Compared to browser-based solutions, container-based development environments offer broader functionality and are not restricted on one specific programming language. The best-known products that implement this are GitPod and GitHub's Codespaces, which was not released until 2021. The infrastructure for the containers is provided by the provider, but GitPod also offers a self-hosting option. There are no restrictions for users in the containers, additional software can be added as desired, and regular network ports can also be opened to the outside. Both solutions can be used in the browser as well as in a local \ac{VSCode} installation via the remote development extension.\newline
Both services are billed either on a monthly basis or on an hourly usage basis \cite{githubcodespace}, \cite{gitpod}.


% Exemplary implementation
% !TeX root = ../thesis_main.tex

\section{Prototype Implementation}\label{sec::solution_code}
This section describes the implementation of the DevContainer concept in a real project that is currently developed by the Symbic GmbH. At the beginning, the initial state of the project is described as well as the goal to be achieved. Subsequently, the concept from section \ref{sec::solution_concept} is applied and an architecture for the DevContainer setup is designed. The following section describes the implementation process and puts special emphasis on potential errors and which catches there are to minimize them. Finally, the archived state is compared to the previously set goal.

    \subsection{Project Information and expected Target State}
    The project presented here is currently developed and maintained by Symbic GmbH. It is a system for deploying and managing \ac{IoT} devices from the agricultural sector and is based on a microservice architecture. The \ac{IoT} devices are used on agricultural machinery to assists vehicle operators in their work by providing an interface between the lead vehicle and trailed machinery. Vehicle and harvest information can be displayed and the position of the vehicle is tracked. New interfaces can be installed on the devices and existing ones can be adapted, while also providing a web interface to display the collected device data. Various \ac{API}s provide different functionalities, which are made available by multiple microservices. In order for the system to operate, various auxiliary services are required, including \ac{SSH}-servers and MQTT-brokers. For the brevity and clarity of this paper, only a subsection of the system is presented in this thesis. This makes a further, more realistic implementation possible without going into an excessive description of the system which would have a detrimental effect on the concept of DevContainers. A detailed description of the projects architecture is given in the following section. \newline
    The goal is to test the concept described in section \ref{sec::solution_concept} for applicability on a real project, any encountered challenges are described below and possible solutions to them are given as well. In the end the resulting solution is compared to the initial expectations described in section \ref{sssec::goal}.

        \subsubsection{About the Prototype Project}\label{ssec::project}
        % !TeX root = ../thesis_main.tex
\begin{figure}[]
    \centering
    \tikzstyle{block} = [rectangle, draw, fill=green!80!blue!70,
    text width=5em, text centered, rounded corners, minimum height=4em]
    \tikzstyle{line} = [draw, very thick, color=black!50, -latex']

    \begin{tikzpicture}[align=center, scale=2, node distance = 5cm, auto]
        % Place nodes
        \node [] (user) {\includegraphics[width=.08\textwidth]{fig/user.png}\\User};
        \node [block, left of=user] (webapp) {WebApp};
        \node [block, below left of=webapp] (authbackend) {Backend};
        \node [block, below right of=webapp] (deviceapi) {Device API};
        % \node [block] (authbackend) at (1,1) {authbackend};

        % Draw edges
        \path [line] (user) -- node [text width=2.5cm, align=center, above]{Interacts\\with} (webapp);
    \end{tikzpicture}
    \caption{IoT Web Service Architecture}\label{fig::arch}
    \label{fig::cd}
\end{figure}

        % % !TeX root = ../thesis_main.tex
\subsection{Figure Alternatives}
\begin{figure}[]
    \centering
    \includegraphics[width=.85\textwidth]{C4-Elements.png}
    \caption{IoT Web Service Architecture - [Alternative]}\label{fig::arch}
\end{figure} % Alternative
        \noindent As described above, the project builds upon a microservice architecture. This allows for the use of different technologies for individual applications. Communication between applications takes place via a platform-independent protocol. If the load on a part of the application gets critical, new instances of this application can be created with little effort.\newline
        Figure~\ref{fig::arch} visualizes the architectural structure of the entire service. Each green box refers to a standalone application and contains an icon with the technology used. Users of the entire service only directly interact via the web interface (WebApp). The web interface is created with the \ac{JS} framework \wordhighlight{Vue.js}, which generates a static bundle of HTML, CSS and \ac{JS} files, served by a static web-server. In order to access the system, users must authenticate themselves in the WebApp. This is done via the authentication backend (Auth-Backend), written in PHP. In addition to the authentication, this application also performs additional tasks, which will be neglected, for reasons of brevity. Provided the user is authenticated, the WebApp can display internal information and functions of all deployed \ac{IoT} devices. These dynamic information are provided by the NodeJS-based System-\ac{API}. All \ac{API} endpoints are implemented under the compliance with the \ac{OAS}. The \ac{OAS} allows easy discoverability and understanding of the \ac{API} for humans and computers. It also provides versioning support of \ac{API}s to avoid unexpected errors in other applications. These applications are part of the microservice cluster, delimited by the dashed frame, which is actively being developed by Symbic; in addition to the services listed here, the cluster consists of additional services.\newline
        Both, the Auth-Backend and the System-\ac{API}, have their own SQL databases that persistently stores relevant information. However, both applications write user events to a shared event database. MongoDB is used to provide an schema flexible NoSQL database. In order to send specific instructions to, or receive information from a \ac{IoT} device, the system \ac{API} communicates with an MQTT-broker via the NodeJS-based MQTT-Connector. The MQTT-Connector provides an unified \ac{REST} \ac{API} for publishing and subscribing messages over the MQTT protocol. On a larger scale, the MQTT-Connector is consumed by several applications and is therefore a standalone application. Since only a part of this project is considered here, the System-\ac{API} is the only application communicating with the MQTT-Connector. In addition to the MQTT-broker, other auxiliary services such as SSH servers are used, which are not discussed here for reasons of brevity. The MQTT-broker exposes a public endpoint offering various topics which devices can subscribe or publish to. Via this system the user can trigger a command in the web interface, which is then processed and logged by the system \ac{API}. Sequentially it is made available on the MQTT-broker via the MQTT-Connector. \ac{IoT} devices, that have subscribed to the corresponding topic, receive this command and execute it. The result is then published to another topic and can be accessed in the same way via the WebApp. Since the \ac{IoT} devices are connected to the Internet via SIM cards, and are therefore on a restricted mobile carrier network, they are not directly accessible from the Internet. Through the described system, the \ac{IoT} devices can execute commands from the outside without being directly accessible via the Internet.

        \subsubsection{Initial State of the Projects Development Environment}
        New developers joining the project have to clone all four repositories, install NodeJS 14, a specific version of the \ac{XAMPP} stack and must customize its configuration. Furthermore, the PHP project requires the package manager \wordhighlight{composer}, which must be installed separately. The configuration details must be taken from the documentation or are provided by a collaborator. In order to test the whole, unified system, developers also need to install a MQTT-broker and the MongoDB database. Both applications need an initial administrator user, which needs to be created beforehand, and the corresponding credentials need to be accessible by each \ac{API}, which is made possible through the creation of \code{.env} files - files, which is not tracked in the \ac{VCS}. Then, a database and its schema needs to be created. Eventually network ports have to be adjusted and secrets for signing tokens have to be created.
        Thus, to set up a new environment, developers must install the all required frameworks, set up databases, customize ports, and enter connection information in the applications \code{.env} files. These steps are necessary to get the project running properly, whereas the setup of developer tools like debugger has not been considered yet. Even in a small project, this can take some time, especially if one is not yet familiar with the project.\newpage
        \noindent Still, even experienced project members encounter challenges when working on such a project. Before the \ac{API}s can be started, it must be ensured that the databases and other auxiliary tools are already running. To start all four applications (WebApp, Auth-backend, System-API, MQTT-Connector) a developer needs four terminal sessions or a terminal multiplexer to operate all applications. The program's outputs (logs) are distributed over all four sessions and can neither be linked to each other directly, nor are they filterable, nor searchable. When everything is running, developers can contribute value to the project. The restrictions mentioned above do not generally hinder developers in their work unless an error occurs. A merge-request containing dependency or database schema updates conducted by a team member can cause the environment of others to break, resulting in a hard to debug error due to the lack of uniformity. The lack of consistent, reproducible setups is the cause for lasting troubleshooting sessions and the well-known iconic saying \textit{"But it works on my machine"}. This mirrors the problems of heterogeneous systems mentioned in section \ref{sec::problem}. It does not matter whether the individual development environments differ from each other or the development environment differs to the production environment. The sheer fact that there are always differences is the main cause of before-mentioned problems. \newline
        These selected problems emerged in just one project. When working on multiple projects, the problem scope increases accordingly. Different databases, runtimes and dependencies further contribute to a greater potential for conflicts and issues. The lack of project isolation may result in unintended side effects between projects that slow down the development process. To prevent these unnecessary slowdowns, the initial setup of an environment and its operation must be as homogeneous as possible.

        \subsubsection{Goal for the Target State}\label{sssec::goal}
        The goal is, by using DevContainers, a faster deployment of the project environments is archived, which is uniform among all developers and has greater similarity to the productive environment. This is made possible by the principle of virtualization, which provides a similar environment, regardless of the host. Accordingly, this should avoid system-dependent errors and eliminate the lack of reproducibility. Thus, any system can be used as host operating system, as long as it is supported by Docker. The DevContainer environment to be created is not intended to forcibly replace existing environments, but to provide an alternative solution that is also platform independent. New team members can quickly set up a working environment based on either Windows, Linux or macOS, while offering an alternative for developers with existing environments, which they can switch to gradually. Thus, a hybrid environment of the traditional development environments and DevContainer should be possible. Developers should also remain free in their choice of editors and utilities.\newline
        At any time, developers should be able to exercise control over the container runtime and the processes within the container. Thus, in the event of an error, they are independently able to replace a non-functioning container with a new one. The DevContainer environment should run locally and not require an Internet connection to external servers or services in order to be unaffected by Internet connection errors. To ease the transition to a DevContainer-based development environment, it must be possible to set up the environment with a single command, as well as to start all services with one command. Developers should not need to have explicit knowledge of how DevContainers work in order to use them.

    \subsection{Applying the DevContainer Approach}\label{ssec::apply}
    This section will apply the concept of DevContainer as described in section \ref{sec::solution_concept} to the Symbic project. First, the rough procedure of the process will be described, followed by the exact implementation and step-by-step descriptions. Problems, possible solutions and limitations of the concept will be described at the end.

        \subsubsection{Approach on the Project}\label{ssec::imp_approach}
        Regarding the goal of a fully comprehensive development environment for all microservices of the Symbic project, each service will get its own DevContainer, as is intended in a microservice architecture. This allows granular control over the individual applications and makes it possible to implement a hybrid operation of conventional running applications and DevContainers. For this purpose, a \code{Dockerfile} file, containing all the necessary instructions for the runtime environments and other developer tools for a pre-built image, must be created for each service. The source code is not copied into the container, but later included via bind-mounts. This prevents a sudden or accidental loss of changes. Images have to be built less often and can be shared between applications with the same runtime. The orchestration of individual services is then done with Docker-Compose. Both, the configuration settings of all services, defined in a \code{docker-compose.yml} file, and their Dockerfiles are then stored in \ac{VCS} according to the \ac{IaC} principle. Using the remote extension for \ac{VSCode}, developers can attach themselves to any container process and take full control over the system and the running application.\newline
        Windows is used as the starting point for the development of the setup, as this is the most widely-used operating system for development at Symbic. Once the setup is operational on Windows systems, it is tested on other operating systems, and adapted if necessary. After the Docker images have been created and tested, they will later be automatically built by the GitLab \ac{CI}. Finally, scripts will be developed for the initial download of all repositories and for the database initialization. The developed interaction of the tools and the created attributes and scripts are described in detail in the following section.

        \subsubsection{The Implementation Process}\label{ssec::imp_process}
        The next paragraphs will provide a step-by-step description of the DevContainer architecture's design, starting with the Dockerfiles for the images, over the creation of the \code{docker-compose.yml} file, up to the usage of the environment via \ac{VSCode}.
        \myparagraph{Creating the Dockerfiles}
        The microservices project described in section \ref{ssec::project} is based on two technology stacks. The WebApp, the System-\ac{API} and the MQTT-Connector are all based on the NodeJS runtime, whereas the authentication backend is written in PHP. Accordingly, two variants of Docker images are needed. As already described in section \ref{sssec::docker}, Docker images are created from Dockerfiles. Each statement in a Dockerfile creates a new layer, which allows building upon of already existing images.\newline
        As a basis for the NodeJS DevContainer the official \code{node:16-bullseye} image is used as a starting point. This Debian-based images was chosen because it also used in the production environment, comes from a trusted source and is updated regularly. Accordingly, it serves as the appropriate basis for the installation of additional tools, such as git, an ssh server, build tools, and the terminal-based editors vim and emacs. As already stated, the source code is not written into the image, because it is mounted into the container via bind-mounts at runtime. The complete Dockerfile for the WebApp, the System-\ac{API} and the MQTT-Connector can be found in Listing \ref{code::docker_dev_node}. The custom developed installation script used in the Dockerfile makes some recommended, low level adjustments to the system and can be found in the appendix.\newline
        % !TeX root = ../thesis_main.tex

\begin{lstlisting}[language=docker, frame=single, caption={NodeJS DevContainer Dockerfile},label=code::docker_dev_node]
# Node.js version: 16-bullseye, 14-bullseye, 12-bullseye
ARG VARIANT=16-bullseye
FROM node:${VARIANT}

# Install needed packages, yarn, nvm and other tools
COPY install-scripts/*.sh /tmp/install-scripts/
RUN apt-get update && bash /tmp/install-scripts/install.sh \
    && npm install -g eslint \
    && apt-get -y install python3 make vim emacs git ssh
\end{lstlisting}

        % % !TeX root = ../thesis_main.tex

\begin{lstlisting}[language=docker, frame=single, caption={NodeJS DevContainer Dockerfile},label=code::docker_dev_node]
# BUILD
ARG NODE_VARIANT="14-buster"
FROM mcr.microsoft.com/vscode/devcontainers/javascript-node:
    ${NODE_VARIANT}

WORKDIR /app
RUN apt-get update && apt-get \
    install -y --no-install-recommends \
    git zsh ssh python3 vim emacs
\end{lstlisting}

        However, as for the PHP application, the Dockerfile is more complex, since besides the PHP runtime and a webserver several plugins are needed that rely on C/C++ libraries which are not installed by default. After its installation, the \wordhighlight{Apache \acs{HTTP} Server} must be configured accordingly to be able to use these plugins. The official PHP Docker image is used as a starting point. It is especially built for PHP applications and already contains the Apache-\acs{HTTP}-Server and a package manager for installing PHP extensions. Line 2 in Listing \ref{code::docker_dev_php} specifies the PHP version used, followed by the installation of several development tools. Subsequently, the PHP extensions used for the app and their dependencies are installed. Since this application also provides static assets, this time, the source code is copied into the image, followed by the installation of additional PHP packages with the help of the \wordhighlight{PHP-Composer}. In the last step, the Apache \acs{HTTP} Server is configured and file permissions are adjusted. The image contains all needed resources in order to start the application and also includes several development tools such as the preconfigured \wordhighlight{Xdebug} debugger.\newline
        For both, the NodeJS image and the PHP image, the versions are passed to the Dockerfile as arguments. This makes it possible to influence the version of the used runtime when executing the build process. Thus, multiple images can be built from one Dockerfile without the need to adapt it and developers have the possibility to adapt new runtimes without much effort. During the process of creation, the images were created locally; whereas in productive use, the GitLab \ac{CI} is responsible for creating regularly updated images. The job for creating the images is described in more detail at the end of this section. The images developed are very extensive in order to offer the developers the greatest possible comfort. DevContainers also work with much narrower images that can be adapted to the needs of the developers over time.

        \myparagraph{Orchestrating the Microservices}
        The applications from the Symbic project illustrated in Figure \ref{fig::arch} need auxiliary services. The project requires an SQL-Database, a MongoDB-Server and a MQTT-Broker. By choosing a MySQL-Server, the Eclipse-Mosquitto-Broker and the official MongoDB-Server, free pre-built, official Docker images are available for all these applications. Listing \ref{code::compose_helper} shows how these services are configured in a \code{docker-compose.yml} file in order to be usable for other applications. Each application is defined as a service, containing their corresponding Docker image as an attribute. In order to make the applications usable outside the virtual docker network, the internal network ports are exposed to the host system (see line 7, 16, 22 in Listing \ref{code::compose_helper}). To ensure that the data stored in the container is not lost when it is renewed, volumes are used to store application data persistently on the host. For this purpose, the name of the volume followed by the mount point of the volume in the container is specified in the services volumes property. Typical for Docker, the initial credentials are configured by means of environment variables. In contrast, the Mosquitto-Server is configured by bind-mounting configuration files from the \code{manage-reop/config} folder to \code{mosquitto/config}.\newline
        % !TeX root = ../thesis_main.tex

\begin{lstlisting}[language=docker-compose-2,caption={Auxiliary Services docker-compose.yml},breaklines=true,label={code::compose_helper}]
version: "3"
services:
  db:
    image: mysql
    environment:
      - MYSQL_ROOT_PASSWORD=yes
    ports:
      - 3306:3306
    volumes:
      - db_sql_data:/var/lib/mysql

  mongo:
    image: mongo:4.4-focal
    environment:
      - MONGO_INITDB_ROOT_USERNAME=root
      - MONGO_INITDB_ROOT_PASSWORD=none
      - MONGO_INITDB_DATABASE=dm-cu-local
    ports:
      - 27017:27017
    volumes:
      - db_mongo_data:/data/db

  mosquitto:
    image: eclipse-mosquitto
    ports:
      - 1883:1883
    volumes:
        - mosquitto_data:/mosquitto/data
        - ./manage-reop/config:/mosquitto/config
\end{lstlisting}

        Without applying the concept of actual DevContainers, the configuration in developed Listing \ref{code::compose_helper} already enriches the developer experience. Developers do not have to install and configure extra auxiliary programs, but can make them available quickly, uniformly and platform-independently with one command.\newline
        In order to use DevContainers, the individual applications must also be defined as services. Listing \ref{code::compose_service} demonstrates this for the System-\ac{API}: the name of the DevContainer image is used as a starting point, just like in the previous section \ref{sec::solution_concept}. Subsequently, the credentials for the auxiliary services are made available to the application via environment variables. Since the source code is not included in the image, it is mapped by a bind-mount into the container (see line 15). Now, all the necessary requirements for starting the application in the DevContainer are fulfilled.\newline
        Since the project is a NodeJS application, all dependencies are stored locally in the project directory under the \code{node\_modules} folder. These dependencies are operating system specific, which is why it may happen that the NodeJS application does not start when developing on a Windows host while the DevContainer is based on Linux. Developers would have to delete and reinstall all dependencies whenever the DevContainer is not in use. However, this can be avoided by using the Docker overlay file system. A volume managed by Docker is created, which overlays the local dependencies mapped via a bind-mount (see line 16 in Listing \ref{code::compose_service}). This makes all dependencies used within the DevContainer independent of those installed locally. For other programming languages not storing their dependencies in the project directory, this is not needed. A volume is also used for the extensions installed by VSCode, which avoids that extensions have to be reinstalled every time the image is updated.\newline
        When the service is started, its entrypoint is executed. For the NodeJS applications, this is a custom developed script similar to Listing \ref{code::entry}, which installs all the necessary dependencies via npm, if not already present, and then starts the application in development mode via nodemon. The main process of the container is an infinite loop in order to keep the container running even if the developed applications exits. The complete \code{docker-compose.yml} file with all application services and auxiliary services can be found in the appendix under Listing \ref{code::compose_service_all}.\newline
        % !TeX root = ../thesis_main.tex

\begin{lstlisting}[language=docker-compose-2,caption={Auxiliary Services \code{docker-compose.yml}},breaklines=true,label={code::compose_service}]
services:                       # Exemplary service
  system-api:                   # configuration
    image: system-api/dev-container
    entrypoint: "/workspace/.devcontainer/entrypoint.sh"
    environment:                # Configure settings
      - PORT=8090               # for auxiliary services
      - MONGO_DB_HOST=mongo
      - MONGO_DB_USER=root
      - MONGO_DB_PW=none
      - SQL_HOST=db
      - SQL_USER=root
      - SQL_PASSWORD=yes
      - SQL_DB_NAME=local-device-db
    volumes:
      - ./system-api:/workspace
      - system_api_node_modules:/workspace/node_modules
      - vscode_extentions:/root/.vscode-server/extensions
    ports:
      - 8090:8090
volumes:                        # Persistent storage
  system_api_node_modules:      # managed by Docker
  vscode_extentions:
\end{lstlisting}


\begin{lstlisting}[language=bash,caption={DevContainer \code{entryscript.sh}},breaklines=true,label={code::entry}]
#!/bin/bash
NODE_MODULES=/workspace/node_modules && cd /workspace

# Check if modules are present
if [ -z "$(ls -A ${NODE_MODULES})" ]; then npm install; fi

# Start the main process in background & remember the pid
./node_modules/.bin/nodemon index.js&
MAIN_PROCESS_PID=$!

echo -n $MAIN_PROCESS_PID > /tmp/pid.tmp
while sleep 1000; do :; done
\end{lstlisting}

        \myparagraph{Enabling Development}
        Following the procedure above, all applications are started in development mode without the need to open an editor. Changes to the code, made by any editor on the host, directly take effect to all NodeJS applications, since \wordhighlight{nodemon} enables hot-code-reloading. This may, however, not be possible in every programming language and is not certainly supported by compiled languages.\newline
        If the developer demands more control over the application, he can open it in the DevContainer via \ac{VSCode} or start an interactive shell to the container via Docker. On startup \ac{VSCode} terminates the, previously automatically started, process via its \ac{PID} and the developer now has complete control over the application via the built-in terminal. However, in order for the container not to exit, as intended when the main container process terminates, a process must continue to run to keep the container alive. For this reason, the entry-script executes an infinite loop at the end. See line 12 in listing \ref{code::entry}. To determine which container \ac{VSCode} connects to, a \code{devcontainer.json} file is required. Listing \ref{code::devcontainer_json} shows such a \code{devcontainer.json} file. It defines the service that the developer currently wants to work on. The overall development process in a DevContainer does not differ, compared to a local \ac{VSCode} instance.

        \myparagraph{Combination of all Components}
        Docker provides the isolation for the application runtime, Docker-Compose orchestrates the individual services, enables the integration of local code and \ac{VSCode} offers a comfortable user interface. In order for this to work, it is imperative to adhere to a certain project structure, that is the same for all developers. Only if this structure is respected, the relative bind-mounts can be applied automatically, otherwise each developer would have to adjust the paths in the \code{docker-compose.yml} file. For the Symbic project, the directory structure can be found in the appendix in figure \ref{fig::dirstructure}. All projects are arranged in subfolders, the \code{docker-compose.yml} file is made available as a system-link from the management repository. To support the Linux architecture, it was necessary to ensure that all files are checked out with the correct line endings and that the appropriate Linux function is used to create a system-link. No further adjustments were necessary. The support of macOS could not be tested due to a lack of hardware. However, since macOS is Unix-based, it can be assumed that only minor or no adjustments would be necessary.\newline
        Configuration files which are project-specific, are stored in a management repository. It also contains handy scripts to perform operations on several repositories at once, set initial configuration and apply the database schema. The GitLab \ac{CI} automatically creates all needed Docker images and provides them via a private registry. Listing \ref{code::ci_build_yml} shows the developed a \code{.gitlab-ci.yml} file for \ac{CI} configuration using the \wordhighlight{kaniko} executor in order to build Docker images. The executor is given the repository context, the Dockerfile and which variant of the container should be built. The build process is executed on every push-event to the GitLab repository.\newpage
        % !TeX root = ../thesis_main.tex

\begin{lstlisting}[language=yml,caption={GitLab \ac{CI} build file for Docker Images},breaklines=true,label={code::ci_build_yml}]
map:
  key1: "foo:bar"
  key2: value2
list:
- element1
- element2
# This is a comment
listOfMaps:
- key1: value1a
  key2: value1b
- key1: value2a
  key2: value2b


\end{lstlisting}

        By adding a logging parameter to the \code{docker-compose.yml} file in listing \ref{code::compose_service_all}, all application logs can also be sent to a dedicated system to display them graphically. In the implemented project of Symbic this is achieved by a \wordhighlight{Grafana} instance and the \wordhighlight{Loki} log-shipper. Since the availability of a central, searchable log dashboard is not mandatory for the development with DevContainers, it will not be discussed further here.

        \subsubsection{Encountered Challenges and Limitations}
        During the creation of the DevContainer architecture some challenges arose. The biggest challenge consisted the differences between a Windows- and a Linux-based system. Since the source code is mapped into the container via bind-mounts and not bundled within the image, all mounted files have the \code{CR} line ending. Linux programs like Bash and \ac{SSH} can not read this format, therefore their execution fails. Creating an \code{.gitattributes} file with the following content solves the problem.
\begin{lstlisting}[language=yml,frame=none, numbers=none, backgroundcolor=\color{codebg}]
* text=auto eol=lf
*.{cmd,[cC][mM][dD]} text eol=crlf
*.{bat,[bB][aA][tT]} text eol=crlf
\end{lstlisting}
\vspace{-0.5cm}
        All files (except \code{.bat}/\code{.cmd} scripts) will be checked out with the Linux \code{LF} line ending. These are supported by most editors including editors focused on Windows.\newline
        Another challenge was the lack of isolation of local NodeJS dependencies. Deleting the \code{node\_modules} folder and reinstalling all dependencies, is not a reasonable solution in case developers switch between the local- or the DevContainer environment. The overlay file system can overshadow individual folders or files that are mounted via bind-mounts. This way, operating system-specific files are being isolated. However, the use of non-Linux focused file-systems create some problems. The NodeJS applications use nodemon to reload the application whenever file changes are made, in order to apply these changes immediately. Nodemon monitors the file system for changes, just like other hot-reloading solutions. However, due to the fundamental differences between an NTFS file-system and the Docker overlay file-system, this monitoring does not work by default. To enable this feature, an additional parameter must be specified so that nodemon actively polls for changes. With this parameter, hot-reloading works, but requires more processing power for active file-polling.\newline
        To make the usage of DevContainer as comfortable as possible, all applications are started simultaneously, with the container start likewise. However, since Docker containers only run until their initial process is terminated, a wrapper had to be created, which starts the application process, but does not terminate when the developer takes control of the application. Listing \ref{code::entry} depicts a short Bash script as the workable solution. The application process is started in the background, while the main process is an infinite loop. \newline
        Depending on the project, different technologies with different requirements and different upcoming solutions are used. This can be seen when comparing the very simple Dockerfile for the NodeJS applications with the way more complex Dockerfile used for the PHP application. For similar applications, such as the various NodeJS applications above, a Docker image can be reused and used as a template. When using other technologies, the implementation will have to be adapted accordingly. If templates for common programming languages are already available, they can be used without much effort, only the mount points in the containers have to be adapted. \newline
        Although there have been challenges in the creation of the architecture and the individual services, these have been solved with minor adjustments. However, the additional system load on Windows due to containerization could not be solved. On a Windows based host, the entire Linux kernel has to be virtualized. For a convenient file accesses between Docker and Windows, further load applies, due to necessary conversion between filesystems. This is even amplified by the active searching of file changes by nodemon. \acs{I/O} heavy tasks, such as the installation of many \code{node\_modules} cause a noticeable overhead on the system. More applications and more demanding the applications, increase the overhead and the load on the system. The impact of this drawback is further described in section \ref{sec::eval}. \newline
        During the development, bugs have also been encountered when using Docker Desktop on Windows. A memory leak caused the system memory to reach capacity when Docker images where created repeatedly. Although this can be prevented by modifying a specific setting on the \ac{WSL} system, it is not a permanent solution to the problem. Apart from that, Docker Inc. changed their licensing model for Docker Desktop during the writing of this thesis, so that, now, its only free of charge by companies with less than 250 employees and less than \$10 million in revenue \cite{dockerblog}. However, this does not apply to Docker Community Edition.

        \subsection{Final State}\label{sec::final}
        % !TeX root = ../..thesis_main.tex


\begin{figure}
    \centering
    \definecolor{host}{RGB}{6, 145, 157}
    \definecolor{docker}{RGB}{8, 199, 217}
    \definecolor{container}{RGB}{38, 230, 247}
    \tikzstyle{block} = [draw, rectangle, text centered]
    \tikzstyle{sum} = [draw, fill=white, circle, node distance=1cm]
    \tikzstyle{input} = [coordinate]
    \tikzstyle{output} = [coordinate]
    \tikzstyle{pinstyle} = [pin edge={to-,thin,black}]

    \pgfdeclarelayer{background}
    \pgfsetlayers{background,main}
    \begin{tikzpicture}[auto, align=center]



        % HOST, Docker, Tools
        \node [block, fill=host, minimum height=2cm, minimum width=\textwidth] (host) {\\\\\color{black}Host};
        \node [block, fill=docker, minimum height=1.5cm, minimum width=0.5\textwidth -0.5cm, above of=host, xshift=0.25\textwidth, yshift=1.4cm, align=right] (docker) {};
        \node [draw, rectangle, dashed, minimum height=5.1cm, minimum width=0.5\textwidth -0.5cm, above of=host, xshift=-0.25\textwidth, yshift=3.2cm, label={[above]Host Tools}] (tools) {};
        \node [above right of=host, xshift=1.8cm, yshift=1.8cm] (docker_t1) {Docker};
        \node [right of=docker_t1, xshift=1.5cm] (docker_t2) {Engine};

        % Tools
        \node[draw,dashed,minimum height=4cm,minimum width=1.8cm,rectangle, above of=host, xshift=-6.3cm, yshift=3.2cm] (tool1){Local\\Editor};
        \node[draw,dashed,minimum height=4cm,minimum width=1.8cm,rectangle, right of=tool1, xshift=1.3cm] (tool2){Testing\\Tools};
        \node[draw,dashed,minimum height=4cm,minimum width=1.8cm,rectangle, right of=tool2, xshift=1.3cm] (tool3){Terminal\\Session};

        % APP1
        \node [above of=docker, xshift=-2.5cm, yshift=0.8cm] (drive1) {\includegraphics[width=.05\textwidth]{fig/drive2.png}};
        \node [above of=drive1] (port1) {\includegraphics[width=.04\textwidth]{fig/port.png}};
        \node [above of=port1] (title1) {APP 1};
        \begin{pgfonlayer}{background}
            \node[draw,fill=container,rectangle,fit=(port1) (drive1) (title1)] (app1){};
        \end{pgfonlayer}

        % APP2
        \node [right of=drive1, xshift=1.5cm] (drive2) {\includegraphics[width=.05\textwidth]{fig/drive2.png}};
        \node [above of=drive2] (port2) {\includegraphics[width=.04\textwidth]{fig/port.png}};
        \node [above of=port2] (title2) {APP 2};
        \begin{pgfonlayer}{background}
            \node[draw,fill=container,rectangle,fit=(port2) (drive2) (title2)] (app2){};
        \end{pgfonlayer}

        % APP3
        \node [right of=drive2, xshift=1.5cm] (drive3) {\includegraphics[width=.05\textwidth]{fig/drive2.png}};
        \node [above of=drive3] (port3) {\includegraphics[width=.04\textwidth]{fig/port.png}};
        \node [above of=port3] (title3) {APP 3};
        \begin{pgfonlayer}{background}
            \node[draw,fill=container, rectangle,fit=(port3) (drive3) (title3)] (app3){};
        \end{pgfonlayer}

        % Docker Network
        \node[draw,dashed,fit=(app1) (app2) (app3), label={[above]Docker\\Network}] (network){};

        % Icons in host
        \node [below of=drive1, yshift=-3.6cm] (hdirve1) {\includegraphics[width=.08\textwidth]{fig/drive1.png}};
        \node [below of=drive2, yshift=-3.6cm] (hdirve2) {\includegraphics[width=.08\textwidth]{fig/drive1.png}};
        \node [below of=drive3, yshift=-3.6cm] (hdirve3) {\includegraphics[width=.08\textwidth]{fig/drive1.png}};
        \node [left of=hdirve1, xshift=-2.2cm] (hport1) {\includegraphics[width=.04\textwidth]{fig/port.png}};
        \node [left of=hport1, xshift=-1.4cm] (hport2) {\includegraphics[width=.04\textwidth]{fig/port.png}};
        \node [left of=hport2, xshift=-1.4cm] (hport3) {\includegraphics[width=.04\textwidth]{fig/port.png}};

        % Ports Text
        \node[above of=host, xshift=-4cm, yshift=-0.36cm] (port_text){\color{black}\footnotesize{Exposed Container Ports}};

        % Ports arrows
        \draw [draw,<->] (hport1) |- (1,0.4) |- (port1.west);
        \draw [draw,<->] (hport2) |- (3.5,0.4) |- (port2.west);
        \draw [draw,<->] (hport3) |- (6,0.4) |- (port3.west);

        % Drive arrows
        \draw [draw,<->] (hdirve1) -- node {} (drive1);
        \draw [draw,<->] (hdirve2) -- node {} (drive2);
        \draw [draw,<->] (hdirve3) -- node {} (drive3);

        % Container arrows
        \draw [draw,->] ([yshift=0.2cm]app1.east) -- ([yshift=0.2cm]app2.west);
        \draw [draw,<-] (app1.east) -- (app2.west);
        \draw [draw,->] ([yshift=0.2cm]app2.east) -- ([yshift=0.2cm]app3.west);
        \draw [draw,<-] (app2.east) -- (app3.west);

        \draw [draw, double, ->] ([xshift=-4.2cm]host.north) -- ([xshift=-.15cm]tools.south);
        \draw [draw, double, <-] ([xshift=-3.8cm]host.north) -- ([xshift=0.25cm]tools.south);

        % Pipes
        \draw[black, line width=20pt] (1.275,1.9) -- ++(0,-1.12) coordinate (L1);
        \draw[black!50,opacity=0.95, line width=19pt] (1.275,1.9) -- ++(0,-1.1) coordinate (L1);
        \draw[fill=black!60] (1.275,1.9) ellipse (0.34 and 0.13);

        \draw[black, line width=20pt] (3.77,1.9) -- ++(0,-1.12) coordinate (L1);
        \draw[black!50,opacity=0.95, line width=19pt] (3.77,1.9) -- ++(0,-1.1) coordinate (L1);
        \draw[fill=black!60] (3.77,1.9) ellipse (0.34 and 0.13);

        \draw[black, line width=20pt] (6.275,1.9) -- ++(0,-1.12) coordinate (L1);
        \draw[black!50,opacity=0.95, line width=19pt] (6.275,1.9) -- ++(0,-1.1) coordinate (L1);
        \draw[fill=black!60] (6.275,1.9) ellipse (0.34 and 0.13);

    \end{tikzpicture}
    \caption{Development Architecture with DevContainers}\label{fig::devarch}
\end{figure}
        The newly created development environment provides a uniform environment across multiple systems regarding the operating system used. Thanks to virtualization, the greatest possible similarity between the proposed environment and productive environment is achieved. Auxiliary processes no longer have to be created and configured individually as their preconfigured definitions are available in the repository. Through the use of Docker, software, which were only available on Linux beforehand, is now also available on Windows and macOS. By uniformizing the environment through virtualization, configuration-related errors become reproducible, traceable and can be avoided.\newline
        Developers are not compelled to only use the new DevContainer environment, as a hybrid use of local applications and container applications, has been achieved. The functions of container applications are available through exposed ports on host. Figure \ref{fig::devarch} shows several containers, each running one application, that can communicate with each other within their docker network. The source code for these applications is stored on the host system and is mapped into the containers by the Docker-Engine via bind-mounts. The network ports provided by the applications are forwarded by the Docker-Engine to the host system. This way, developers can use locally installed programs, in order to access the network functions of the applications. The source code can be modified by using any editor on the host, thanks to the bind-mounts. Using the \ac{VSCode} editor, developers can even connect to a container, as shown in Figure \ref{fig::vscodecontainer}, and thus use all the additional programs available within the container. This allows hybrid operation of applications using local and virtualized environments. Gradually, the DevContainer principle can be applied to one application after the other. The command \code{docker-compose up service\_1 service\_2 \ldots~service\_n } only starts the explicitly specified applications as a container, while all other applications can still be executed locally. Explicit changes to the applications in order to use DevContainers were not necessary.\newline
        However, an overhead due to the use of the DevContainer was noticed during use, especially on a Windows based system. The effects of this additional load are discussed in the next section along with the presented advantages of the DevContainer environment.


\section{Performance Evaluation and Analysis}\label{sec::eval}
In the following section, the usability of the DevContainer concept is discussed and compared with other virtualization-based solutions. For this purpose, a description of all metrics and comparison aspects is given in advance.\\
Was wird verglichen?
\begin{itemize}
    \item Anzahl der Schritte notwendig für das Einrichten einer neuen Entwicklungsumgebung
    \item Anzahl der zur Verfügung stehenden Programme (mit und ohne Docker)
    \item Gleichheit der Entwicklungsumgebung zur Produktivumgebung
    \item Verhalten im Fehlerfall
    \item Performance wie viele Container gehen unter Windows
    \item Vergleich zu Stackblitz, GitPod und co
\end{itemize}
\subsection{Metrics and how to Evaluate}
The evaluation is divided into two sections. First, the Symbic project before the usage of DevContainers is compared to the DevContainer implementation presented in section \ref{ssec::imp_approach}. Subsequently, the solution described in this paper was compared with alternative solutions already existing on the market. For this purpose, the alternatives described in section \ref{ssec::alternatives} are taken into account.\newline

\subsection{Evaluation and Results}\label{sses::eval_compare}
% Adding a local environment increases the overall amount of configuration effort, requiring a reliable, efficient way to manage it.
\subsection{Discussion of Evaluation}

\section{Future Potential and Outlook}\label{sec::outlook}
\section{Conclusion}\label{sec::conclusion}


\newpage
% STATS
% 11800 Words
% 66.400/78.100 Characters

% Anhang
\singlespacing{}
\lhead{Appendix}
\renewcommand{\thesubsection}{\Alph{subsection}}
\pagenumbering{Roman}
\setcounter{page}{\value{lastroman}}
\vspace*{\fill}
    \makebox[0.92\textwidth]{\Huge Appendix}
\vspace*{\fill}
\addcontentsline{toc}{section}{Appendix}

% Abbreviations
% !TeX root = ../main.tex
\newcommand{\abbr}{Abbreviations}
\subsection{Abbreviations}
%\addcontentsline{toc}{subsection}{Abbreviations}

\begin{acronym}[1234567890ABC]		%[längste Abkürzung]
\setlength{\itemsep}{-\parsep}	% sorgt dafür, dass das Verzeichnis kompakt dargestellt wird.

\acro{CI}[CI]{Continuous Integration}
\acro{CD}[CD]{Continuous Delivery}
\acro{IDE}[IDE]{Integrated Development Environment }
\acro{nvm}[nvm]{NodeJS Version Manager}
\acro{python-venv}[python-venv]{python virtual environments}

\end{acronym}
% \newpage

%Code
\subsection{Code Listings}
% !TeX root = ../thesis_main.tex

\begin{lstlisting}[language=docker, frame=single, caption={PHP DevContainer Dockerfile},label=code::docker_dev_php]
# PHP version: 8-apache-bullseye, 7-apache-bullseye, and more
ARG VARIANT=7.3-apache-bullseye
FROM php:${VARIANT}

# Copy scripts and install tools
COPY install-scripts/*.sh /tmp/install-scripts/
RUN apt-get update && bash /tmp/install-scripts/install.sh \
    && apt-get -y install lynx libmemcached-dev zip \
    zlib1g-dev libicu-dev libxml2-dev libssl-dev \
    htop less curl unzip git iputils-ping unzip cron \
    zlib1g-dev libpng-dev libjpeg-dev zlib1g-dev libzip-dev \
    && docker-php-ext-install bcmath \
    && docker-php-ext-install sockets \
    && docker-php-ext-install pdo pdo_mysql\
    && docker-php-ext-install mysqli \
    && docker-php-ext-install mbstring \
    && docker-php-ext-install simplexml \
    && docker-php-ext-install gd \
    && docker-php-ext-install zip \
    && a2enmod rewrite \
    && usermod -aG www-data ${USERNAME} \
    && sed -i -e "s/Listen 80/Listen 80\\nListen 8080/g" \
    /etc/apache2/ports.conf

# Install xdebug
RUN yes | pecl install xdebug \
    && echo "zend_extension=$(find /usr/local/lib/php/extensions/ -name xdebug.so)" > /usr/local/etc/php/conf.d/xdebug.ini \
    && echo "xdebug.mode = debug" >> /usr/local/etc/php/conf.d/xdebug.ini \
    && echo "xdebug.start_with_request = yes" >> /usr/local/etc/php/conf.d/xdebug.ini \
    && echo "xdebug.client_port = 9000" >> /usr/local/etc/php/conf.d/xdebug.ini \
    && rm -rf /tmp/pear

# Install composer
RUN curl -sSL https://getcomposer.org/installer | php \
    && chmod +x composer.phar \
    && mv composer.phar /usr/local/bin/composer
WORKDIR /var/www/html
COPY . /var/www/html
RUN composer install --no-interaction --no-plugins \
    --no-scripts --prefer-dist && a2enmod rewrite

# Fix permission and set config files
COPY contrib/docker-apache.conf /etc/apache2/sites-enabled/
RUN ln -s /usr/local/bin/php /usr/sbin/php \
    && chown -R www-data:www-data /var/www/html/var/log \
    && chown -R www-data:www-data /var/www/html/var/cache \
    && chown -R www-data:www-data /var/www/html/var/openvpn \
    && mv docker-entryscript.sh /docker-entryscript.sh \
    && chmod +x /docker-entryscript.sh \
    && mv application/configs/application.docker.php \
    application/configs/application.php
\end{lstlisting}




% !TeX root = ../thesis_main.tex

\begin{lstlisting}[language=docker-compose-2,caption={All Symbic Services in one \code{docker-compose.yml}},breaklines=true,label={code::compose_service_all}]
services:
  webapp:
    image: webapp/dev-container
    entrypoint: "/workspace/.devcontainer/entrypoint.sh"
    environment:
      - PORT=3000
      - BASE_URL=http://localhost:3000
      - AUTH_BACKEND_URL=http://localhost:80
      - SYSTEM_BACKEND_URL=http://localhost:8090
    volumes:
      - ./portal-webapp:/workspace
      - portal_node_modules:/workspace/node_modules
      - vscode_extentions:/root/.vscode-server/extensions
    ports:
      - 3000:3000

  system-api:
    image: system-api/dev-container
    entrypoint: "/workspace/.devcontainer/entrypoint.sh"
    environment:
      - PORT=8090
      - MONGO_DB_HOST=mongo
      - MONGO_DB_USER=root
      - MONGO_DB_PW=none
      - MONGO_DB_NAME=local-log-db
      - SQL_HOST=db
      - SQL_USER=root
      - SQL_PASSWORD=yes
      - SQL_DB_NAME=local-device-db
    volumes:
      - ./system-api:/workspace
      - system_api_node_modules:/workspace/node_modules
      - vscode_extentions:/root/.vscode-server/extensions
    ports:
      - 8090:8090

  mqtt-endpoint:
  image: mqtt-endpoint/dev-container
  entrypoint: "/workspace/.devcontainer/entrypoint.sh"
  environment:
    - MQTT_CLIENT_ID=mqtt-connector
    - MQTT_HOST=mosquitto
    - MQTT_USERNAME=mqtt_user
    - MQTT_PASSWORD=mqtt_pw
    - HTTP_PORT=8080
  volumes:
    - ./mqtt-endpoint:/workspace
    - mqtt_endpoint_api_node_modules:/workspace/node_modules
    - vscode_extentions:/root/.vscode-server/extensions
  ports:
    - 8080:8080

  auth-backend:
    image: auth-backend/dev-container
    volumes:
      - ./auth-backend/src:/var/www/html/src
    ports:
      - 80:80

####### Auxiliary services #######
  db:
    image: mysql
    environment:
      - MYSQL_ROOT_PASSWORD=yes
    ports:
      - 3306:3306
    volumes:
      - sql_data:/var/lib/mysql
  mongo:
    image: mongo
    environment:
      - MONGO_INITDB_ROOT_USERNAME=root
      - MONGO_INITDB_ROOT_PASSWORD=none
    ports:
      - 27017:27017
    volumes:
      - mongo_data:/data/db
  mosquitto:
    image: eclipse-mosquitto
    ports:
      - 1883:1883
    volumes:
        - mosquitto_data:/mosquitto/data
        - ./manage-reop/config:/mosquitto/config

####### Database managment services #######
  myadmin:
    image: phpmyadmin
    environment:
      - PMA_HOST=db
      - PMA_USER=root
      - PMA_PASSWORD=yes
    ports:
      - 8888:80
volumes:
  vscode_extentions:
  webapp_node_modules:
  system_api_node_modules:
  sql_data:
  mongo_data:
  mosquitto_data:
\end{lstlisting}

\newpage

\subsection{Additional Figures}
% !TeX root = ../thesis_main.tex
\begin{figure}[!h]
  \begin{forest}
      for tree={
        font=\ttfamily,
        grow'=0,
        child anchor=west,
        parent anchor=south,
        anchor=west,
        calign=first,
        edge path={
          \noexpand\path [draw, \forestoption{edge}]
          (!u.south west) + (7.5pt,0) |- node[fill, inner sep=1pt] {} (.child anchor)\forestoption{edge label};
        },
        before typesetting nodes={
          if n=1
            {insert before={[,phantom ]}}
            {}
        },
        s sep=1.5pt,
        fit=band,
        before computing xy={l=15pt},
      }
    [ProjectRoot
      [manage-reop
        [configs]
        [scrips]
        [docker-compose.yml]
      ],
      [auth-repo
        [src]
        [vendor]
        [Dockerfile]
        [composer.json]
      ],
      [system-repo
        [src]
        [node\_modules]
        [Dockerfile]
        [package.json]
      ],
      [mqtt-con-repo
        [src]
        [node\_modules]
        [Dockerfile]\
        [package.json]
      ],
      [webapp-repo
        [src]
        [node\_modules]
        [Dockerfile]
        [package.json]
      ],
      [docker-compose.yml \textit{(symlink)}]
    ]
  \end{forest}

  \caption{Directory Structure of the Project}
  \label{fig::dirstructure}
\end{figure}
\newpage

% List of XXX
\listoffigures
\listoftables
\lstlistoflistings{}
\newpage

%Bibliographie
\addcontentsline{toc}{section}{References}
% \bibliographystyle{alpha}
\bibliographystyle{IEEEtranSA}
\bibliography{bib/sources}

% Authorship
\include{inc/ensure.inc}
% or
% % !TeX root = ../thesis_main.tex
\begin{otherlanguage}{ngerman}
    \section*{Erklärung zur selbständigen Abfassung der Bachelorarbeit}

    \vspace{5cm}

    ~\\
    Ich versichere, dass ich die eingereichte Bachelorarbeit selbständig und ohne unerlaubte
    Hilfe verfasst habe. Anderer als der von mir angegebenen Hilfsmittel und Schriften habe
    ich mich nicht bedient. Alle wörtlich oder sinngemäß den Schriften anderer Autoren entnommenen
    Stellen habe ich kenntlich gemacht.

    \vspace{3cm}
    \begin{flushright}

        \rule{8cm}{0.2mm} \\
        Unterschrift (\myName)
    \end{flushright}

    \vspace{2cm}
    \place, der \submission{}
\end{otherlanguage}

\end{document}
% END DOC/EOF

% !TeX root = ../thesis_main.tex
\newcommand{\abbr}{Abbreviations}
\subsection{Acronyms}
%\addcontentsline{toc}{subsection}{Abbreviations}

\begin{acronym}[1234567890]		%[längste Abkürzung]
\setlength{\itemsep}{-\parsep}	% sorgt dafür, dass das Verzeichnis kompakt dargestellt wird.

\acro{API}[API]{Application Programming Interface}
\acro{AWS}[AWS]{Amazon Web Services}
\acro{CI}[CI]{Continuous Integration}
\acro{CD}[CD]{Continuous Deployment}
\acro{DB}[DB]{\LU{D}{d}atabase}
\acro{DNS}[DNS]{Domain Name System}
\acro{GUI}[GUI]{Graphical User Interface}
\acro{HTTP}[HTTP]{Hypertext Transfer Protocol}
\acro{I/O}[I/O]{\LU{I}{i}nput/\LU{O}{o}utput}
\acro{IaC}[IaC]{Infrastructure-as-Code}
\acro{IDC}[IDC]{International Data Corporation}
\acro{IDE}[IDE]{\LU{I}{i}ntegrated \LU{D}{d}evelopment \LU{E}{e}nvironment\LU{}{s}}
\acro{IETF}[IETF]{Internet Engineering Task Force}
\acro{IoT}[IoT]{Internet of Things}
\acro{IP}[IP]{Internet Protocol}
\acro{IPC}[IPC]{\LU{I}{i}nterprocess \LU{C}{c}ommunication}
\acro{ISP}[ISP]{Internet Service Provider}
\acro{MSA}[MSA]{\LU{M}{m}icroservice \LU{A}{a}rchitecture}
\acro{NVM}[NVM]{Node Version Manager}
\acro{OS}[OS]{\LU{O}{o}perating \LU{S}{s}ystem}
\acro{QA}[QA]{\LU{Q}{q}uality \LU{A}{a}ssurance}
\acro{SDLC}[SDLC]{Software Development Life Cycle}
\acro{REST}[REST]{Representational State Transfer}
\acro{RSA}[RSA]{Rivest-Shamir-Adleman}
\acro{SSH}[SSH]{Secure Shell Protocol}
\acro{TDD}[TDD]{Test-Driven Development}
\acro{UI}[UI]{\LU{U}{u}ser \LU{I}{i}nterface}
\acro{UX}[UX]{\LU{U}{u}ser \LU{E}{e}xperience}
\acro{VCS}[VCS]{Version Control System}
\acro{VM}[VM]{\LU{V}{v}irtual \LU{M}{m}achines}
\acro{VPN}[VPN]{\LU{V}{v}irtual \LU{P}{p}rivate \LU{N}{n}etwork\LU{}{s}}
\acro{VSCode}[VSCode]{Visual Studio Code}
\acro{WSL}[WSL]{Windows Subsystem for Linux}
\acro{XP}[XP]{Extreme Programming}
\end{acronym}

% !TeX root = ../thesis_main.tex
\newcommand{\abbr}{Abbreviations}
\subsection{Abbreviations}
%\addcontentsline{toc}{subsection}{Abbreviations}

\begin{acronym}[1234567890]		%[längste Abkürzung]
\setlength{\itemsep}{-\parsep}	% sorgt dafür, dass das Verzeichnis kompakt dargestellt wird.

\acro{AWS}[AWS]{Amazon Web Services}
\acro{CI}[CI]{Continuous Integration}
\acro{CD}[CD]{Continuous Delivery}
\acro{HTTP}[HTTP]{Hypertext Transfer Protocol}
\acro{IDC}[IDC]{International Data Corporation}
\acro{IETF}[IETF]{Internet Engineering Task Force}
\acro{I/O}[I/O]{Input/Output}
\acro{IDE}[IDE]{Integrated Development Environment }
\acro{IP}[IP]{Internet Protocol}
\acro{ISP}[ISP]{Internet Service Provider}
\acro{nvm}[nvm]{NodeJS Version Manager}
\acro{OS}[OS]{Operating System}
\acro{REST}[REST]{Representational state transfer}
\acro{SSH}[SSH]{Secure Shell Protocol}
\acro{VM}[VM]{Virtual Machines}
\acro{XP}[XP]{Extreme Programming}
\end{acronym}
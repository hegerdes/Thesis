% !TeX root = ../thesis_main.tex

\section{Introduction}\label{sec::intro}
Digital service solutions are becoming more and more relevant as a result of the ever-increasing possibilities and availability of technology. New remote working tools, digital education resources, a connected healthcare system and seamless (video) communication systems are needed. The overall revenue in the software market is expected to grow from \$532 billion (2020) to \$772 billion in 2025 \cite{software_industry_groth}. The need for digital business solutions is bigger than ever before. Just as the availability of on-demand computing capacity, which is greater than ever before thanks to the ever-growing cloud platforms, such as \ac{AWS}, Microsoft Azure and Google Cloud. Services can be made available quickly across the world without the need to build up an infrastructure on site. To provide fast and suitable software solutions, developers need adequate development setups, called development environments. This thesis analyses modern, agile software development environments, points out potential problems and purposes a virtualized software development solution to improve development efficiency. This solution is implemented on a real project and then evaluated.\newline
The following section will briefly describe the fundamental question of this thesis, point out its relevance and presents a possible solution to the problems found. Subsequently, a clear delineation is given as to what is and what is not covered in this thesis.

    \subsection{Problem Description}
    The rise of agile development and microservices has been accelerated by the emergence of new technologies changing the way \textit{where} and \textit{how} software is running. Applications could be distributed and scaled quickly through cloud computing, to the liking of customers, favoring for quick changes. Yet, the actual coding setup has not changed significantly.\newline
    Technical requirements for projects and their development environments differ depending on their purpose. The development process of \ac{GUI} applications for PCs and smartphones apps is quite different to the development of web-services. As the functionality of web services continues to grow, their adoption is a cost-effective, economical way to deliver cross-platform products, and it is becoming increasingly popular. However, the operating system used for the development process is often different from the operating system used to run these applications in production. This can cause platform-dependent errors. Managing the runtime versions of multiple programming languages between team members or different projects becomes a challenge, as these can lead to unexpected program behavior or result in library version conflicts. The usage of a \acl{MSA} poses further problems, due to its scalability, adaptability and rapid development. Microservices require additional configuration effort and increase the difficulty of whole system tests. The initial setup for new developers can get quite complex, requires time and might even discourage developers to contribute to open source projects.\newline
    These characteristics add additional effort, can introduce new errors and slow down the development speed, resulting in higher costs and a lower customer experience.

    \subsection{Goal of the Thesis}
    The aforementioned problems were also encountered by Symbic GmbH, in whose cooperation this thesis is being produced. In further detail, the aforementioned challenges and obstacles of modern software development environments are presented in detail and their cause is identified. Based on these findings a solution concept for these challenges is created, by using virtualization technologies. These allow to abstract the operating system used and provide greater similarity between development and productive environments. The technologies used are explained, and the developed concept is then applied to a real project at Symbic. Subsequently, the practicability of the solution is evaluated and classified. In principle, the goal of this thesis is to identify obstacles in current software development setups and to analyze the effectiveness of the proposed solution.

    \subsection{Scope of the Thesis}
    Different software development environments cover a wide range of tools, which differs significantly depending on the project. Since the development of web-based solutions is becoming more and more popular, priority is given to these types of projects only. Native application development for PCs and smartphones, as well as the development of embedded systems and other hardware-related solutions, are not covered by this thesis.\newline
    In particular, the concept shown is not a general solution. It is intended only as a solution template that contains many ways for dealing with the problems described in section \ref{sec::problem}. Although section \ref{sec::backgrund} offers an explanation of fundamental topics, deeper insight into the technologies used is given in section \ref{ssec::toolsused}. Nevertheless, not all basic concepts can be explained in depth. Basic knowledge of the software development process, essential programs and an abstract understanding of different operating systems and network techniques are recommended.

    \subsection{Structural Overview}
    The following section provides overall background information, which are helpful for further understanding. Common and modern development methodologies are described in section \ref{ssec::devops}, followed by microservice and container basics in section \ref{ssec::microservices}. Section \ref{sec::problem} will be a detailed analysis of current development setups and the problems encountered in these setups. The problems are illustrated with examples and their consequences for the development process. Section \ref{sec::solution_concept} proposes a conceptual solution, defines its use cases and workings while also providing limitations for such a solution. This concept is then applied to a practical reference project in section \ref{sec::solution_code}. The implementation process is described and additional challenges and handy tools for the practical use are presented. Section \ref{sec::eval} discusses the proposed solution and shows its strengths and weaknesses. As to conclude section \ref{sec::conclusion} summarizes the key findings of this thesis and gives an outlook on future application areas of the concept presented.

% !TeX root = ../thesis_main.tex

\section{Introduction}\label{sec::intro}
% Symbic erwähnen
Digital service solutions are becoming more and more relent and popular as a result of the ever-increasing possibilities and availability of technology. Due to the COVID-19 pandemic new remote working tools, digital education resources, a connected healthcare system and seamless (video) communication systems are needed. The e-commerce marked grew by 32\% between 2019 and 2020 up to sales volume of \$791.70 billion \cite{online_shopping_inc} and the overall revenue in the software market is expected to grow from 532 billion (2020) to 772 billion in 2025 \cite{software_industry_groth}. The need for digital business solution is bigger than ever before. The availability of on-demand computing capacity is greater than ever before thanks to the ever-growing cloud platforms such as \ac{AWS}, Microsoft Azure and Google Cloud. Services can be made available quickly across the world without having to build an own infrastructure. To provide fast and suitable software solutions developers need adequate development setups, called development environments. This thesis analyses modern agile software development environments, points out potential problems and purposes a virtualized software development solution to improve development efficiency. This solution is implemented on a real project and then evaluated.\newline
The following section will briefly describe the fundamental core question of this thesis, point out its relevance and presents a possible solution approach. Subsequently, a clear delineation is then given as to what is and is not covered in this thesis. The last Section of the introduction gives a structural overview of how the thesis is structured and how it will approach the given problem.

    \subsection{Problem Description}
    The rise of agile development and microservices was accelerated by the emergence of new technologies that changed the way \textit{where} and \textit{how} software is running. Applications could be distributed and scaled quickly through cloud computing which benefited customers favors for quick changes. Yet the actual coding setup has not changed significantly.\newline
    Requirements for development environments differ depending on the project type. The development process of (\ac{GUI}) applications for PC's and smartphones is quite different to web-services development. As the functionality of web services continues to grow, while being a cost-effective way to deliver across platforms products, this method is becoming increasingly popular. However, the operating system used for the development process often differs to the operating system used to run these applications in production. This can cause platform dependent errors. Managing programming language runtime versions between team members or different projects become a challenge, as these can lead to unexpected program behavior or result in library version conflicts. The usage of a \acl{MSA}, due to its easy scalability, adaptability and rapid development, brings further problems. Microservices require additional configuration effort and increase the difficulty level for whole system tests, called end-to-end tests. The initial setup for new developers can get quiet complex, requires time and might even discourage developers in open source projects.\newline
    These characteristics add additional effort, can introduce new errors and slow down the development speed, resulting in higher cost and lower customer experience.

    \subsection{Goal of the Theses}
    In the process of this work, the challenges and obstacles of modern software development environments are identified and presented. Based on these findings, a solution concept, for these challenges, is created based on virtualization technologies. The technologies used are explained, and the solution concept is applied to a real project in cooperation with Symbic GmbH. Subsequently, the practicability of the solution is evaluated and classified. In principle, the goal of this thesis is to identify obstacles in current software development setups and to analyze the effectiveness of the proposed solution.
    % TODO
    % Im „Goal“-Abschnitt sollte zumindest grob auf die vorige Problem Beschreibung eingegangen werden. Denn: Warum sollten Probleme benannt werden, wenn nicht erkennbar ist, wie sie gelöst werden könnten? Strategisch ist es hier sinnvoll, für mind. eines der beschriebenen Probleme eine konkrete Lösung zu skizzieren; also einen Ausblick auf einen Teil deines Konzepts zu geben. In ein oder zwei knappen Sätzen, sodass eine minimale Idee vermittelt wird, wohin die Reise geht.

    \subsection{Scope of the Thesis}
    The range of different software development environments is large and differs significantly from each other depending on the project. Since the development of web-based solutions is becoming more and more popular, priority is given only to these types of projects. Native application development for PCs and smartphones, as well as the development of embedded systems and other hardware-related solutions are not covered in this thesis.\newline
    In particular, the solution concept shown is not a general solution that is a perfect fit for all projects and therefore is not generally transferable. It is intended only as a solution template that contains many tricks for dealing with the problems described in Section \ref{sec::problem}. Although Section \ref{sec::backgrund} offers an explanation of fundamental topics and Section \ref{ssec::toolsused} a deeper insight into the functionality of technologies used, nevertheless not all addressed contents can be explained fundamentally. Basic knowledge of the software development process, essential programs and abstract understanding of different operating systems and network techniques are recommended.

    \subsection{Structural Overview}
    The following section provides overall background information, which are recommended for further understanding. Common and modern development methodologies are described in Section \ref{ssec::devops}, followed by microservice and container basics in Section \ref{ssec::microservices} and \acl{CI} tools in \ref{ssec::devops}. In Section \ref{sec::problem} will be a detailed analysis of current development setups and the problems encountered in these setups. The problems are illustrated with examples and their consequences for the development process, the overall project and its quality, which is emblematic of user satisfaction, are presented. Section \ref{sec::solution_concept} proposes a conceptual solution, defines its usability and workings while also providing conditions and limitations for such a solution. This solution concept is applied in Section \ref{sec::solution_code} by implementing it in a practical reference project. The implementation process is described and additional challenges and tricks for practical use are presented. Section \ref{sec::eval} discusses the proposed solution and shows its strengths and weaknesses. At the end, an outlook on future application areas and alternative solutions is given in Section \ref{sec::outlook}, followed by the conclusion in Section \ref{sec::conclusion}.

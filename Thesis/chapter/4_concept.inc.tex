% !TeX root = ../thesis_main.tex


\section{The Concept of DevContainers}\label{sec::solution_concept}
This section proposes the usage of Development Containers (DevContainers) as a solution to the problems of local development environments as described in section \ref{sec::problem}. The concept is oriented towards the \ac{PaaS} principles predominant in the cloud. Developers should only have to deal with the application. The runtime environment and network configuration are managed by the DevContainer. They combine the application runtime and its configuration into an isolated environment by using the lightweight virtualization approach of containers. One difference between DevContainers and the \ac{PaaS} principle is that DevContainers provide neither own hardware, nor its management.\newline
The details of such a DevContainer concept are described below. This is followed by a delimitation of the cases, in which its application scenarios are sense and when they are not. Finally, the advantages and limitations of this solution are outlined.

    \subsection{Description of a Conceptual Environment}
    The idea of DevContainers is to bundle the application code, its runtime and configuration into an isolated system. Only the minimum necessary scope for accessing the application is made available on the host system, all other resources remain isolated. According to the design of \ac{VM}s and containers, they can be started and stopped at will without encountering the risk of permanent changes on the host system by the containerized applications. When the container is started, all port, path and secret configurations are already set. Developers are not required to do any further manual configuration and additional steps. Different branches and versions of the applications all have their own container, which are completely independent from the local settings. Auxiliary applications and interdependent services can all be started simultaneously and in the correct order. The DevContainer environment builds upon the Linux kernel, is completely independent of the host \ac{OS} and is thus much closer to the production environment, potentially decreasing system-specific errors. Even in case of an error and the environment ends up in an undefined state, the principles of the server world can be applied. The DevContainer can simply be discarded and recreated from a well-defined template within seconds. This design enables environments that are 100\% reproducible.\newline
    Such a setup allows developers to choose any host \acl{OS} because the entire application code is executed in a virtualized environment. It increases the initial setup time and prevents configuration drift since all configuration settings follow the \ac{IaC} principle and are stored as code. This way, new runtimes or dependencies can be tested without the risk to corrupt the local development environment. Through automatic orchestration, integration tests become easier and can be performed earlier, decreasing the time until an error is detected. Dependencies and common software like debugger are already present in the container, so they do not need to be installed separately. DevContainers promise to solve the problems described in section \ref{sec::problem} and allow developers to focus on programming rather than configuring and maintaining their working environment.

    \subsection{Pre-requirements for DevContainers}
    Before using DevContainers, one has to verify them being the appropriate approach to the problems encountered and them fulfilling all necessary.\newline
    DevContainers are based on virtualization technology, and one of their goals is to achieve the greatest possible similarity between the development and the production environment. In order to take full advantage of this possibility, the production environment must already be designed for the use of virtualization with containers. As can be seen in section \ref{ssec::toolsused}, the majority of virtualization solutions are based on Linux, which is also the most widely used system in the server domain. Applications that require Windows can also use Windows-based containers, but these may require additional licenses and configuration, accordingly they will not be discussed any further in this paper. The application to be developed must be suitable for the use in a container. Containers do not offer direct support of graphical output by default. Functions are exposed to the outside world via the usage of sockets or mounted devices. Therefore, experience in the area of virtualization and a functioning \ac{CI}/\ac{CD} pipeline for the rapid delivery of new application versions is therefore also recommended.\newline
    Although, DevContainers reduce the configuration effort for each developer, the architecture and settings files for the use of DevContainers must be created once and then be maintained. As section \ref{sssec::virtual} has already shown,  any kind of virtualization, regardless of whether it is VM-based or container-based, there is a performance overhead. If, as in a typical microservice architecture, several applications are run at the same time, this overhead adds up and places an additional load on the developer's system. Modern hardware with sufficient memory and computing capacity is therefore necessary for the use of DevContainers. The amount of additional load depends on the technology stack used.

    \subsection{Creating a DevContainer setup}\label{ssec::toolsused}
    section \ref{ssec::getting_devops} has already presented concrete programs for providing the virtualization concept and automating certain processes with \ac{CI}. These form the precondition  for the provision of DevContainers. This concept must now be made available to developers in actual concepts and, in doing so, solve further problems described in section \ref{sec::problem}. Developers need to be able to initialize a DevContainer based environment quickly and must be offered a way to interact with the DevContainer, especially with the application within, without a significant impacting on their workflow. As to illustrate, tools and concepts that make this possible are presented below.

        \subsubsection{Defining the Application Runtime}
        The entire application runtime environment is virtualized by using Docker. However, the basics of this runtime environment must be defined beforehand. In the case of Docker, this is done with Dockerfiles. These provide the instructions for building Docker-images, which are needed for a container in order to start. \ac{CI} platforms typically execute the building process of images automatically \cite{docker2020}.\newline
        % !TeX root = ../thesis_main.tex

\begin{lstlisting}[language=docker, frame=single, caption={Python DevContainer Dockerfile},label=code::docker_dev_example]
FROM debian:buster
RUN apt-get update && apt-get install -y \
    git python3 python3-pip make vim emacs && \
    pip3 install pandas numpy matpltlib

#Optional
Copy . /app
\end{lstlisting}

        Listing \ref{code::docker_dev_example} shows such a Dockerfile for a Python application. The Linux distribution Debian is chosen as the starting point for the image, followed by the installation of all required dependencies and programs. There are two possibilities for implementing a DevContainer concept, when it comes to the source code. Optionally, the program code can be copied into the image so that the image already contains all the program components. The resulting fully self-contained image is the typical method in production operation. The disadvantage of this approach is that the program code can only be accessed within the container and is not present on the host system. While this is desirable in production, due to security and strict isolation, it results in additional development overhead. Developers can only use applications within the container to edit the code, and the \ac{CI} system must create new images for every change. Prematurely discarded containers can lead to the loss of changes that have not yet been published to the remote source code repository \cite{dockerdocs}.\newline
        Alternatively, copying the program code can be omitted and only the well-defined runtime environment is provided in the image. The program code on the host system is then mapped into the container via a bind-mount when the container is started. This way, the program code can be edited and used by any editor on the host. In the case of a discarded container, the changes made are still available on the host. Furthermore, a new image only needs to be created when a change in the Dockerfile or dependencies occurs, which saves significant computational effort in the \ac{CI} system. After the build process all images are made available on a private or public image registry in order to be accessible. In the working directory of the application, the following commands are executed by the \ac{CI} system in order to create, name and upload the image to a registry:
        \begin{lstlisting}[language=bash, frame=none, numbers=none, backgroundcolor=\color{codebg}]
docker build -t my-python-app .
docker push my-python-app
        \end{lstlisting}
        \vspace{-1cm}

        \subsubsection{Orchestrating the Application Containers}
        In order for the Python application above to work, it needs a database. Instead of each developer having to install and set up a database on their own, which may then shared between different projects, this can be done in isolation and automatically for each project by using virtualization and composition. Applications in an \ac{MSA} have even more services they depend on. Arranging all these programs to create one greater service is called orchestration. Well-known applications for container-based orchestration are \wordhighlight{Docker-Compose} and \wordhighlight{Kubernetes}. Since Kubernetes does not ship with Docker by default, is quite complex and production oriented, Docker-Compose is suggested for local orchestration.\newline
        Docker-Compose requires a configuration file, declared in the \acs{YAML} format, which contains all the necessary information for orchestrating multiple applications. Each application is defined as a service with a unique name when using Docker-Compose. The \code{docker-compose.yml} file in Listing \ref{code::compose_example} defines the services \code{app} and \code{db}. Each service has an image, which is the initial state for each container. With Docker containers, it is common to use environment variables to influence and configure the application within the container. The initial credentials for the database service are set via environment variables and the python app uses them in order to connect to the \acl{DB}. It should be noted that the database host can be just the service name of the database server instead of an IP address. Within the Docker managed network, the integrated DNS server automatically resolves the service names to the IP address of the \ac{DB} server. This virtual network allows the containers to communicate with each other, but in order for the functionality to be available on the host, the network ports used must be explicitly exposed to the host. The Python application uses the alternative \acs{HTTP} port \code{8080} and the database server uses the standard MySQL port. When starting the containers, the corresponding ports are allocated on the host system and local programs can access the defined services using \code{localhost} and the corresponding network port \cite{dockerdocs}.\newline
        If the source code is not copied into the image, it must be mapped into the container using bind mount point. Line 11 in Listing \ref{code::compose_example} binds the local \code{app-src} directory from the host to the container at the location \code{/workspace}. The database service, on the other hand, uses a volume mount, in which docker manages the allocated storage persistently itself. The configuration is completed with an entry-point for the python app that will be executed when the container is started.\newline
        % !TeX root = ../thesis_main.tex

\begin{lstlisting}[language=docker-compose-2,caption={Exemplary Python Project \code{docker-compose.yml}},breaklines=true,label={code::compose_example}]
services:
  app:
    image: my-python-app
    entrypoint: python3 app.py
      - DB_HOST=db
      - DB_PW=yes
      - DB_USER=root
    ports:
      - 8080:8080
    volumes:
      ./app-src:/workspace

  db:
    image: mysql
    environment:
      - MYSQL_ROOT_PASSWORD=yes
    ports:
      - 3306:3306
    volumes:
      - sql_data:/var/lib/mysql
volumes:
  sql_data:
\end{lstlisting}

        Docker-Compose ensures that all interdependent services are started in the correct order, that all persistent volumes are created, and that all containers can communicate within their network \cite{docker2020}, \cite{dockerdocs}. Since all the necessary configurations are stored in an ordinary text file, the \ac{IaC} principle applies here. The state of the \code{docker-compose.yml} file and all changes are tracked in \ac{VCS}. Accordingly, even development work requiring changes to the runtime can take place in a separate branch without affecting the work of other developers.\newline
        The command \code{docker-compose up} is used in order to start all services specified in the \code{docker-compose.yml}. If the docker images used are not already available on the host, they are automatically downloaded from available image-registries. When all services are started, Docker-Compose attaches itself to all running containers and writes the color-coded output of the every program to the console.\newline
        Through this concept, developers have isolated program environments that can be created quickly, easily and are producible. Multiple projects, which run simultaneously, no longer have to share a database server and different language frameworks, interpreters and compilers can be tested independently. In case a fast cache-server is needed, another service can be created quickly in order to provide a \wordhighlight{Redis} or \wordhighlight{Memcached} server without the need to installing it to the host. Individual, manual and diverging configurations for each developer are eliminated. New environments that are similar to the production environment can be created quickly, independent of the hosts operating system. Even in a microservice architecture, developers have the opportunity to test the interaction of multiple applications before committing changes to the \ac{VCS}. This way, potential errors are already detected before \ac{CI} integration tests, which means that they can be corrected more quickly. These characteristics are promising in terms of solving the problems described in section \ref{sec::problem} regarding heterogeneity, lack of testing facilities, and the tedious configuration of the setup. In order for this concept to be adopted by developers, there must be an equally effective way for developers to interact with their applications within the DevContainer.

        \subsubsection{Interacting with DevContainers}
        The primary way developers interact with their code and the application is through their editor. Their variability and the number of different editors is great. \wordhighlight{Visual Studio, XCode, Atom, Sublime, Eclipse, Emacs} and \wordhighlight{VIM} are well-known general purpose editors. There is usually a distinction between simple text editors and \acl{IDE}. While text editors are quite simple and only provide basic functionalities like syntax highlighting, \ac{IDE}s are much more comprehensive with powerful IntelliSense suggestions, built-in project management, \ac{VCS}, debugger, graphical visualization and build tools. The choice of the editor is a personal decision for most developers, and they are customized according to their preferences. For this reason, the proposed solution does not require a specific editor for DevContainers, but gives a recommendation which will be used as a reference for the rest of the work.\newline
        \ac{VSCode} is a free and platform independent editor with extensive extendibility, which is used by over 70\% of all developers accordingly to StackOverflow~\cite{stackoverflow2021}. One of its beneficial features for virtualized workloads is its remote development functionally. This officially provided, extension allows developers to use the intuitive comfort of a graphical \ac{UI} while running the code, the application and auxiliary processes like a debugger on another, remote machine. Figure \ref{fig::vscodecontainer} shows how the \ac{VSCode} frontend connects to a remote machine or container and installs a server instance of the editor. The server side instance manages the access to the remote file system and the execution of processes while communicating with the local \ac{VSCode} frontend instance for comfortable access to these functions. This type of remote development works for \ac{SSH} connections, the \ac{WSL}, and on Docker containers. \ac{WSL} is a built-in Windows feature to provide a Linux environment on Windows hosts without the need of a separated \ac{VM}. The implementation of Docker for Windows is build upon this feature \cite{vscodedevcontainer}. In the further course, only the remote functions for containers will be considered here.
        \begin{figure}
            \centering
            \includegraphics[width=.95\textwidth]{architecture-containers.png}
            \caption{Architecture of \ac{VSCode} Development Container Setup \\\textit{Source:~\cite{vscodedevcontainer}}}\label{fig::vscodecontainer}
        \end{figure}
        In order for container services to be accessed, the appropriate ports are exposed via Docker-Compose, even though \ac{VSCode} provides a similar feature through the remote extension. Remote processes are automatically made available locally through port forwarding. \ac{VSCode}s implementation is not expose ports for multiple containers simultaneously, making the feature only useful for a service that are currently worked on, or for services with dynamically changing network ports. However, this feature can be used to make a short-lived process available on the host quickly without having to modify the configuration in the \code{docker-compose.yml} file. This allows the usage of local testing or exploration tools on any remote services.\newline
        In order for \ac{VSCode} to connect to or start a DevContainer, a configuration file is required. This is provided by a \code{devcontainer.json} file. For each service developed, the path to the \code{docker-compose.yml} file is specified, as well as the service \ac{VSCode} should connect to. Listing \ref{code::devcontainer_json} shows such a configuration file. It also specifies, which directory the \ac{VSCode} server should open, which extensions have be installed and what happens when the container starts or stops. If this file exists, \ac{VSCode} automatically offers to reload the local project using the DevContainer. All services are started automatically, directories are mounted, and the network ports are allocated accordingly. Thus, the development work is nearly identical to a local setup \cite{vscodedevcontainer}.\newline
        % !TeX root = ../thesis_main.tex

\begin{lstlisting}[language=json,caption={\ac{VSCode}s Container Configuration File \code{devcontainer.json} },breaklines=true,label={code::devcontainer_json}]
{
  "name": "MY-PYTHON-APP",

  // Path to the docker-compose.yml file.
  "dockerComposeFile": ["../../docker-compose.yml"],

  // The service property for the container that VSCode uses.
  "service": "app",

  // The project folder to be opened by VSCode when connected.
  // Keep your containers running after VS Code shuts down.
  "workspaceFolder": "/workspace",
  "shutdownAction": "none",

  // Extensions to be installed when the container is created.
  "extensions": [
      "ms-python.python",
      "ms-python.vscode-pylance"
  ],

  // Runs this command after the container is created
  "postCreateCommand": "/workspace/.devcontainer/kill_app.sh"
}


\end{lstlisting}

        All these functions can be archived without \ac{VSCode} by using remote filesystem mounts, (\ac{SSH}) port-forwarding or terminal-based editors. Even without VSCode, any file changes made by any editor in the container will take effect, since the source code directories are bind-mounted. Nevertheless, it cannot be disregarded that adjustments to the \code{docker-compose.yml} file are necessarily compared to the setup described in section \ref{ssec::imp_approach}, when using editors other than \ac{VSCode}.

        \subsubsection{Variations and Additional Supporting Tools}
        In addition to the tools mentioned above and in section \ref{ssec::getting_devops}, other auxiliary programs can be used. To simplify certain workflows and automate recurring tasks, scripts will be used. The scripting language \wordhighlight{Bash} can be used natively on Linux and macOS, the installation of Git for Windows also brings Bash support to Windows. Accordingly, one uniform scripting language can be used to perform platform-independent operations and to simplify complex instructions. In order for developers not to have to wait for the creation of the DevContainer images, these should be built automatically by a \ac{CI}-tool and made available in a private container registry. Accordingly, uniform and up-to-date images are always available to all developers. \newline
        It is also possible to use other management tools via Docker. Database management tools, such as \wordhighlight{phpMyAdmin}, can simplify administration by adding further services. The same applies to the management of containers via \wordhighlight{Portainer}. Although Docker-Compose offers a color-coded log output, even this can become overwhelming if there are too many logs. Since the Docker stack is used anyway, enterprise log aggregators and analysis tools like \wordhighlight{Grafana} or \wordhighlight{Elastic-Search} can be used to get a persistent and searchable log dashboard. This even allows to reproduce and test the entire production service structure in a local environment. The configuration of these services will not be discussed further in this paper, as this is beyond the scope.

    \subsection{Security Aspects of this Concept}\label{ssec::sec}
    The presented solution is oriented to the development process and not to productive operation. For this reason, common security practices for container operation are not implemented. The images contain extra programs that are not mandatory, but convenient. The processes in the container run with super-user privileges to any avoid additional configuration effort and interruptions due to permission errors. In the \code{docker-compose.yml} file, passwords are defined in plain text to allow a consistent and easy setup across all systems. Functions and data within the containers are made available through exposed ports on the host. Accordingly, it must be ensured that the host is not accessible from the Internet. If certain services, such as databases or log dashboards, are shared between developers via a central server, it must be ensured that this server is only accessible within the company network in order to avoid the unintentional publication of confidential information. The same security measures must be applied for a non-containerized local development environment.

    \subsection{Strengths, Weaknesses and Limits}\label{ssec::limits}
    The software and methods described in section \ref{ssec::toolsused} are already standard practice in DevOps enabled teams. Accordingly, the entry barrier is small compared to new and unknown tools. DevContainer allow for a homogenization of development and production environment. The application runtime is identical, accordingly \ac{OS} or runtime specific errors are prevented. Developers can use a ready-to-go development project setups without having to install and configure the application runtime themselves. Dependencies, keys, supporting tools like debuggers, and configurations can already be shipped within the container to enable a quick initial setup. Instead of examining the environment for a long time in the event of an error, it can quickly be torn down and recreated to a known good state. The possibility to start a large service, consisting of many applications, with one command improves the workflow and allows for much more extensive testing possibilities. These are the key features that DevContainer promise to provide, in addition to solutions to the problems described in section \ref{sec::problem}. They extend the existing toolset of agile and DevOps teams by another tool that allows developers to be more flexible and to focus more on coding.\newline % To informal ... Like any other ...
    It should be noted that DevContainers are not a perfect solution to all problems in the software development stack. Like any other tool, they come with their own set of quirks. Expertise for the use of Docker containers must be available, and the production architecture must be transformable to a local DevContainer-based configuration. This configuration must be kept up to date and maintained. Developers may need to adapt to minor adjustments in their workflow. The \ac{CI}/\ac{CD} solution used must always be available and provide up-to-date container images. Every virtualization approach creates an additional overhead that cannot be ignored, especially when using multiple containers on Windows based systems. Details on the exact effects are given in section \ref{sec::eval}.\newline
    As mentioned before, there are also projects, for which the use of DevContainers is not suitable. Heavy monolithic applications are not in the sense of containers and therefore not ideal for DevContainers. Graphical applications can run in containers, but require a graphical X-Server on the host, so that the resulting experience is not comparable to a native \ac{UX}. Similar limitations apply to applications which require a Windows stack. Containers based on Windows do exist, but they require a Windows host, accordingly they are no longer platform-independent, requires additional licenses and further adaptations. For embedded projects that require specific hardware, a virtualization approach is not feasible, and, accordingly, these are not suitable for DevContainers either.\newline
    In order to demonstrate an appropriate use of DevContainers, the next section describes how to migrate from a traditional development environment to a DevContainer-based solution based on a real project.

\subsection{Alternative Solutions}\label{ssec::alternatives}
Apart from the concept presented here, there are already alternative solutions on the market which promise to solve problems similar to the ones described in section \ref{sec::problem}. A selection of these solutions is presented in the following. These alternative development environments can roughly be classified into two categories: browser-based and container-based solutions. A comparison of these to the solution presented here will be described in section \ref{sses::eval_compare}.
\myparagraph{Browser-Based Development Environments}
Browser-based development solutions provide an editor that is entirely built upon web technologies. Thereby, they can run on every device with a modern web browser. Their goal is to provide a quick functioning setup without any configuration.\newline
The products codesandbox.io, by CodeSandbox, and Stackblitz implement this approach. Both solutions offer a \ac{VSCode}-like editor in the browser and allow the development of NodeJS-based JavaScript projects. The functionality of codesandbox.io is provided by a client-server architecture where the browser-editor is the client and the server is managed by CodeSandbox itself. Therefore, Developers have little control over the actual runtime and always need an internet connection. Stackblitz, on the other hand, can also be used without an Internet connection. Stackblitz is a \ac{PWA}, which brings the NodeJS runtime into the browser by using Webassembly. Accordingly, Stackblitz does not need a backend-server, because it is already running in the browser. While codesandbox.io does not provide any console at all and is completely managed by the service provider, Stackblitz provides a minimal shell for installing, copying and launching files. However, since both solutions run within the browser, regular TCP/UDP connections are restricted, due to browser security regulations.  Network connections are only made available via HTTP or the HTTP-based WebSockets protocol \cite{codesandbox}, \cite{stackblitz}.

\myparagraph{Container-Based Development Environments}
Similar to the solution presented in this paper, this approach uses containers in, in which everything can be installed theoretically. Compared to browser-based solutions, container-based development environments offer broader functionality and are not restricted to one specific programming language. The source code and all the dependencies it takes to run it are bundled within one container. The best-known products that implement this are Gitpod and GitHub's Codespaces, which was not released until 2021. The infrastructure for the containers is offered by the provider, but GitPod also offers a self-hosting option. There are no restrictions for users in the containers, additional software can be added as desired, and regular network ports can also be opened to the outside. Both solutions can be used in the browser as well as in a local \ac{VSCode} installation via the remote development extension.\newline
Both services are billed either on a monthly basis or on an hourly usage basis \cite{githubcodespace}, \cite{gitpod}.

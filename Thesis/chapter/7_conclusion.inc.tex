% !TeX root = ../thesis_main.tex

\section{Conclusion}\label{sec::conclusion}
In the course of this thesis, basic definitions and practices of modern programming methodologies were described. Concepts form the DevOps culture were highlighted and the microservice architecture was described in more detail. By taking a closer look at the workflows for developing such a microservice or comparable architectures, various problems were identified that slow down the development speed and cause an increased error potential. The origin of these problems is mostly the heterogeneity of different systems and their manual configuration. Especially with regard to cloud native microservice applications, it has become clear that production great environments differs significantly from commonly used development setups. Differing operating systems, configurations and a limited testing setup are responsible for unexpected errors.\newline
This thesis has highlighted these limitations in current development environments and pointed out that there is potential for improvement. With the use of software from the DevOps and server world, a concept was created that addresses these shortcomings in local development environments. By abstracting the operating system through container technology, it can be ensured that applications can be reliably executed in well-defined consistent environments. This eliminates platform specific errors and enables a homogeneous development and production environment. The widely used container tool Docker was used implement this. To avoid configuration errors and manual tasks, all configuration-specific settings have been standardized and are stored in a \ac{VCS} via a common file format. This allows multiple applications to be orchestrated reliably and seamlessly without any manual effort. Several tools have been used in order to provide developers with a fast, error-resistant and familiar development setups, which also allows the use of local, non-container based, programs.\newline
The implementation of the concept to a prototype has shown that it is quite applicable in the real world. Depending on the application, adjustments to the settings of the developer tools may be necessary in order to use functions such as hot-reloading. A measurable overhead of this concept could also be determined in the performance evaluation when the concept is applied to Windows hosts. However, the time saved by the automatic management of the application processes and their environment, as well as the significantly reduced potential for errors and the greater choice of software, demonstrate the success of this concept.\newline
The evaluation has further shown that the concept presented here is quite relevant for large software companies. The comparison with the, still quite new, commercial alternatives Codespaces and Gitpod shows that these companies see a market for platform-independent, pre-configured development setups. This concept allows to actively develop software with any internet capable device, regardless of the computing capacity, instead of focusing on the configuration and maintenance of one's setup.

\myparagraph{Further Potential}
All solutions presented here are still in a very early state (of their development). Gitpod was announced in 2019, while Codespaces was released in 2021, during the writing process of this thesis. Like DevContainers, these services promise to provide a uniformly managed development infrastructure. This concept intends to allow developers to focus on contributing to the project and is meant to prevent interruptions due to maintenance and errors in their own working environments. Through the power of cloud computing, computing capacity can be booked at will in order to provide products and services independent of the underlying hardware. In the event of an application instance failure, it is simply discarded and rebuilt the instance. Now this is also possible with containerized development environments. In the case of Codespaces and Gitpod, the available computing capacity can even be increased without having to purchase new devices for developers. This kind of external hosting allows the usage of any device for development. The increasing number of mobile devices now can connect to powerful development servers. This offers new possibilities for mobile working setups. DevContainers are also quite suitable for other development platforms. With the tool \code{docker buildx}, developers can compile code for other processor architectures such as ARM on their primary x86 bases system. Developers in the embedded world can develop and test their software with only one device without the need to manually copying data to an external additional device.\newline
The concept of DevContainers offers considerable opportunities for opensource projects that rely on contributions of voluntary developers. Projects with extensive and complicated setup requirements can be discouraged and thus miss out on valuable contributors and employees. For such projects, DevContainers, with a predefined environment which works independently of the host, are an easy entry point in order to offer volunteers a working project environment. While the limitations of browser-based development environments can be too restrictive for professional developers, they still offer potential for the use in education. Pupils or students can have a working environment within seconds, in which they can gain hands-on experience with tutors, regardless of used device.\newline
Based on this work, the question arises for a comprehensive study of the usability of these container-based development techniques. This study should particularly take into account a possible gain in productivity and the different cost models of all solutions.

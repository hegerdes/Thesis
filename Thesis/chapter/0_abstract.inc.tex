% !TeX root = ../thesis_main.tex

\section*{Abstract}
\begin{otherlanguage}{ngerman}
    \renewcommand{\abstractname}{}
    \begin{abstract}
        \textbf{Deutsch:}
        Das Wachstum von Cloud Computing und die Möglichkeit zum schnellen, automatischen Bereitstellen von skalierbaren Anwendungen führt zur stetig steigenden Beschleunigung für die Verfügbarkeit von Softwarelösungen. Die Unabhängigkeit von Hardware und Betriebssystem wird dabei durch Containerisierung, einer Form der Virtualisierung, erreicht. Während dies zum Standard für den produktiven Betrieb von Software in der Cloud geworden ist, wird während des Entwicklungsprozesses weiterhin auf die lokale Installation und Konfiguration von individuellen Entwicklersystemen gesetzt, welche manuelle, zeitintensive Konfiguration benötigen, wodurch es zur erhöhten Fehleranfälligkeit kommt. Diese Arbeit schlägt den Einsatz von Containerisierung für den Entwicklungsprozess von Software vor. Diese Entwicklungscontainer (DevContainer) nutzen Prozessisolierung um eine schnellere, weniger fehleranfällige, homogene Entwicklungslandschaft bereitzustellen. Das konzeptionelle Design einer solchen Entwicklungsumgebung wird beschrieben und anschließend mittels einer exemplarischen Umsetzung auf Anwendbarkeit überprüft. Die daraus gewonnenen Erkenntnisse werden mit kommerziellen Alternativen verglichen und zeigen das Potenzial von DevContainer für effizientere Entwicklungsprozesse.
    \end{abstract}
\end{otherlanguage}
\begin{otherlanguage}{english}
    \renewcommand{\abstractname}{}
    \begin{abstract}
        \textbf{English:}
        The growth of cloud computing and the possibility of rapid, automatic deployments of scalable applications is leading to an ever-growing acceleration in the availability of software solutions. Thereby, containerization a form of virtualization offers independence form hardware and operating system. While this approach has become the default in the cloud for offering software solutions, the development process continues to rely on local configuration and operation on individual systems. These systems require manual, time-intensive configuration, leading to increased error-proneness. This thesis proposes the use of the containerization concept for the development process of software. These development containers (DevContainers) use process isolation to provide a faster, less error-prone and homogeneous development landscape. The conceptual design of such a development environment is described, and then exemplary implemented in order to review the applicability. The resulting findings are compared to commercial alternatives and the potential of DevContainers for more efficient development processes are stated.
    \end{abstract}
\end{otherlanguage}
\newpage


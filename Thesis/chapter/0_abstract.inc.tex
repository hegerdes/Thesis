% !TeX root = ../thesis_main.tex

\section*{Abstract}
\begin{otherlanguage}{ngerman}
    \renewcommand{\abstractname}{}
    \begin{abstract}
        \textbf{Deutsch:}
        Das Wachstum von Cloud Computing und die Möglichkeit zum schnellen, automatisch bereitstellen von skalierbaren Anwendungen, führt zur stetig steigenden Beschleunigung für die Verfügbarkeit von Softwarelösungen. Die Unabhängigkeit von Hardware und Betriebssystem wird dabei durch Containerisierung, einer Form der Virtualisierung erreicht. Während dies zum Standard für den produktiven Betrieb von Software in der Cloud geworden ist, wird während des Entwicklungsprozesses weiterhin auf die lokale Installation und Konfiguration von individuellen Entwicklersystemen gesetzt, welche manuelle, zeitintensive Konfiguration benötigt, wodurch es zur erhöhten Fehleranfälligkeit kommt.\newline
        Diese Arbeit schlägt den Einsatz von Containerisierung, für den Entwicklungsprozess von Software vor. Diese Entwicklungscontainer (DevContainer) nutzen Prozessisolierung um eine schnellere, weniger Fehler anfällige homogene Entwicklungslandschaft bereitzustellen. Das konzeptionelle Design einer solchen Entwicklungsumgebungen wird beschrieben und anschließend auf Anwendbarkeit überprüft. Die daraus gewonnenen Erkenntnisse werden mit kommerziellen Alternativen verglichen und zeigen dessen Potenzial für effizientere Entwicklungsprozesse.
    \end{abstract}
\end{otherlanguage}
\begin{otherlanguage}{english}
    \renewcommand{\abstractname}{}
    \begin{abstract}
        \textbf{English:}
        The growth of cloud computing and the possibility of rapid, automatic deployment of scalable applications is leading to an ever greater acceleration in the availability of software solutions. Hardware and operating system independence is achieved through containerization, a form of virtualization. While this has become the standard for the productive operation of software in the cloud, the development process continues to rely on the local installation and configuration of individual developer systems. These require manual, time-intensive configuration, leading to increased error-proneness.\newline
        This work proposes the use of the containerization concept, for the development process of software. These development containers (DevContainers) use process isolation to provide a faster, less error prone homogeneous development landscape. The conceptual design of such a development environment is described and then tested for applicability. The lessons learned are compared to commercial alternatives and show its potential for more efficient development processes.
    \end{abstract}
\end{otherlanguage}
\newpage




% ORI:
% Das Wachstum von Cloud Computing und die Möglichkeit zum schnellen, automatisch deployen von skalierbaren Anwendungen, führt zu einer immer größeren Beschleunigung für die Verfügbarkeit von Softwarelösungen.  Die Unabhängigkeit von der Hardware. Betriebssystem, installierten Anwendungen und dessen Konfiguration wird dabei durch Containerisierung, einer Form der Virtualisierung erreicht. Während dies zum Standard für den produktiven Betrieb von Softwarelösungen in der Cloud geworden ist wird während des Entwicklungsprozesses dieser Anwendungen weiterhin auf die lokale installation und Konfiguration von individuellen Entwicklersystemen gesetzt. Diese benötigen manuelle, zeitintensive Konfiguration, unterscheiden sich von der produktiv eingesetzten Umgebung wodurch zur erhöhten Fehleranfälligkeit kommt.
% Diese Bachelorarbeit schlägt den Einsatz von des in der Cloud genutzten Containerisierung Konzepts, für den Entwicklungsprozess von Software vor. Diese Entwicklungscontainer (DevContainer) nutzen Isolierungs- und Abstraktionsebende von Containertechnologie um eine homogenere Entwicklungslandschaft für eine schnellere und weniger Fehler anfällige Entwicklungsumgebung bereitzustellen. Der Nutzen und das konzeptionelle Design einer solchen Entwicklungsumgebungen wird beschrieben und anschließend auf Anwendbarkeit mittels eines Prototypen überprüft. Die daraus gewonnenen Erkenntnisse werden mit erst kürzlich vorgestellten kommerziellen Alternativen wie GitHubs Codespaces verglichen und zeigen dessen potential für effizientere und modernere Entwicklungsprozesse von Software.
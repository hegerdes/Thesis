% !TeX root = ../thesis_main.tex

\section{Future Potential and Outlook}\label{sec::outlook}
This section describes the further potential of the concept presented earlier and what improvements can still be made. Furthermore, an insight into further possible uses and a possible future of this concept is given.

\myparagraph{Possible Improvments}
Docker enables access to software, which were previously exclusively available in the server world. This software allows to further improve the concept presented above. Instead of directly exposing application ports, a reverse proxy can be placed in front of every application. Only the port of the proxy is then exposed to the host. This allows developers to access their frontend by requesting \code{forntend.myapp}. The proxy than automatically forwards the requests to the appropriate application. Accordingly, Developers do not need to remember application ports or have to pay attention to port collisions on \code{localhost}. The reverse proxy routes all requests thought the internal docker network to the appropriate application. This is another strategy that is often used in the production environment, making the DevContainer environment even more similar to the production one. Furthermore, to minimize the load on the developer's machine, static services can be made available on a shared internal server within the company. The MQTT-broker or the \ac{SSH} server may not be needed on every machine and can be shared. There is also room for improvements in the interaction of non-Linux host systems with the containers. Investing work in improving support for Windows file-systems could have a big impact on performance.

\myparagraph{Further Potential}
All solutions presented here are still in a very early state (of their development). Gitpod was announced in 2019, while Codespaces was released in 2021, during the writing process of this thesis. Like DevContainers, these services promise to provide a uniformly managed development infrastructure. This concept intends to allow developers to focus on contributing to the project and is meant to prevent interruptions due to maintenance and errors in their own working environments. Through the power of cloud computing, computing capacity can be booked at will in order to provide products and services independent of the underlying hardware. In the event of an application instance failure, it is simply discarded and rebuilt the instance. Now this is also possible with containerized development environments. In the case of Codespaces and Gitpod, the available computing capacity can even be increased without having to purchase new devices for developers. This kind of external hosting allows the usage of any device for development. The increasing number of mobile devices now can connect to powerful development servers and offers new possibilities for mobile working. DevContainers are also quite suitable for other development platforms. With the tool \code{docker buildx}, developers can compile code for other processor architectures such as ARM on their primary computer. Developers in the embedded world can develop and test their software with only one device without the need to manually copying data to an additional device.\newline
The concept of DevContainers offers considerable opportunities for opensource projects that rely on contributions of voluntary developers. Projects with extensive and complicated setup requirements can be discouraged and thus miss out on valuable contributors and employees. For such projects, DevContainers, with a predefined environment which works independently of the host, are an easy entry point in order to offer volunteers a working project environment. While the limitations of browser-based development environments can be too restrictive for professional developers, they still offer potential for the use in education. Pupils or students can have a working environment within seconds, in which they can gain hands-on experience with tutors, regardless of used device.\newline
Based on this work, the question arises for a comprehensive study of the usability of these container-based development techniques. This study should particularly take into account a possible gain in productivity and the different cost models of all solutions.

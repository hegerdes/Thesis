% !TeX root = ../thesis_main.tex

\section{Future Potential and Outlook}\label{sec::outlook}
All solutions presented here can still be described as very early state. Gitpod was announced in 2019 while Codespaces was released in 2021, during the writing of this paper. Like DevContainer, these services promise to provide a unified managed development infrastructure. This is intended to allow developers to focus on contributing to the project and prevent interruptions due to maintenance and errors of their own working environments. Through the cloud, computing capacity can be booked at will to provide products and services independent of the underlying hardware. In the event of an application instance failure, it is simply discarded and rebuilt. This is also possible with containerized development environments. In the case of Codespaces and Gitpod, the available computing capacity can even be increased without having to purchase new end devices for developers. This kind of external hosting allows to use any device for development. Especially due to the increasing number of mobile devices, new forms of mobile working are created.\newline
The concept of DevContainern offers particular opportunities for opensource projects that rely on the voluntary cooperation of developers. Projects with extensive and complicated setup requirements can be discouraged and thus miss out on valuable employees. For such projects, DevContainers can be offered to provide volunteers with a working, predefined environment that works independently of their local setup. While the limitations of browser-based development environments can be too restrictive for professional developers, they still offer potential for use in education. Pupils or students can have a working environment within seconds in which they can gain hands-on experience with tutors, regardless of the device.\newline
Based on this work, the question arises for a comprehensive study of the usability of these container-based development techniques. In particular, taking into account a possible productivity gain and the different cost models of all solutions.

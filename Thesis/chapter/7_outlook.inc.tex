% !TeX root = ../thesis_main.tex

\section{Future Potential and Outlook}\label{sec::outlook}
All solutions presented here are still in a very early state (of their development). Gitpod was announced in 2019, while Codespaces was released in 2021, during the writing process of this thesis. Like DevContainers, these services promise to provide a uniformly managed development infrastructure. This procedure intends to allow developers to focus on contributing to the project and is meant to prevent interruptions due to maintenance and errors in their own working environments. Through the power of cloud computing, computing capacity can be booked at will in order to provide products and services independent of the underlying hardware. In the event of an application instance failure, it is simply discarded and rebuilt the instance. Now this is also possible with containerized development environments. In the case of Codespaces and Gitpod, the available computing capacity can even be increased without having to purchase new devices for developers. This kind of external hosting allows the usage of any device for development. The increasing number of mobile devices now can connect to powerful servers and offers new possibilities for mobile working.\newline
The concept of DevContainers offers considerable opportunities for opensource projects that rely on contributions of voluntary developers. Projects with extensive and complicated setup requirements can be discouraged and thus miss out on valuable contributors and employees. For such projects, DevContainers, with a predefined environment which works independently of the host, are an easy entry point in order to offer volunteers a working project environment. While the limitations of browser-based development environments can be too restrictive for professional developers, they still offer potential for the use in education. Pupils or students can have a working environment within seconds, in which they can gain hands-on experience with tutors, regardless of used device.\newline
Based on this work, the question arises for a comprehensive study of the usability of these container-based development techniques. This study should particularly take into account a possible gain in productivity and the different cost models of all solutions.

% !TeX root = ../thesis_main.tex

\section{Conclusion}\label{sec::conclusion}

This work has highlighted limitations in current development environments and made clear that there is potential for improvement in these. Especially with regard to cloud native microservice applications, it has become clear that the productive environment differs significantly from the most commonly used development setup.  Different operating systems, different configurations and limited testing possibilities can be significantly responsible for unexpected errors.\newline
With the use of software from the DevOps world, a concept could be created that addresses these shortcomings in local development environments. Basic concepts, practices and programs from the DevOps world were explained and shown how they can be used for local development. The implementation of the concept in a prototype has shown that it is quite applicable and what challenges they bring with them. The evaluation further showed that previously identified problems could indeed be solved, even if there can be noticeable performance limitations when using Windows as the host.
However, the existence and increasing prevalence of conceptually similar alternatives in the commercial market shows that the concept of virtualized development environments is of interest in the market.


% Diese Arbeit hat Limitierungen in aktuellen Entwicklungsumbegnungen aufgezeigt und deutlich gemacht das in diesen Potential für Verbesserungen vorhanden ist. Besonders im Hinblick auf Cloud native microservice Anwendungen ist deutlich geworden, dass sich die produktive Umgebung maßgeblich von des zumeist eingesetzten Entwickungssetups unterscheidet.  Verschiedene Betriebssysteme, unterschiedliche konfigurationen und eingeschränkte Testmöglichkeiten können maßgeblich für unerwartete Fehler verantwortlich sein.
% Mit der Verwendung von Software aus der DevOps welt konnte ein Konzept geschaffen werden welches diese unzulänglichkeiten in lokalen Entwicklungsumgebungen beheben. Dabei wurden grundlegende Begriffe, Praktiken und Porgramme aus der DevOps Welt erleutert und gezeigt wie diese für die lokale Entwickung eingesetzt werden konnen.
% Die Umsetzung des Konzeptes in einem Prototypen hat gezeigt, dass dieses durchaus anwendbar ist und welche Herausforderungen diese mitsichbringen. Die Evaluierung zeigte weiter, dass zuvor identifizierten Probleme in der tat gelöst werden konnten, auch wenn es bei der Verwendung von Windows als Host zu bemerkbaren Performancelimitierungen kommen kann.
% Die Existence und die steigende verbreitung von konzeptionell ahnlichen Alternativen auf dem kommerziellen markt zeigt jedoch das Konzept von Virtualisierten Entwicklungsumgebungen von Interesse auf dem Markt ist.
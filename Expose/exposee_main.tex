\documentclass[12pt, a4paper]{article}

% !TeX root = ../main.tex
\usepackage{a4wide}

\usepackage[utf8]{inputenc}

%\usepackage[ngerman]{babel}
\usepackage[english]{babel}
\usepackage[T1]{fontenc}
\usepackage{palatino}
\usepackage{graphicx}
\usepackage{caption}
\usepackage{url}
\usepackage{acronym}
\usepackage{tocloft}
\usepackage{mathpazo}
\usepackage{amsmath}
\usepackage{amsfonts}
\usepackage{adjustbox}
\usepackage{hhline}
\usepackage{fancyhdr}
\usepackage{amssymb}
\usepackage{floatflt}
\usepackage{setspace}
\usepackage{float}
\usepackage{booktabs}
\usepackage{color}
\usepackage{enumitem}
\usepackage{listings}
\usepackage{array}
\usepackage{scrhack}
\usepackage[inkscapearea=page]{svg}
\usepackage{xcolor}
\usepackage{wrapfig}
\usepackage[hidelinks]{hyperref}
\usepackage{lmodern}
\usepackage{multirow}
\usepackage{tabularx}
\usepackage{etoolbox}
\usepackage{subfig}
\usepackage{cleveref}
\usepackage{pstricks}
\usepackage{lipsum}
\usepackage[bottom, hang]{footmisc}
\usepackage{tikz}
\usetikzlibrary{shapes,arrows,calc,positioning,fit}
% !TeX root = ../main.tex
%%%%%%%%%%%%%%%%%%%%%%%%%%%%%%%%%%%%%%%%%%%%%%%%%%%%%%%%%%%%%%%%%%%%%%%%
% Data about you and the Document%
%%%%%%%%%%%%%%%%%%%%%%%%%%%%%%%%%%%%%%%%%%%%%%%%%%%%%%%%%%%%%%%%%%%%%%%%

% % Main Title of Document:
\newcommand{\myMaintitle}{Untersuchung und Aufbau einer DevOps Development Umgebung}

% % Sub Title of DocInput:
\newcommand{\mySubtitle}{Developing holistic software solutions through integration of existing individual solutions.}

% % Ihr Name:
\newcommand{\myName}{Henrik Gerdes}

% % Matrikelnummer:
\newcommand{\myMatrikel}{MatNr: 969272}

% % Ihr Geburtsort:
\newcommand{\brith}{Osnabrück}

% % Ihr Geburtsort:
\newcommand{\place}{Osnabrück}

% % Ihr Abgabedatum:
\newcommand{\submission}{\today}

% % Ihr Abgabedatum:
\newcommand{\mycourse}{Exposé für den B.Sc.}

% % Name des Betreuers/Erstprüfenden:
\newcommand{\fistSupervisor}{Dennis Ziegenhagen}
\newcommand{\secSupervisor}{Achim Hendrix}

% % In welchem Semester befinden Sie sich?
\newcommand{\mySemester}{6. Semester}

\title{\myMaintitle}

\author{\myName}
% !TeX root = ../main.tex
% % Zeilenabstand im Haupttext auf anderthalb-zeilig setzen
%\linespread{1.25}\selectfont

% Line spacing
%\onehalfspacing{}

%Pfad für Grafiken
\graphicspath{{fig/}}

%Styleregeln
\widowpenalty10000 % Vermeidet einzelne Zeilen eines Absatzes zu Beginn einer Seite
\clubpenalty10000 % Vermeidet einzelne Zeilen eines Absatzes am Ende einer Seite
\addtocontents{toc}{\protect\sloppy}
\setcounter{tocdepth}{3}


% % \sloppy bewirkt, dass Latex beim Blocksatz nicht über den rechten Rand hinausschreibt.
% % und dafür größere Lücken in einer Zeile in Kauf nimmt
\sloppy

% % Setzt Dokumenteigenschaften für PDFs, wenn das Paket 'hyperref' geladen wurde.
\hypersetup{pdftitle=\myMaintitle,pdfauthor=\myName,bookmarksopen=true}

%Source for picture captions
\newcommand{\source}[1]{\caption*{Source: {#1}} }

\newcommand{\code}[1]{\texttt{#1}}

\newcommand{\myparagraph}[1]{\paragraph{#1}\mbox{}\\}

\newcommand{\RM}[1]{\MakeUppercase{\romannumeral{} #1{}}}

\newcommand{\HRule}{\rule{\linewidth}{0.5mm}} % Defines a new command for horizontal


\definecolor{dkgreen}{rgb}{0,0.6,0}
\definecolor{gray}{rgb}{0.5,0.5,0.5}
\definecolor{mauve}{rgb}{0.58,0,0.82}

\lstset{ %
  language=Java,                  % the language of the code
  basicstyle=\footnotesize,       % the size of the fonts that are used for the code
  numbers=left,                   % where to put the line-numbers
  numberstyle=\tiny\color{gray},  % the style that is used for the line-numbers
  stepnumber=1,                   % the step between two line-numbers. If it's 1, each line
                                  % will be numbered
  numbersep=5pt,                  % how far the line-numbers are from the code
  backgroundcolor=\color{white},  % choose the background color. You must add \usepackage{color}
  showspaces=false,               % show spaces adding particular underscores
  showstringspaces=false,         % underline spaces within strings
  showtabs=false,                 % show tabs within strings adding particular underscores
  frame=single,                   % adds a frame around the code
  rulecolor=\color{black},        % if not set, the frame-color may be changed on line-breaks within not-black text (e.g. commens (green here))
  tabsize=4,                      % sets default tabsize to 2 spaces
  captionpos=b,                   % sets the caption-position to bottom
  breaklines=true,                % sets automatic line breaking
  breakatwhitespace=false,        % sets if automatic breaks should only happen at whitespace
  title=\lstname,                 % show the filename of files included with \lstinputlisting;
                                  % also try caption instead of title
  keywordstyle=\color{blue},          % keyword style
  commentstyle=\color{dkgreen},       % comment style
  stringstyle=\color{mauve}         % string literal style
}

%%%%%%%%%%%%%%%%%%%%%%%%%%%%%%%%%%%%%%%%%%%%%%%%%%%%%%%%%%%%%%%%%%%%%%%%%%%%%%%%%%%%%%%%%
%Examples
%%%%%%%%%%%%%%%%%%%%%%%%%%%%%%%%%%%%%%%%%%%%%%%%%%%%%%%%%%%%%%%%%%%%%%%%%%%%%%%%%%%%%%%%%
% \pdfmarkupcomment[markup=Squiggly,color=green]{with pdfcomment}{move to the front}.
% \pdfmarkupcomment[markup=StrikeOut,color=red]{stupid}{replace stupid with funny}
% \pdfmarkupcomment[markup=Highlight,color=yellow]{Of course, you can highlight complete sentences.}{Highlight}
% \pdfcomment[icon=Note,color=blue]{insert graphic!}

\begin{document}
\nocite{*}

\pagenumbering{gobble}
\include{inc/title.inc}

\tableofcontents
\newpage
\newcounter{lastroman}
\setcounter{lastroman}{\value{page}}

\pagestyle{plain}
\pagenumbering{arabic}
% \maketitle

\section{Topic, Content and Structure}
\subsection{Topic}
With the continuing shift towards microservices in the software industry, more and more new tools are being developed to operate, scale and maintain services. The microservice culture stipulates new features are delivered to customers faster and more frequently, bringing developers and administrators closer together. The resulting principals for this short agile development cycle are combined under the concept of DevOps. While analyses, guidelines and experiences for the production operation of DevOps are available extensively, the benefits of their toolset for the use in development environments have hardly been analyzed yet.\newline
Individual Development environments are quiet unique for every developer, reflect personal preferences are often time-consuming and complex to setup. Microservices and DevOps tools can help to quickly set up a consistent development environments and yet provide the freedom to let developers use the preferred applications. Companies like Docker Inc. and GitHub/Microsoft are also pushing this concept forward with products like Development-Containers and CodeSpaces that allow users to quickly checkout code form any location even within a browser to test and apply changes to their applications.
\subsection{Goal}
The goal of the bachelor thesis is to provide an overview of development tools in a DevOps environment and to develop an exemplary container-based developer environment based on a practical example. The advantages and disadvantages of such an environment will be discussed and analyzed. If possible, a small evaluation based on practical experience will also be part of the theses and Knowledge gain.
\subsection{Research status}
\subsection{Concept}
The increasing demand and complexity of software alongside more frequent feature requests in the software industry resulted in creation and adoption of microservices. With the broader availability of containerization solutions and tools this trend continues. Containerization allows software to be independent of the host environment that runs the code. The container already ships with a pre- and well-defined runtime environment and therefore decouples the link between host and software. Multiple instances of the software can be deployed quickly and on demand without the need of tedious, time consuming and error-prone host configuration. However, these advantages also bring new challenges. Monolithic applications are developed and tested as one unit, components are tightly coupled so that errors between components can be found quickly. Microservices are often orchestrated and provide together one extensive application. Changes in one service may break other parts of the application, which often only becomes apparent during operation in the overall orchestrated system. Continuous Integration and Continuous Deployment are two concepts, which are increasingly adopted, that try to solve this problem. The creation and practice of these concepts are bundled under the principles of DevOps culture. While there are already comprehensive guides and materials for implementing CI/CD in DevOps, these mostly focus on the process of testing and delivery these resources fall short in the management of the development environment.\newline
Development environments are highly heterogeneous. Developers use different operating systems in various configurations, different versions of languages and their preferred tools. Further development environments have additional requirements like usage of demo data, source-maps, debug-symbols and opt-out security features. It is not uncommon that configuring such an environment takes a lot of time. The fact that these development endowments differ to the CI/CD builds may introduce unexpected behavior. This bachelor theses introduce the idea of using the same containerization approach and CI/CD pipeline to build well-defined development environments and evaluates the idea based on a practical example project.\newline
The process of building, application and usability of such a developer environment will be described in this bachelor thesis by means of an practical example.
\subsection{Motivation}
\section{Outline draft}
\section{Schedule}
\section{Literatur}
\bibliographystyle{IEEEtranSA}
\bibliography{bib/sources}
\newpage

% Anhang
\renewcommand{\thesubsection}{\Alph{subsection}}
\pagenumbering{Roman}
\setcounter{page}{\value{lastroman}}
\section*{Appendix}
\addcontentsline{toc}{section}{Appendix}
\listoffigures
\include{inc/ensure.inc}

\end{document}
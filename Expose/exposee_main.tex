\documentclass[12pt, a4paper]{article}

% !TeX root = ../main.tex
\usepackage{a4wide}

\usepackage[utf8]{inputenc}

%\usepackage[ngerman]{babel}
\usepackage[english]{babel}
\usepackage[T1]{fontenc}
\usepackage{palatino}
\usepackage{graphicx}
\usepackage{caption}
\usepackage{url}
\usepackage{acronym}
\usepackage{tocloft}
\usepackage{mathpazo}
\usepackage{amsmath}
\usepackage{amsfonts}
\usepackage{adjustbox}
\usepackage{hhline}
\usepackage{fancyhdr}
\usepackage{amssymb}
\usepackage{floatflt}
\usepackage{setspace}
\usepackage{float}
\usepackage{booktabs}
\usepackage{color}
\usepackage{enumitem}
\usepackage{listings}
\usepackage{array}
\usepackage{scrhack}
\usepackage[inkscapearea=page]{svg}
\usepackage{xcolor}
\usepackage{wrapfig}
\usepackage[hidelinks]{hyperref}
\usepackage{lmodern}
\usepackage{multirow}
\usepackage{tabularx}
\usepackage{etoolbox}
\usepackage{subfig}
\usepackage{cleveref}
\usepackage{pstricks}
\usepackage{lipsum}
\usepackage[bottom, hang]{footmisc}
\usepackage{tikz}
\usetikzlibrary{shapes,arrows,calc,positioning,fit}
% !TeX root = ../main.tex
%%%%%%%%%%%%%%%%%%%%%%%%%%%%%%%%%%%%%%%%%%%%%%%%%%%%%%%%%%%%%%%%%%%%%%%%
% Data about you and the Document%
%%%%%%%%%%%%%%%%%%%%%%%%%%%%%%%%%%%%%%%%%%%%%%%%%%%%%%%%%%%%%%%%%%%%%%%%

% % Main Title of Document:
\newcommand{\myMaintitle}{Untersuchung und Aufbau einer DevOps Development Umgebung}

% % Sub Title of DocInput:
\newcommand{\mySubtitle}{Developing holistic software solutions through integration of existing individual solutions.}

% % Ihr Name:
\newcommand{\myName}{Henrik Gerdes}

% % Matrikelnummer:
\newcommand{\myMatrikel}{MatNr: 969272}

% % Ihr Geburtsort:
\newcommand{\brith}{Osnabrück}

% % Ihr Geburtsort:
\newcommand{\place}{Osnabrück}

% % Ihr Abgabedatum:
\newcommand{\submission}{\today}

% % Ihr Abgabedatum:
\newcommand{\mycourse}{Exposé für den B.Sc.}

% % Name des Betreuers/Erstprüfenden:
\newcommand{\fistSupervisor}{Dennis Ziegenhagen}
\newcommand{\secSupervisor}{Achim Hendrix}

% % In welchem Semester befinden Sie sich?
\newcommand{\mySemester}{6. Semester}

\title{\myMaintitle}

\author{\myName}
% !TeX root = ../main.tex
% % Zeilenabstand im Haupttext auf anderthalb-zeilig setzen
%\linespread{1.25}\selectfont

% Line spacing
%\onehalfspacing{}

%Pfad für Grafiken
\graphicspath{{fig/}}

%Styleregeln
\widowpenalty10000 % Vermeidet einzelne Zeilen eines Absatzes zu Beginn einer Seite
\clubpenalty10000 % Vermeidet einzelne Zeilen eines Absatzes am Ende einer Seite
\addtocontents{toc}{\protect\sloppy}
\setcounter{tocdepth}{3}


% % \sloppy bewirkt, dass Latex beim Blocksatz nicht über den rechten Rand hinausschreibt.
% % und dafür größere Lücken in einer Zeile in Kauf nimmt
\sloppy

% % Setzt Dokumenteigenschaften für PDFs, wenn das Paket 'hyperref' geladen wurde.
\hypersetup{pdftitle=\myMaintitle,pdfauthor=\myName,bookmarksopen=true}

%Source for picture captions
\newcommand{\source}[1]{\caption*{Source: {#1}} }

\newcommand{\code}[1]{\texttt{#1}}

\newcommand{\myparagraph}[1]{\paragraph{#1}\mbox{}\\}

\newcommand{\RM}[1]{\MakeUppercase{\romannumeral{} #1{}}}

\newcommand{\HRule}{\rule{\linewidth}{0.5mm}} % Defines a new command for horizontal


\definecolor{dkgreen}{rgb}{0,0.6,0}
\definecolor{gray}{rgb}{0.5,0.5,0.5}
\definecolor{mauve}{rgb}{0.58,0,0.82}

\lstset{ %
  language=Java,                  % the language of the code
  basicstyle=\footnotesize,       % the size of the fonts that are used for the code
  numbers=left,                   % where to put the line-numbers
  numberstyle=\tiny\color{gray},  % the style that is used for the line-numbers
  stepnumber=1,                   % the step between two line-numbers. If it's 1, each line
                                  % will be numbered
  numbersep=5pt,                  % how far the line-numbers are from the code
  backgroundcolor=\color{white},  % choose the background color. You must add \usepackage{color}
  showspaces=false,               % show spaces adding particular underscores
  showstringspaces=false,         % underline spaces within strings
  showtabs=false,                 % show tabs within strings adding particular underscores
  frame=single,                   % adds a frame around the code
  rulecolor=\color{black},        % if not set, the frame-color may be changed on line-breaks within not-black text (e.g. commens (green here))
  tabsize=4,                      % sets default tabsize to 2 spaces
  captionpos=b,                   % sets the caption-position to bottom
  breaklines=true,                % sets automatic line breaking
  breakatwhitespace=false,        % sets if automatic breaks should only happen at whitespace
  title=\lstname,                 % show the filename of files included with \lstinputlisting;
                                  % also try caption instead of title
  keywordstyle=\color{blue},          % keyword style
  commentstyle=\color{dkgreen},       % comment style
  stringstyle=\color{mauve}         % string literal style
}

%%%%%%%%%%%%%%%%%%%%%%%%%%%%%%%%%%%%%%%%%%%%%%%%%%%%%%%%%%%%%%%%%%%%%%%%%%%%%%%%%%%%%%%%%
%Examples
%%%%%%%%%%%%%%%%%%%%%%%%%%%%%%%%%%%%%%%%%%%%%%%%%%%%%%%%%%%%%%%%%%%%%%%%%%%%%%%%%%%%%%%%%
% \pdfmarkupcomment[markup=Squiggly,color=green]{with pdfcomment}{move to the front}.
% \pdfmarkupcomment[markup=StrikeOut,color=red]{stupid}{replace stupid with funny}
% \pdfmarkupcomment[markup=Highlight,color=yellow]{Of course, you can highlight complete sentences.}{Highlight}
% \pdfcomment[icon=Note,color=blue]{insert graphic!}

\begin{document}
\nocite{*}

\pagenumbering{gobble}
\include{inc/title.inc}

\tableofcontents
\newpage
\newcounter{lastroman}
\setcounter{lastroman}{\value{page}}

\pagestyle{plain}
\pagenumbering{arabic}
% \maketitle

% With the continuing shift towards microservices in the software industry, more and more new tools are being developed to operate, scale and maintain services. The microservice culture stipulates that new features are delivered faster and thereby more frequent to customers, bringing developers and administrators closer together. The resulting principals for this short, agile development cycle are combined under the concept of DevOps. While analyses, guidelines and experiences for the production operation of DevOps are available extensively, the benefits of their toolset for the use in development environments have hardly been analyzed yet.\newline
% Individual development environments are quite unique for every developer, reflect personal preferences and are often time-consuming and complex to setup. Microservices and DevOps tools can help to instantly set up a consistent development environments and yet provide the freedom to let developers use the preferred applications.

\section{Topic, Content and Structure}
The following sections describe the general topic, current state of it and the concept of the bachelor thesis.
\subsection{Topic}
% Data mix unterschiedlicher Kunden DB
The rise of Microservices was accelerated by the emergence of new technologies and changed the way \textit{where} and \textit{how} software was running. Applications could be distributed and scaled quickly through the cloud which benefited customers favors for quick changes. Mythologies such as version control, \ac*{CI}, \ac*{CD} with automated integration and end-to-end testing alongside monitoring are common in modern microservice based products. Yet the actual coding setup has not changed significantly. The development environment consists of an operating system, which is most likly differs to the production environment, locally installed application, libraries and the preferred \ac{IDE}. The initial setup of these tools tend to take quiet some time. The significant differences between development and production environments are predetermined for configuration and runtime based errors. These errors are found at the earliest during the integration test of the CI and at the latest during operation. The later a bug is found, the more time and cost it takes to fix it, increasing development costs and lowering customer experience.\newline
Besides the configuration errors, developers most certainly work on more than one project at a time. These projects may need different runtimes or liberties that are not compatible with each other. Projects with legacy code require runtimes that are may no longer be supported at all.\newline
These problems, among others, complicate and delay the (further) development of applications. Technologies and tools from the Microservices and DevOps environment promises possibilities to solve these problems. This bachelor thesis will exemplify requirements of modern development environments, pick a selection of DevOps tools and discuss their practicability based on a practical project.
\subsection{Goal}
The goal of this bachelor thesis is to pick several tools from the Microservice and DevOps approach and analyze there practicability for the actual coding process. The problems of current environments are described, possible solutions are presented and underpinned by a practical project.  An overview of the Microservice and DevOps world is given as a basis. There will be a description of the tools that are practically suitable to be used in a development environment alongside with the creation of a practical container-based developer environment example. The advantages and disadvantages of such an environment will be discussed and analyzed. If possible, a small evaluation based on practical experience will also be part of the thesis.
\subsection{Research status}
The educational resources about microservice are widely available. (A search on GoogleScholar shows about 14k results). While the used tools consistently evolve, the basic concept is not considered new anymore. A report published in 2020 by Mike Loukides and Steve Swoyer show that over 75\% of the companies surveyed have adopted some microservice tools in the last five years.\cite{msadoption}
On the other hand, technical journals about the use of these tools in developments are rather rare. Papers that exist focus primarily on the \acl*{CI} process. There are some internet blog entries with exemplary web-development setups using Docker. However, the actual progression in platform independent development environment setups are primarily driven by real practical applications and are not currently covert by scientific journals. Companies like Docker Inc. and GitHub/Microsoft are working products based on the concept of virtualized development environments with products like Dev-Containers and CodeSpaces. These allow users to quickly checkout code from any location, even within a browser, to test and apply changes to their applications.
\subsection{Concept}
The increasing demand and complexity of software, alongside with more frequent feature requests in the software industry, resulted in creation and adoption of microservices. This trend continues with the broader availability of containerization solutions and tools. Containerization allows software to be independent of the host environment that runs the code. The container already ships with a pre- and well-defined runtime environment and therefore decouples the link between host and software. Multiple instances of a software can be deployed quickly and on demand without the need of tedious, time consuming and error-prone host configuration. However, these advantages bring also new challenges. Monolithic applications are developed and tested as one unit, components are tightly coupled so that errors in between components can be found quickly. Microservices are often orchestrated and provide together one extensive application. Changes in one service may break other parts of the application, which only becomes apparent during operation in the overall orchestrated system. Continuous Integration and Continuous Deployment are two concepts, which are increasingly adopted, that try to solve this problem. The creation and practice of these concepts are bundled under the principles of DevOps culture. While there are already comprehensive guides and materials for implementing \ac{CI} and \ac{CD} in DevOps, these focus mostly on the process of testing and delivery. These resources fall short in the management of the development environment.\newline
Development environments are highly heterogeneous. Developers use different operating systems in various configurations, multiple versions of programming languages and their personal preferred tools. Further development environments have additional requirements like usage of demo data, source-maps, debug-symbols and opt-out security features. It is not uncommon that configuring such an environment time consuming. The fact that these development environments differ to the \ac{CI}/\ac{CD} builds may result in unexpected behavior. This bachelor thesis introduces the idea of using the same containerization approach and \ac{CI}/\ac{CD} pipeline to build well-defined development environments and evaluates the idea based on a practical example project.\newline
The process of describing, building, applying and discussing the usability of such a development environment will be the concept of this bachelor thesis. The concept is illustrated by means of an exemplary practical project, which illustrates potential problems and challenges.
\subsection{Motivation}
Development environments tend to be complex especially for larger projects. Many projects require a specific version of a compiler, interpreter, library and other parts of the toolchain. The specifics on how to set up these environments differ based on the (flavor of the) operating system. However, when developing on multiple projects, different versions of compiler, interpreter and library are often required which can lead to conflicts. The existence of package manager tools like \ac{python-venv} and \ac{nvm}, that try so solve the problem of incompatible libraries, poofs that there is a need for dedicated clean project environments.\newline
Finally, setting up such an environment takes a lot of time and is a challenge, especially for new members of a project team. It may also act as an entry barrier in open source projects.\newline
Containerization is an approach to make applications independent of the host system and prevents configuration based errors. These concepts are promising to reduces the setup time of an application as well as to eliminate the before mentioned configuration errors. The approach of development containers makes it possible to quickly create a working development environment anywhere, even on a different host. The only thing the developer needs is an editor and a connection to that host. Services like codesandbox.io and GitHubs CodeSpaces are some of the first to implement these concepts for small projects.
\newpage
\section{Outline draft}
This is a first draft of the theses outline.

\begin{enumerate}
    \item Introduction
    \begin{enumerate}
        \item Scope of the thesis
        \item Goal of the theses
        \item Structural overview
    \end{enumerate}
    \item Container and DevOps Basics
    \begin{enumerate}
        \item What is DevOps
        \item Container and Microservice basics
        \item \acl{CI} and \acl{CD} concepts
        \item Containerization in development environments
    \end{enumerate}
    \item Concept of DevContainers
    \begin{enumerate}
        \item Pre requirements for DevContainers
        \item Description of a conceptual environment
        \item Available tools and resources
        \item Possible implementations approach's
        \item Strengths, weaknesses and limits
    \end{enumerate}
    \item Exemplary implementation process
    \begin{enumerate}
        \item Current state and goal
        \item Implementation approach
        \item The implementation process
        \item Encountered challenges and limits
        \item Final state
    \end{enumerate}
    \item Performance evaluation and analysis
    \begin{enumerate}
        \item Metrics and what to evaluate
        \item Evaluation and results
        \item Discussion of evaluation
    \end{enumerate}
    \item Future potential and outlook
    \item Conclusion
\end{enumerate}

\section{Schedule}
\begin{table}[H]
    \centering
    \begin{tabular}{ l p{14cm} }
    \hline
    Period & Task \\ \hline\hline
    07.06.2021 & Further literature research\\ \hline
    18.06.2021 & Survey of the initial condition\\ \hline
    30.06.2021 & Implementation of the DevContainer setup\\ \hline
    31.07.2021 & Extensive testing of the new environment and deployment at the company\\ \hline
    15.08.2021 & Finalize the written part up to and including the exemplary implementation processes.\\ \hline
    31.08.2021 & Survey on practicability and satisfaction new development environment and writing of the evaluation \& conclusion \\ \hline
    17.09.2021 & Proofreading, printing and delivery\\ \hline
    \end{tabular}
    \caption{Approximate schedule}
    \label{tab::time}
    \end{table}
\section{Literatur}
\bibliographystyle{IEEEtranSA}
\bibliography{bib/sources}
\newpage

% Anhang
\renewcommand{\thesubsection}{\Alph{subsection}}
\pagenumbering{Roman}
\setcounter{page}{\value{lastroman}}
\section*{Appendix}
\addcontentsline{toc}{section}{Appendix}
% \listoffigures
%Abkürzungsverzeichnis
% !TeX root = ../main.tex
\newcommand{\abbr}{Abbreviations}
\subsection{Abbreviations}
%\addcontentsline{toc}{subsection}{Abbreviations}

\begin{acronym}[1234567890ABC]		%[längste Abkürzung]
\setlength{\itemsep}{-\parsep}	% sorgt dafür, dass das Verzeichnis kompakt dargestellt wird.

\acro{CI}[CI]{Continuous Integration}
\acro{CD}[CD]{Continuous Delivery}
\acro{IDE}[IDE]{Integrated Development Environment }
\acro{nvm}[nvm]{NodeJS Version Manager}
\acro{python-venv}[python-venv]{python virtual environments}

\end{acronym}
\include{inc/ensure.inc}

\end{document}
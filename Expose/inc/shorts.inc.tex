% !TeX root = ../main.tex
\newcommand{\abbr}{Abbreviations}
\subsection{Abbreviations}
%\addcontentsline{toc}{subsection}{Abbreviations}

\begin{acronym}[1234567890]		%[längste Abkürzung]
\setlength{\itemsep}{-\parsep}	% sorgt dafür, dass das Verzeichnis kompakt dargestellt wird.

\acro{BLEST}[BLEST]{BLock ESTimation}
\acro{CWND}[CWND]{Congestion Window}
\acro{DAPS}[DAPS]{Delay-Aware Packet Scheduler}
\acro{ECF}[ECF]{Earliest Completion First}
\acro{HTTP}[HTTP]{Hypertext Transfer Protocol}
\acro{ISP}[ISP]{Internet Service Provider}
\acro{IETF}[IETF]{Internet Engineering Task Force}
\acro{LRF}[LRF]{Lowest-RTT-First}
\acro{MSS}[MSS]{Maximum Segment Size}
\acro{MPTCP}[MPTCP]{Multipath TCP}
\acro{MPTCPSW}[MPTCP\textsubscript{SW}]{MPTCP's send window}
\acro{OTIAS}[OTIAS]{Out-of-Order Transmission for In-Order Arrival Scheduler}
\acro{RR}[RR]{Round Robin}
\acro{RTT}[RTT]{Round Trip Time}
\acro{SCTP}[SCTP]{Stream Control Transmission Protocol}
\acro{sRTT}[sRTT]{Smoothed RTT}
\acro{STTF}[STTF]{Shortest Transfer Time First}
\acro{TCP}[TCP]{Transmission Control Protocol}
\acro{VoIP}[VoIP]{Voice over IP}
\end{acronym}